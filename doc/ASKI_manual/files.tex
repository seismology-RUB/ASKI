%-----------------------------------------------------------------------------
%   Copyright 2013 Florian Schumacher
%
%   This file is part of the ASKI manual as a LaTeX document with main file
%   manual.tex
%
%   Permission is granted to copy, distribute and/or modify this document
%   under the terms of the GNU Free Documentation License, Version 1.3
%   or any later version published by the Free Software Foundation;
%   with no Invariant Sections, no Front-Cover Texts, and no Back-Cover Texts.
%   A copy of the license is included in the section entitled ``GNU
%   Free Documentation License''. 
%-----------------------------------------------------------------------------
%
This chapter collects documentation on file formats involved in \ASKI. 
%
%++++++++++++++++++++++++++++++++++++++++++++++++++++++++++
\section{Parameter Files} \label{files,sec:parfiles}
%++++++++++++++++++++++++++++++++++++++++++++++++++++++++++
%
Parameter files are simple text files.

The following type of lines are ignored:
\begin{itemize}
\item comment lines, i.e.\ lines STARTING with an arbitrary number of blanks followed by a ``\#'' character
\item empty lines and lines containing blanks only
\item lines not containing any ``='' character
\end{itemize}

How to specify one parameter:
\begin{itemize}
\item valid lines have the form ``keyword = value'' (blanks leading or following ``keyword'', ``='', or ``value'' are ignored)
\item in a valid line, all characters in front of ``='' (without leading and appending blanks)
are interpreted as the keyword, allowing for blank characters within the keyword (e.g.\ for lines  
\mbox{``\hspace{5mm}key~word~=~value\hspace{3mm}''}, the string ``key~word'' is used as the keyword)
\item all characters behind ``='' (without leading and appending blanks) are interpreted as the value string from 
which the value is read, which in particular means that ``\#'' comments at the end of a line (such as 
\mbox{``\hspace{3mm}keyword = value\hspace{2mm}\# comment\hspace{3mm}''}) are \emph{not} allowed!
\end{itemize}
By convention, specify \emph{paths} (i.e.\ directory names, which will be concatenated with a filename of 
a file in that directory) always ending on ``/'' and specify \emph{filenames} always \emph{without} leading ``/''.
%
%----------------------------------------------------------
\subsection{Main Parameter File} \label{files,sec:main_parfile}
%----------------------------------------------------------
%
Here, shortly all keywords required in the main parameter file for your specific program operation, are described.
%- - - - - - - - - - - - - - - - - - - - - - - - - - - - - 
\subsubsection{\lcode{FORWARD_METHOD}} \label{files,sec:main_parfile,itm:forward_method}
\begin{itemize}
\item[] \lcode{GEMINI}
\item[] \lcode{SPECFEM3D} $\rightarrow$ \lcode{SPECFEM3D_Cartesian} , \lcode{SPECFEM3D_GLOBE}
\end{itemize}
For details on the methods and references to their documentation, refer to section~\ref{basic_steps,sec:forward_problem}
%- - - - - - - - - - - - - - - - - - - - - - - - - - - - -
\subsubsection{\lcode{MODEL_PARAMETRIZATION}} \label{files,sec:main_parfile,itm:mod_pmtrz}
\begin{itemize}
\item[] \lcode{isoLame} $\rightarrow$ isotropic Lam\´e parameters $\rho$, $\lambda$, $\mu$
\item[] \lcode{isoVelocity} $\rightarrow$ isotropic seismic velocities $\rho$, $v_p$, $v_s$
\end{itemize}
Nothing else supported yet
%- - - - - - - - - - - - - - - - - - - - - - - - - - - - -
\subsubsection{\lcode{MAIN_PATH_INVERSION}} \label{files,sec:main_parfile,itm:main_path}
All subpaths for filenames are considered relative to this main path. This
directory is thought to contain all your relevant output and (temporary) data.\\
Example: \lcode{MAIN_PATH_INVERSION = /scratch/inversions/Aegean1/}
%- - - - - - - - - - - - - - - - - - - - - - - - - - - - - 
\subsubsection{\lcode{CURRENT_ITERATION_STEP}} \label{files,sec:main_parfile,itm:cur_iter_step}
Example: \lcode{CURRENT_ITERATION_STEP = 3}
%- - - - - - - - - - - - - - - - - - - - - - - - - - - - - 
\subsubsection{\lcode{ITERATION_STEP_PATH}} \label{files,sec:main_parfile,itm:iter_path}
Relative to main path, defining name of subdirectory of \lcode{MAIN_PATH_INVERSION} which contains 
all relevant (meta)data of an inversion step. A three-digit integer (= \lcode{CURRENT_ITERATION_STEP}) 
and ``/'' will be appended to \lcode{ITERATION_STEP_PATH} (i.e.\ ``001/'', ``002/'', \dots) defining the 
first, second \dots iteration step directory.\\
Example: \lcode{ITERATION_STEP_PATH = iteration_step_}
%- - - - - - - - - - - - - - - - - - - - - - - - - - - - - 
\subsubsection{\lcode{PARFILE_ITERATION_STEP}} \label{files,sec:main_parfile,itm:iter_parfile}
File name of iteration step specific parameter file, relative to \lcode{MAIN_PATH_INVERSION/ITERATION_STEP_PATH}
Example: \lcode{PARFILE_ITERATION_STEP = iter_parfile}
%- - - - - - - - - - - - - - - - - - - - - - - - - - - - - 
\subsubsection{\lcode{PATH_MEASURED_DATA,PATH_EVENT_FILTER,PATH_STATION_FILTER}} 
\label{files,sec:main_parfile,itm:path_mdata_filters}
Paths where \ASKI finds files related to the measured data files. These paths can be everywhere, e.g.\ close to where you 
have stored/processed your (time domain) data, or in directory \lcode{MAIN_PATH_INVERSION}, etc.\ \dots
The naming convention of files in these directories is:\\
\lcode{FILE_MEASURED_DATA}: \lcode{data_EVENTID_STATIONNAME_COMP},
\lcode{FILE_EVENT_FILTER}: \lcode{filter_EVENTID},
\lcode{FILE_STATION_FILTER}: \lcode{filter_STATIONNAME_COMP}, 
where filters are dependet on component and \lcode{STATIONNAME} and \lcode{EVENTID} are 
defined in \lcode{FILE_STATION_LIST} and \lcode{FILE_EVENT_LIST} file, and \lcode{COMP} 
is a valid component supported by module \lcode{componentTransformation}\\
Example: \\
\lcode{PATH_MEASURED_DATA = /mydata/your_name_of_inversion/ASKI_data/}\\
\lcode{PATH_EVENT_FILTER = /mydata/your_name_of_inversion/ASKI_event_filter/}\\
\lcode{PATH_STATION_FILTER = /mydata/your_name_of_inversion/ASKI_station_filter/}
%- - - - - - - - - - - - - - - - - - - - - - - - - - - - - 
\subsubsection{\lcode{FILE_EVENT_LIST}} \label{files,sec:main_parfile,itm:file_event_list}
Absolute filename where ASKI finds a text file defining a set of events in the required format 
(\ref{files,sec:event_list})\\
Example: \lcode{FILE_EVENT_LIST = /mydata/your_name_of_inversion/ASKI_events}
%- - - - - - - - - - - - - - - - - - - - - - - - - - - - - 
\subsubsection{\lcode{FILE_STATION_LIST}} \label{files,sec:main_parfile,itm:file_station_list}
Absolute filename where ASKI finds a text file defining a set of stations in the required format 
(\ref{files,sec:station_list})\\
Example: \lcode{FILE_STATION_LIST = /mydata/your_name_of_inversion/ASKI_stations}
%- - - - - - - - - - - - - - - - - - - - - - - - - - - - - 
\subsubsection{\lcode{MEASURED_DATA_FREQUENCY_STEP,MEASURED_DATA_NUMBER_OF_FREQ,MEASURED_DATA_INDEX_OF_FREQ}} 
\label{files,sec:main_parfile,itm:mdata_freq}
Discretized frequency window of measured data (same expected in event\_filter/station\_filter!) given by a frequency 
step \lcode{FREQUENCY_STEP} [Hz] and a vector of frequency indices \lcode{INDEX_OF_FREQ}
(of length \lcode{NUMBER_OF_FREQ}), where for specific frequency index $i$ the corresponding frequency $f_i$ [Hz] 
computes to $f_i = i \cdot$ \lcode{FREQUENCY_STEP}\\
Example:\\
\lcode{MEASURED_FREQUENCY_STEP = 10.}\\
\lcode{MEASURED_NUMBER_OF_FREQ = 5}\\
\lcode{MEASURED_INDEX_OF_FREQ = 2 3 5 7 10}\\
which corresponds to the 5 frequencies $20,30,50,70,100$ Hz
%- - - - - - - - - - - - - - - - - - - - - - - - - - - - - 
\subsubsection{\lcode{DEFAULT_VTK_FILE_FORMAT}} 
Either \lcode{BINARY} or \lcode{ASCII} defining the default type of \lcode{vtk} files 
which will be produced in the course of running the programs.
%
%----------------------------------------------------------
\subsection{Parameter File for Specific Iteration Step} \label{files,sec:iter_parfile}
%----------------------------------------------------------
%
Here, shortly all keywords required in a parameter file for a specific iteration step, i.e.\ \\
\lcode{MAIN_PATH_INVERSION/ITERATION_STEP_PATH/PARFILE_ITERATION_STEP} , are described.
%- - - - - - - - - - - - - - - - - - - - - - - - - - - - - 
\subsubsection{\lcode{ITERATION_STEP_NUMBER_OF_FREQ,ITERATION_STEP_INDEX_OF_FREQ}} \label{files,sec:iter_parfile,itm:iter_freq}
Frequency discretization of this iteration step, must be a subset of global frequency discretization 
for this inversion defined as defined by \ref{files,sec:main_parfile,itm:mdata_freq}. \\
\lcode{ITERATION_STEP_NUMBER_OF_FREQ <= MEASURED_DATA_NUMBER_OF_FREQ} and vector \\
\lcode{ITERATION_STEP_INDEX_OF_FREQ} (of length \lcode{ITERATION_STEP_NUMBER_OF_FREQ})
must only contain indices contained in \lcode{MEASURED_DATA_INDEX_OF_FREQ}\\
All indices here are assumed in accordance with the global frequency step \lcode{MEASURED_DATA_FREQUENCY_STEP}
%- - - - - - - - - - - - - - - - - - - - - - - - - - - - - 
\subsubsection{\lcode{TYPE_INVERSION_GRID,PARFILE_INVERSION_GRID}} \label{files,sec:iter_parfile,itm:invgrid}
Type of inversion grid (as supported, cf.\ \ref{basic_steps,sec:invgrid}) and corresponding
filename of parameter file defining this inversion grid, relative to \\
\lcode{MAIN_PATH_INVERSION/ITERATION_STEP_PATH/}
%- - - - - - - - - - - - - - - - - - - - - - - - - - - - - 
\subsubsection{\lcode{TYPE_INTEGRATION_WEIGHTS}}
Type of integration weights (integer number), cf.\ \ref{basic_steps,sec:intw} for supported values.
%- - - - - - - - - - - - - - - - - - - - - - - - - - - - - 
\subsubsection{\lcode{FILE_INTEGRATION_WEIGHTS}} 
Filename of the integration weights file, which will be created and used, relative to 
\lcode{MAIN_PATH_INVERSION}. 
%- - - - - - - - - - - - - - - - - - - - - - - - - - - - -
\subsubsection{\lcode{FILE_WAVEFIELD_POINTS}} 
Filename of the wavefield points file, relative to \lcode{MAIN_PATH_INVERSION}, which
is in general created by the method you are using. Just refer here to this file. 
%- - - - - - - - - - - - - - - - - - - - - - - - - - - - -
\subsubsection{\lcode{FILE_KERNEL_REFERENCE_MODEL}} \label{files,sec:iter_parfile,itm:model}
Dependent on the method you are using, these filenames may be handled individually. Please refer to the respective 
documentation of the methods for recommendations how to use these parameters, or which naming to choose.
%- - - - - - - - - - - - - - - - - - - - - - - - - - - - - 
\subsubsection{\lcode{FILEBASE_BASIC_STATS}}
Base filename of vtk stats output files (related to inversion grid, wavefield points, integration weights,
events, stations), relative to \lcode{MAIN_PATH_INVERSION/ITERATION_STEP_PATH}. 
%- - - - - - - - - - - - - - - - - - - - - - - - - - - - - 
\subsubsection{\lcode{PATH_OUTPUT_FILES}}
Folder relative to which some sensitivity analysis and inversion programs write their output (relatively small output
like models, coefficients etc., NO wavefields/kernels etc.!), relative to 
\lcode{MAIN_PATH_INVERSION/ITERATION_STEP_PATH}.  Be sure, the path ends on ``/".
%- - - - - - - - - - - - - - - - - - - - - - - - - - - - -
\subsubsection{\lcode{PATH_KERNEL_DISPLACEMENTS}} 
Subdirectory of current iteration step path
\lcode{MAIN_PATH_INVERSION/ITERATION_STEP_PATH} which contains the 
kernel displacement files. Be sure, the path ends on ``/".
%- - - - - - - - - - - - - - - - - - - - - - - - - - - - -
\subsubsection{\lcode{PATH_KERNEL_GREEN_TENSORS}} 
Subdirectory of current iteration step path
\lcode{MAIN_PATH_INVERSION/ITERATION_STEP_PATH} which contains the 
kernel green tensor files. Be sure, the path ends on ``/".
%- - - - - - - - - - - - - - - - - - - - - - - - - - - - -
\subsubsection{\lcode{PATH_SENSITIVITY_KERNELS}} 
Subdirectory of current iteration step path
\lcode{MAIN_PATH_INVERSION/ITERATION_STEP_PATH} which contains the 
velocity kernel files. Be sure, the path ends on ``/".
%- - - - - - - - - - - - - - - - - - - - - - - - - - - - -
\subsubsection{\lcode{PATH_SYNTHETIC_DATA}} 
Subdirectory of current iteration step path
\lcode{MAIN_PATH_INVERSION/ITERATION_STEP_PATH} which contains the 
files with synthetic data. Be sure, the path ends on ``/".

%
%++++++++++++++++++++++++++++++++++++++++++++++++++++++++++
\section{Event List File} \label{files,sec:event_list}
%++++++++++++++++++++++++++++++++++++++++++++++++++++++++++
%
Please find the template event list file \lcode{template/file_event_list_template}.

\begin{itemize}
\item first line contains single character ``C'' or ``S'', defining the coordinate system (``C''artesian or ``S''pherical)
  with respect to which the given event coordinates lat,lon are interpreted
\item each following non-empty line of the file is interpreted as a definition of one event and must 
  contain the following space-separated values:
  \begin{itemize}
  \item[eventid]  13 character name, (e.g. \lcode{2006.10.2977} or \lcode{061113_141238}) 
    should \emph{not} contain whitespace!
  \item[origintime]  characters of form \lcode{yyyymmdd_hhmmss_nnnnnnnnn} or \lcode{yyyymmdd_hhmmss}
    (i.e.\ with or without nano-seconds), e.g. \lcode{20130320_170012} or 
    \lcode{20130320_170002_718000000}
  \item[lat] latitude in degrees, \lcode{-90 <= lat <= 90} (``S'') or first coordinate in 
    wavefield points / inversion grid - frame (``C'') $\rightarrow$  read the section on inversion grid definitions 
    (\ref{basic_steps,sec:invgrid})
  \item[lon] longitude in degrees, \lcode{0 <= lon <= 360} (``S'') or second  coordinate in 
    wavefield points / inversion grid -frame (``C'') (read \ref{basic_steps,sec:invgrid})
  \item[depth] source depth in km (``S''), or third coordinate in wavefield points / inversion grid -frame (``C'') 
    (read \ref{basic_steps,sec:invgrid})
  \item[typ] source type:  0 = force, 1 = moment tensor, -1: not specified
  \item[mag] factor on source mechanism
  \item[mom/frce] either 3 values (force vector) or 6 values (moment tensor)
  \end{itemize}
\end{itemize}
%
%++++++++++++++++++++++++++++++++++++++++++++++++++++++++++
\section{Station List File} \label{files,sec:station_list}
%++++++++++++++++++++++++++++++++++++++++++++++++++++++++++
%
Please find the template station list file \lcode{template/file_station_list_template}.

\begin{itemize}
\item first line contains single character ``C'' or ``S'', defining the coordinate system (``C''artesian or ``S''pherical)
  with respect to which the given event coordinates lat,lon are interpreted
\item each following non-empty line of the file is interpreted as a definition of one station and must 
  contain the following space-separated values:
  \begin{itemize}
  \item[station\_name] 5 character name, which should \emph{neither} contain whitespace \emph{nor} underscors ``\_''!
  \item[network\_code] 6 character network code
  \item[lat] latitude in degrees, \lcode{-90 <= lat <= 90} (``S'') or first coordinate in 
    wavefield points / inversion grid - frame (``C'') $\rightarrow$  read the manual on inversion grid definitions 
    (\ref{basic_steps,sec:invgrid})
  \item[lon] longitude in degrees, \lcode{0 <= lon <= 360} (``S'') or second  coordinate in 
    wavefield points / inversion grid -frame (``C'') (read \ref{basic_steps,sec:invgrid})
  \item[elevation] altitude of station (``S''), or third coordinate in wavefield points / inversion grid -frame (``C'') (read \ref{basic_steps,sec:invgrid})
  \end{itemize}
\end{itemize}
%
%++++++++++++++++++++++++++++++++++++++++++++++++++++++++++
\section{Measured Data Files} \label{files,sec:measured_data}
%++++++++++++++++++++++++++++++++++++++++++++++++++++++++++
%
All measured data files are expected to be in the directory \lcode{PATH_MEASURED_DATA} as defined
in the main parameter file.

One measured data file contains all data values for one specific receiver component and a specific event.
Its filename is by convention \lcode{data_EVENTID_STATIONNAME_COMP}

The files are text files containing $1$ column of \lcode{MEASURED_DATA_NUMBER_OF_FREQ} complex numbers, 
which can be understood by \lcode{FORTRAN} \lcode{read} command.

Line \lcode{i} contains measured data values for the \lcode{i}\textsuperscript{th} frequency, as defined by vector
of indices \lcode{MEASURED_DATA_INDEX_OF_FREQ} and frequency step \lcode{MEASURED_DATA_FREQUENCY_STEP}. \\
In particular, this means that \emph{all} measured data files must contain the \emph{same} frequency discretization, given
by parameters \lcode{MEASURED_DATA_INDEX_OF_FREQ}, \lcode{MEASURED_DATA_FREQUENCY_STEP} of the main 
parameter file.
%
%++++++++++++++++++++++++++++++++++++++++++++++++++++++++++
\section{Synthetic Data Files} \label{files,sec:synth_data}
%++++++++++++++++++++++++++++++++++++++++++++++++++++++++++
%
All synthetic data files are expected to be in the directory \lcode{PATH_SYNTHETIC_DATA} as defined 
in the parameter file of the current iteration step.

One synthetic data file contains the complete synthetic data values for one specific path (i.e.\ a specific
source-receiver combination). Its filename is by convention \lcode{synthetics_EVENTID_STATIONNAME}

The files are text files containing \lcode{ITERATION_STEP_NUMBER_OF_FREQ} lines and $3$ columns of 
complex numbers, which can be understood by \lcode{FORTRAN} \lcode{read} command:\\
Line \lcode{i} contains synthetic data values for the \lcode{i}\textsuperscript{th} frequency, as defined by vector
of indices \lcode{ITERATION_STEP_INDEX_OF_FREQ} and frequency step \lcode{MEASURED_DATA_FREQUENCY_STEP}.
The $3$ complex numbers on a line refer to the $3$ Cartesian components \lcode{CX}, \lcode{CY}, \lcode{CX}.
%
%++++++++++++++++++++++++++++++++++++++++++++++++++++++++++
\section{Vtk Files} \label{files,sec:vtk_files}
%++++++++++++++++++++++++++++++++++++++++++++++++++++++++++
%
For visualization of basic objects of the inversion, such as the inversion grid, 
the wavefield points, the integration weights etc., as well as some inversion results
and models, we use the \lcode{vtk} file format.\\
General information on this file format may be found under \url{www.vtk.org/VTK/img/file-formats.pdf}

PUT HERE:\\
General info about the two types of vtk files (invgridVtk, wavefield points Vtk files)\\
Some basic content description about some special vtk files.
%
%++++++++++++++++++++++++++++++++++++++++++++++++++++++++++
\section{Data and Model Space File} \label{files,sec:dmspace}
%++++++++++++++++++++++++++++++++++++++++++++++++++++++++++
%
Files in which a data and model space is defined have the following form. Also have a look at 
example template files \lcode{template/data_model_space_info_template_*}

The blocks described in the subsections below should be put into a file, one after another. The header block must come
first, then the data space block and model parameter block. The order of the latter two is arbitrary, both orders are 
allowed, however if the model parameters block is defined first, an additional check for empty kernel file names will
be done in the processing of the data samples block.

%---------------------------------------------------------
\subsection{Header Block}
%---------------------------------------------------------
{\bf line 1}: currently ignored (file format version specification possible, header comment)

{\bf line 2}: must either contain \lcode{ASCII} or \lcode{BINARY}\\
currently ignored (possible form definition like ASCII / mixed ASCII/BINARY (similar to in vtk(?)))\\
At the moment, this file must be a formatted text file. 
%---------------------------------------------------------
\subsection{Model Parameters Block}
%---------------------------------------------------------

{\bf line}: \lcode{MODEL PARAMETERS}\\
this defines that the definition of the model parameters starts here.

{\bf line}: \lcode{INVERSION_GRID_CELLS value}\\
where \lcode{value} is either \lcode{ALL} (all inversion grid cells are taken) or \lcode{SPECIFIC} 
(specific definition of set of invgrid cells following below)

{\bf line}: \lcode{PARAMETERS value}\\
where \lcode{value} is either \lcode{ALL} (all inversion grid cells are taken) or \lcode{SPECIFIC} 
(specific definition of model parameters for each invgrid cell, following below. Only allowed if 
\lcode{INVERSION_GRID_CELLS SPECIFIC}). 

{\bf If \lcode{PARAMETERS ALL}, line}: \lcode{nparam pmtrization_1 param_1 ... pmtrization_n param_n}\\
defines the parametrization used for all inversion grid cells.

If \lcode{INVERSION_GRID_CELLS SPECIFIC}, the following line must contain the number of cells \lcode{ncell} 
which should be taken, followed by \lcode{ncell} blocks of lines, each defining an inversion grid cell. In case
of \lcode{PARAMETERS ALL}, these blocks consist of a single line line containing an inversion grid cell index.
In case of \lcode{PARAMETERS SPECIFIC}, these blocks consist of two lines: one line containing an inversion grid cell
index and an additional second line of the form\\
\lcode{nparam pmtrization_1 param_1 ... pmtrization_n param_n} \\
defining the parameters to be used for this specific inversion grid cell.

%---------------------------------------------------------
\subsection{Data Samples Block}
%---------------------------------------------------------
line ``DATA SAMPLES''
%
line of form: ``PATHS value'', where value is either ``ALL'' (all paths for a given set of event and station indices
are used), or ``SPECIFIC'' (a specific definition of paths as a series of event and station index pairs follows below)

If ``PATHS ALL'', the next two lines are of form ``nev iev\_1 ... iev\_n'' and ``nstat istat\_1 ... istat\_n'', defining the set
of event and station indices, which form (by all combinations) the used paths. 

line of form: ``COMPONENTS value'', where value is either ``ALL'' (for all paths, the same components are used) or ``SPECIFIC''
(only allowed if ``PATHS SPECIFIC'', for each path a specific set of components may be defined)

If ``COMPONENTS ALL'', the next line is of form ``ncomp comp\_1 ... comp\_n'' defining the component indices for all paths.

line of form: ``FREQUENCIES value'', where value is either ``ALL'' (for all paths, the same frequency indices are used) or ``SPECIFIC''
(only allowed if ``PATHS SPECIFIC'', for each path a specific set of frequency indices may be defined)

If ``FREQUENCIES ALL'', the next line is of form ``nfreq ifreq\_1 ... ifreq\_n'' defining the frequency indices for all paths.

line of form: ``IMRE value'', where value is either ``ALL'' (for all paths, the same set of imaginary/real parts are used) or ``SPECIFIC''
(only allowed if ``PATHS SPECIFIC'', for each path a specific set of imaginary/real parts may be defined)

If ``IMRE ALL'', the next line is of form ``nimre imre\_1 ... imre\_n'' defining imaginary (i.e. imre\_i = ``im'') or real parts 
(imre\_i = ``re'') for all paths.

If ``PATHS SPECIFIC'', the following line must contain the number ``npaths'' of paths which should be used, followed by 
npahts blocks of lines, each defining the path and the data samples for that path. \\
These blocks constist of at least one line containing the event /station index pair ``iev istat''. \\
For each keyword ``COMPONENTS'', ``FREQUENCIES'' and ``IMRE'' -- if ``SPECIFIC'' -- one line is added to such a block 
of lines, in the same form as the line following ``keyword ALL'' (see above), 
defining the specific components, frequencies or set of imaginary/real parts for each of the specific paths.
%
%++++++++++++++++++++++++++++++++++++++++++++++++++++++++++
\section{\lcodetitle{ecartInversionGrid} Files} \label{files,sec:ecart_invgrid}
%++++++++++++++++++++++++++++++++++++++++++++++++++++++++++
%
%---------------------------------------------------------
\subsection{Nodes Coordinates Files}
%---------------------------------------------------------
These files contain a collection of points in space, given in Cartesian \lcode{X}-, \lcode{Y}-, 
\lcode{Z}-coordinates. They must be text files and have the following format.

The first line contains a single integer value, indicating the number of lines to come (i.e.\ the 
number of points).\\
Each following line contains 3 floating point numbers (separated by white space) defining Cartesian 
\lcode{X}-, \lcode{Y}-, \lcode{Z}-coordinates of a point.

%---------------------------------------------------------
\subsection{Cell Connectivity Files}
%---------------------------------------------------------
These files contain the definition of cells, based on points as defined in the nodes coordinates files.
They must be text files and have the following format.

The first line contains a single integer value, indicating the number of lines to come (i.e.\ the 
number of cells).\\
Each following line contains \lcode{n} integer numbers (separated by white space, \lcode{n = 4} 
in case of tet4-type cells, \lcode{n = 8} in case of hex8-type cells), which define the control nodes 
of the cell and correspond to the point indices in the respective nodes coordinates file, whereby the 
lowest point index is 1, corresponding to the second line (first point) in the nodes coordinates file.

The order of the point indices in a line is assumed to correspond to the vtk cell conventions!
In case one of the cell connectivity files not existing, or their first line containing value 0, no cells
of the respective type will be created.

%---------------------------------------------------------
\subsection{Cell Neighbours File}
%---------------------------------------------------------
The terminology ``lines'' below refers to the case of this file not being binary, but a text file. 
In case of this file being binary, the file content is expected value by value as on the rows of the
text file. It will be opened by \lcode{FORTRAN} code with attribute \lcode{access='stream'}
(i.e.\ expecting the values as a simple byte stream) and expects integer values of \lcode.

The first line contains the total number of inversion grid cells \lcode{ncell}.\\
The next \lcode{ncell} lines (one for each cell in order of the cell index) are of the form:\\
\lcode{nnb icell_1 ... icell_nnb}\\
whereby \lcode{nnb} is the number of neighbours of the respective cell (must be \lcode{0} if 
no neighbours) followed by \lcode{nnb} cell indices \lcode{icell_1 ... icell_nnb}, defining 
the neighbour cells, if there are any neighbours.


%%% Local Variables:
%%% mode: latex
%%% TeX-master: "manual"
%%% End:
