% -*-LaTex-*-

%-----------------------------------------------------------------------------
%   Copyright 2016 Florian Schumacher
%
%   This file is part of the ASKI manual as a LaTeX document with main file
%   manual.tex
%
%   Permission is granted to copy, distribute and/or modify this document
%   under the terms of the GNU Free Documentation License, Version 1.3
%   or any later version published by the Free Software Foundation;
%   with no Invariant Sections, no Front-Cover Texts, and no Back-Cover Texts.
%   A copy of the license is included in the section entitled ``GNU
%   Free Documentation License''. 
%-----------------------------------------------------------------------------
%
%#########################################################################
% ATTENTION: THERE ARE STILL SEVERAL PROBLEMS TO COMPILE THIS DOCUMENT RESULTING
% IN A LOT OF WARNINGS. YOU PROBABLY NEED TO COMPILE THIS DOCUMENT IN MODE 
% ``nonstopmode'' by:
% 
% pdflatex \\nonstopmode\\input manual.tex
% bibtex manual
% pdflatex \\nonstopmode\\input manual.tex
% pdflatex \\nonstopmode\\input manual.tex
% pdflatex \\nonstopmode\\input manual.tex
% 
%#########################################################################



%#########################################################################
%%   TODO DOCUMENTATION:
%% 
%%  ->  put text boxes for ``simple solution'' and ``advanced options'', where applicable
%% 
%%  ->  possible name for successor program package:
%%         ANSI - ANalysis of Sensitivity and Inversion
%%      with regard to the common (however, not official) name ``ANSI'' of a group of 
%%      ASCII-based 8-byte character sets like Latin 1, UTF-8.
%% 
%#########################################################################


\documentclass[12pt,a4paper]{book}

\usepackage[english]{babel} %language selection
\selectlanguage{english}

\pagenumbering{arabic}

\usepackage[affil-it]{authblk}
\usepackage{times} % 'times new roman' script style

\usepackage{amsmath}
\usepackage{amssymb}
\usepackage[pdftex]{graphicx}

\usepackage[nodayofweek]{datetime}
\newdateformat{mydate}{\shortmonthname[\THEMONTH] \THEYEAR}
\newdateformat{myyear}{\THEYEAR}

% use package url with [obeyspaces] in order to correctly display \nolinkurl WITH spaces 
%(used in \newcommand{\lcode} below). As hyperref internally loads package url, you can pass
% option obeyspaces of package url to package hyperref as follows
\PassOptionsToPackage{obeyspaces}{url}\usepackage{hyperref}
%\hypersetup{colorlinks, 
%           citecolor=black,
%           filecolor=black,
%           linkcolor=black,
%           urlcolor=black,
%           bookmarksopen=true,
%           pdftex}
%\hfuzz = .6pt % avoid black boxes

% the following is an ugly solution of allowing line breaks in urls additionally after every normal 
% alphabetic character which (if \nolinkurl is used in \newcommand{\lcode} below) at all allows line 
% breaks of long routine names like 'transformToStandardCellInversionGrid', BUT of course also breaks
% any other term formatted by \lcode at any character, which is maybe not very nice.
%\let\origUrlBreaks\UrlBreaks
%\renewcommand*{\UrlBreaks}{\origUrlBreaks\do\a\do\b\do\c\do\d\do\e\do\f\do\g\do\h\do\i\do\j\do\k\do\l\do\m\do\n\do\o\do\p\do\q\do\r\do\s\do\t\do\u\do\v\do\w\do\x\do\y\do\z\do\A\do\B\do\C\do\D\do\E\do\F\do\G\do\H\do\I\do\J\do\K\do\L\do\M\do\N\do\O\do\P\do\Q\do\R\do\S\do\T\do\U\do\V\do\W\do\X\do\Y\do\Z}


%% POSSIBLE PACKAGES TO DISPLAY CODE
%%
%% package alltt: verbatim environment within which math is displayed correctly
%% usage: \begin{alltt}\end{alltt}
%\usepackage{alltt}
%%
%% package listings: provides environments to display code fragments (with a lot of special characters) in a more evolved fashion than verbatim (alltt)
%% only uncomment (both next lines), if used in \newcommand{\lcode} below
\usepackage{listings}
\lstset{basicstyle =\ttfamily}%\small}

\usepackage[paperwidth=21.0cm,paperheight=29.7cm, left=2.5cm,right=2.5cm,top=2.0cm,
            bottom=2.0cm,headheight=0in,footskip=1.0cm]{geometry}
%-------------------------------
%
% COMMANDS FOR IN-LINE PHRASES IN CODE-STYLE
%
%%% ttfamily does not properly support any special characters
%\newcommand{\lcode}[1]{ {\ttfamily #1 }}
%
%%% lstinline is a good solution, in general, but it makes problems in line breaks!
%\newcommand{\lcode}[1]{\lstinline[breaklines=true]$#1$}
%
%%% although there are no actual links, \nolinkurl uses the same font as lstinline (when \lstset{basicstyle =\ttfamily}), 
%%% but produces better line breaks!
\newcommand{\lcode}[1]{\nolinkurl{#1}}
%
%%% need \lcodetitle, since \nolinkurl in a title of a numerated (sub)section (not *) causes problems in bookmark 
%%% view in adobe reader (why?! what is the actual problem?), \lcodetitle, however, does NOT support stuff like '_' etc.
\newcommand{\lcodetitle}[1]{ {\ttfamily #1} }
%
%
\newcommand{\ASKI}{{\ttfamily ASKI}}
%
%
% OTHER NEW COMMANDS
%
\newcommand{\inotice}[1]{ \fbox{\parbox[t]{0.9\textwidth}{{\bf Important:} \\#1}} }
\newcommand{\notice}[1]{ \fbox{\parbox[t]{0.9\textwidth}{#1}} }
\newcommand{\myref}[1]{\ref{#1} (page~\pageref{#1})}
\newcommand{\myaref}[1]{$\rightarrow$~\ref{#1} (page~\pageref{#1})}
%
\newcommand{\vecthree}[3]{
  \begin{pmatrix}
    #1 \\ #2 \\ #3
  \end{pmatrix}
}
\newcommand{\weights}{$w_1,\dots,w_{n_c}$}
\newcommand{\weightsS}{$w^S_1,\dots,w^S_{n_c}$}
\newcommand{\wpG}{$\mathbf{x}_1,\dots,\mathbf{x}_{n_c}$}
\newcommand{\wpS}{$\mathbf{x}^S_1,\dots,\mathbf{x}^S_{n_c}$}
\newcommand{\RRR}{\mathbb{R}^3}
\newcommand{\Rd}{\mathbb{R}^d}
\newcommand{\R}[1]{$\mathbb{R}^{#1}$}
\newcommand{\brackr}[1]{\left( #1 \right)}
\newcommand{\brackg}[1]{\left\{ #1 \right\}}
%
%-------------------------------
%
% END OF PREAMBLE
%####################################################################
%
\begin{document}
\sloppy
%
\setlength{\parindent}{0em}
\setlength{\parskip}{0.5em}
% TeX's first attempt at breaking lines is performed without even trying hyphenation: 
% TeX sets its "tolerance" of line breaking oddities to the internal value \pretolerance
% an "infinite" tolerance is represented by the value 10000, but may lead to very bad line breaks indeed!
%\pretolerance=10000
%
%-------------------------------
% TITLE PAGE(s)
%
%-----------------------------------------------------------------------------
%   Copyright 2016 Florian Schumacher
%
%   This file is part of the ASKI manual as a LaTeX document with main file
%   manual.tex
%
%   Permission is granted to copy, distribute and/or modify this document
%   under the terms of the GNU Free Documentation License, Version 1.3
%   or any later version published by the Free Software Foundation;
%   with no Invariant Sections, no Front-Cover Texts, and no Back-Cover Texts.
%   A copy of the license is included in the section entitled ``GNU
%   Free Documentation License''. 
%-----------------------------------------------------------------------------
%


%#########
% classical titlepage using \maketitle

%\title{\thispagestyle{empty} \tt {\Huge ASKI} {\rm --} {\Huge A}{\large nalysis of} {\Huge S}{\large ensitivity \\ and } {\Huge\tt K}{\large ernel} {\Huge\tt I}{\large nversion} \\ \vspace*{1cm} Version 0.3 \\ User Manual}
%\title{\thispagestyle{empty} \tt {\Huge ASKI} {\rm --} version 0.3 \\ User Manual}

% without \usepackage[affil-it]{authblk} e.g.:
%\author{Florian Schumacher \thanks{\texttt{florian.schumacher@rub.de}; corresponding author} \and Wolfgang Friederich \thanks{\texttt{wolfgang.friederich@rub.de}}}
% WITH authblk:
%\author[1]{Florian Schumacher, Wolfgang Friederich}
%\author[1]{Florian Schumacher \thanks{\texttt{florian.schumacher@rub.de}; corresponding author}}
%\author[1]{Wolfgang Friederich \thanks{\texttt{wolfgang.friederich@rub.de}}}
%\affil[1]{Ruhr-Universit\"at Bochum} % for this you need \usepackage[affil-it]{authblk}

%\date{\today}
%\date{6.12.2004}
%\date{} % no date

%\maketitle
% END classical titlepage using \maketitle
%#########




%#########
% individual titlepage using \begin{titlepage},\end{titlepage}
\begin{titlepage}
\thispagestyle{empty}

  \begin{center}
    \tt \Huge ASKI
  \end{center}
  \vspace*{2cm}

  \begin{minipage}{0.5\textwidth}
    \begin{flushleft}
      \fontsize{20}{40} \selectfont
      {\tt {\Huge A}{\large nalysis of} \\ {\Huge S}{\large ensitivity and } \\ {\Huge\tt K}{\large ernel} \\ {\Huge\tt I}{\large nversion} }
    \end{flushleft}
  \end{minipage}
  \hfill
  \begin{minipage}{0.5\textwidth}
    \begin{flushright}
      {\fontsize{20}{40} \selectfont \tt {\Huge User Manual} \\  ASKI {\rm --} {\large version 1.2} \\ {\large \mydate \today} \\}
      %{\tt {\large Florian Schumacher \\Wolfgang Friederich} \\ {\small Ruhr-Universit\"at Bochum, Germany} }
      {\tt {\large Florian Schumacher} \\ {\small Ruhr-Universit\"at Bochum, Germany} }
    \end{flushright}
  \end{minipage}

\vspace*{3cm}
%\vfill

\begin{center}
  \setlength{\fboxsep}{0pt}%
  \setlength{\fboxrule}{2pt}%
  \fbox{\includegraphics[width=\textwidth]{images/title_time_kernel_analysis.jpg}}
\end{center}

\end{titlepage}
% END individual titlepage
%#########


%
%-------------------------------
% LICENSE
Copyright \copyright {\myyear \today} Florian Schumacher.
Permission is granted to copy, distribute and/or modify this document
under the terms of the GNU Free Documentation License, Version 1.3
or any later version published by the Free Software Foundation;
with no Invariant Sections, no Front-Cover Texts, and no Back-Cover Texts.
A copy of the license is included in the section entitled ``GNU
Free Documentation License''.

\vspace{1cm}

If you use \ASKI{} for your own research, please cite one of our papers \cite{Schumacher16}, 
or \cite{Schumacher16b}, as appropriate:

F.\ Schumacher, W.\ Friederich and S.\ Lamara, \\
"A flexible, extendable, modular and 
computationally efficient approach to scattering-integral-based seismic full waveform 
inversion", \\
\emph{Geophysical Journal International}, (February, 2016) 204 (2): 1100-1119\\
\url{http://dx.doi.org/10.1093/gji/ggv505}

Schumacher F, Friederich W.\\
"ASKI: A modular toolbox for scattering-integral-based seismic full waveform 
inversion and sensitivity analysis utilizing external forward codes".\\
\emph{SoftwareX} 5 (2016) 252--259,\\
\url{http://dx.doi.org/10.1016/j.softx.2016.10.005}

\vspace{1em}

This documentation was written in the hope that it will be useful to the user,
but it \emph{cannot be assured} that it is accurate in every respect or complete in any sense.
In fact, at some places \emph{this manual is work in progress}.\\
Please do not hesitate to report any inconsistencies by
opening (or adding to) an "issues" topic on \url{https://github.com/seismology-RUB/ASKI}
or to improve this documentation by incorporating your experiences with \ASKI{} 
and your personal experience of getting used to it (at best by modifying the source and issuing a pull request
on gitHub, in any case let us know about it! Thanks). 
When you have developed new \ASKI{} components or 
have modified existing once, please extend / modify the \ASKI{} documentation accoringly.

Furthermore, I am aware of the poor \LaTeX coding of this document (at the moment, \verb+\sloppy+ is used
at the beginning of the document to avoid overfull hboxes in many places). There is a lot of potential
to improve the document 
style, hence the readability of the manual as a whole, as well as the coding style of the 
particular \lcode{.tex} files. \emph{Please do not hesitate to improve!}

The \LaTeX source files and all related components of this document are available via\\
\url{https://github.com/seismology-RUB/ASKI}~, subdirectory \lcode{doc/ASKI_manual/} of the repository.
\begin{flushright}
Florian Schumacher, \mydate \today
\end{flushright}
%
%-------------------------------
% TABLE OF CONTENTS
\tableofcontents
%
%-------------------------------
% CHAPTER Introduction
\chapter*{What is \ASKI{}?} \label{guide,sec:ASKI}
\phantomsection  % so hyperref creates bookmarks
\addcontentsline{toc}{chapter}{What is \ASKI?}
% -*-LaTex-*-

%-----------------------------------------------------------------------------
%   Copyright 2016 Florian Schumacher (Ruhr-Universitaet Bochum, Germany)
%
%   This file is part of the ASKI manual as a LaTeX document with main file
%   manual.tex
%
%   Permission is granted to copy, distribute and/or modify this document
%   under the terms of the GNU Free Documentation License, Version 1.3
%   or any later version published by the Free Software Foundation;
%   with no Invariant Sections, no Front-Cover Texts, and no Back-Cover Texts.
%   A copy of the license is included in the section entitled ``GNU
%   Free Documentation License''. 
%-----------------------------------------------------------------------------
%
\ASKI{} is a modularized software package written by the main authors Florian Schumacher and
Wolfgang Friederich (Ruhr-University Bochum, Germany). It offers sensitivity and regularization 
analysis tools for seismic datasets
as well as a scattering-integral-type full waveform inversion (FWI) concept based on waveform sensitivity 
kernels (i.e.\ Fr\'echet derivatives of seismic waveform data functionals) derived from Born scattering theory,
having Gauss-Newton convergence properties.
This inversion concept is presented in our paper \cite{Schumacher16} and in Florian's doctoral
dissertation \cite{_743d334d-dfa4-4a16-8cc5-91cdadc95271}.

\ASKI{} does not implement an intrinsic code for simulation of seismic wave propagation in order to 
solve the forward problem, but instead provides generalized interfaces to different external wave 
propagation codes. At the moment, the 1D semi-analytical code Gemini \cite{friederich_wd1995} and the 3D spectral 
element code SPECFEM3D \cite{TrKoLi08} are supported in \emph{both}, Cartesian and spherical framework. 
Additionally \ASKI{} supports the 3D nodal discontinuous Galerkin code NEXD \cite{Lambrecht.2015}
and plan to implement support for the Finite Difference code SOFI \cite{bohlen2002parallel}
in the near future. 
Please do not hesitate to add your own forward code, we are happy to support you during that process.

The resulting very modular concept of \ASKI{} allows us to keep the spatial description of the inverted 
models completely independent of the spatial description of the model used for solving the forward 
problem. This way, the overall iverse problem can be approached and regularized more physically compared
with using the forward grid also for inversion. The normally very fine spatial discretization of the model
domain is employed by the forward solver. This discretiation is usually weigh too fine to be used
directly for inversion since usually seismic data can only resolve much coarser structures and the
resolution also varies throughout the model domain. 
The separation concept of \ASKI{} \emph{naturally} alows to choose the spatial model
resolution heterogeneously and \emph{anew} in each iteration of FWI.

Instead of using time-dependent values of ground motion (i.e.\ samples of a time-series of seismic data), 
\ASKI{} uses freqency-dependent complex values of ground motion, recorded at a certain receivers excited 
by a certain seismic sources. This, mainly, has reasons of computational feasibility.
For time-domain forward methods, we implement an on-the-fly Fourier Transform in order to provide the
required spectral waveforms. 

Using sensitivity kernels $K$, data residuals $\delta d_i$ are connected to model uptdates $\delta m$ by 
an integral relation $\delta d_i = \int_{\text{earth}} \delta \vec{m} \cdot \vec{K}_i$. In order to build a linear 
system, the model update $\delta \vec{m}$ is assumed to be constant throughout small scattering volumes 
$\Omega_j$, where $\Omega = \overset{\centerdot}{\bigcup}_j \Omega_j$. These volumes constitute the cells of the volumetric 
inversion grid and the sensitivity matrix contains entries of preintegrated kernels $\int_{\Omega_j} K_i$.

The sensitivity kernels $K$ are computed from forward wavefields and strains produced on a set of points in 
space, which
is dependent on the particular forward method. This set of points is refered to as \emph{wavefield points}. The 
wavefields are written to file, by the respective forward method, which may require large discspace. 
Providing methods for constructing quadrature rules for arbitrary point sets contained in 
arbitrary volumes, 
\ASKI{} computes integration weights for integration of functions on the wavefield points over the volumetric
cells of the inversion grid. The inversion grid takes care of the localization of wavefield points 
inside the inversion grid cells and, if requested, the transformation of cells to a hexahedral (or tetrahedral) 
standard 
cell for the computation of the integration weights. Hence, some combinations of wavefield points (i.e.\ 
forward methods), integration weight types and inversion grid types are not possible. 

The preintegrated kernel values are also written to files, which may be flexibly read in for arbitrary 
subsets of data
by the independent binary programs conducting any sensitivity analysis 
or an interation step in the iterative full waveform inversion.
Those tools work on the sensitivity matrix, which in the FWI is used in a linear system of equations which
relates data residua $\delta d_i$ to a model improvement $\delta m$. 
In the course of the iterative full waveform inversion, it is naturally possible by \ASKI{} to gradually increase
the frequency content of the inverted data and to choose smaller inversion grid cells in each iteration, i.e.\ increasing
the spatial resolution dependent on frequency.


%% \vspace*{1cm}
%% {\bf {\Huge TODO} PUT A GENERAL OVERWIEW OF ASKI HERE, SO THAT THE SECTIONS IN ``BASIC STEPS'' DO NOT NEED TO 
%% EXPLAIN EVERY BIT OF BASIC STUFF: in the following there is a collection of unsorted bits of documentation
%% which could be involved in this section\\
%% also include some pictures: a set of points inside the volumes of the inversion grid, a sensitivity kernel
%% for one specific datum (real part of some frequency sample of a seismogram (src,rec,comp)) explaining that 
%% positive high values at one scatterer mean that this very datum increases (significantly) if the current 
%% model parameter is increased at this scatterer}

%% All operations \ASKI{} can conduct are based on Waveform Sensitivity Kernels. 

%% IN ESSENCE, PUT SOME SHORTENED, MORE INTUITIVELY UNDERSTANDABLE VERSION OF THE \ASKI{} PAPER (when existend) 
%% HERE, WHICH EXPLAINS THE TERMINI ``SENSITIVITY KERNELS'', ``INERSION GRID'', ``WAVEFIELD POINTS'', ``INTEGRATION WEIGHTS''
%% AND GIVES THE PRINCIPLE IDEAS OF COMPUTING THE KERNELS FROM WAVEFIELD GIVEN ON THE WAVEFIELD POINTS, INTEGRATING THEM ONTO THE
%% INVERSION GRID (hence, connecting forward simulation = data with the model) AND SOLVING LINEAR SYSTEMS 
%% CONTAINING THE SENSITIVITY VALUES. IT SHOULD ALSO BE EXPLAINED THAT ASKI USES DATA (hence, kernel sensitivity values) IN THE FREQUENCY DOMAIN AND WHY:


%
%-------------------------------
% CHAPTER How to get started
\chapter*{How to get started}
\phantomsection  % so hyperref creates bookmarks
\addcontentsline{toc}{chapter}{How to get started}
% -*-LaTex-*-

%-----------------------------------------------------------------------------
%   Copyright 2016 Florian Schumacher (Ruhr-Universitaet Bochum, Germany)
%
%   This file is part of the ASKI manual as a LaTeX document with main file
%   manual.tex
%
%   Permission is granted to copy, distribute and/or modify this document
%   under the terms of the GNU Free Documentation License, Version 1.3
%   or any later version published by the Free Software Foundation;
%   with no Invariant Sections, no Front-Cover Texts, and no Back-Cover Texts.
%   A copy of the license is included in the section entitled ``GNU
%   Free Documentation License''. 
%-----------------------------------------------------------------------------
%
%
%++++++++++++++++++++++++++++++++++++++++++++++++++++++++++
\section*{How to use this manual}
%++++++++++++++++++++++++++++++++++++++++++++++++++++++++++
%
Only chapter~\myref{guide} is intended to be read through, presenting recipes (todo lists/algorithms)
for different workflows that can be conducted by software package \ASKI{}.
For this reason, that chapter is held as compact 
as possible and may itself be regarded as ``the manual'', with the appending chapters only 
containing more specific detail on processes or objects which chapter~\ref{guide} refers to.
After all, the very modular character of \ASKI{} requires documentation which itself is modular.

In other words: just start reading the respective section of chapter~\ref{guide}, which you are interested in 
and whenever you feel the need for more detail follow the respective references. This way, we try to focus
the user on necessary information and successfully guide through the lot of details contained in this document. 

When you conduct a specific \ASKI{} operation for the first time, we recommend you to first fully read through the 
respective guiding list and the referred basic steps before you start running any programs. This way you will 
get an impression of the requirements for your operation.

All chapters appending chapter~\ref{guide} are not intended to be read through section by section but may well
serve the user as a reference. Especially section~\myref{programs_scripts,sec:bin_prog} can provide the 
experienced user with additional executables that conduct some more features which are not referenced somewhere
in the document.

\subsection*{Please preserve your experience!}
If you struggled with the existing \ASKI{} documentation (user manual, comments in code, doxygen, developers 
manual) because it was inconsistent, incomplete or simply wrong and you invested time to find out how it works, 
\emph{please let future generations of users and developers benefit from your gained knowledge!}
Everybody knows that documenting code and writing manuals consumes a lot of time, but correct documentation 
is essential for everyone using and developing software, and I'm sure you know that from your own experience.
So, please invest a bit more time in correcting/extending the \ASKI{} documentation
(where applicable: user manual, comments in code, doxygen, developers manual) and
at best modify the respective source files and issue a pull request on \lcode{gitHub}. In any case let us 
know about it (via \url{https://github.com/seismology-RUB} or \url{http://www.rub.de/aski})!
Otherwise your knowledge is lost forever (you might even lose your knowledge yourself after some while, 
so please write it down).

\emph{Thank you!} (on behalf of everybody)

%
%++++++++++++++++++++++++++++++++++++++++++++++++++++++++++
\section*{Installing \ASKI} %\label{basic_steps,sec:install_ASKI}
%++++++++++++++++++++++++++++++++++++++++++++++++++++++++++
%
\begin{itemize}
\item Clone the latest version of the master branch of the \ASKI{} \lcode{gitHub} repository to some directory 
  (exemplarily called \lcode{/your/programs/}) on your local computer by executing\\
  \lcode{git clone --depth 1 --branch master https://github.com/seismology-RUB/ASKI}\\
  (in one line, of course) from local path \lcode{/your/programs/}~. This will create subdirectory \lcode{/your/programs/ASKI} containing
  the code and documentation of the current release of the \ASKI{} main package.

  Alternatively, go to \url{https://github.com/seismology-RUB/ASKI} and download the content of the master branch
  as a \lcode{.zip} or try executing\\
  \lcode{wget https://github.com/seismology-RUB/ASKI/archive/master.zip}\\
  (in one line, of course) and extract it in such a way that the code files are contained in \lcode{/your/programs/ASKI/}~.

  Webpage \url{http://www.rub.de/aski} might provide additional information for you, just have a look.
\item Follow the directions in file \lcode{ASKI/README.md} for configuration and compilation of the \ASKI{} ececutables.
\end{itemize}
Throughout the \ASKI{} documentation, ``\ASKI{} installation directory'' refers to directory \lcode{ASKI/}, i.e.\
\lcode{/your/programs/ASKI}, where you have cloned/extracted the \lcode{git} repository to.
%
%++++++++++++++++++++++++++++++++++++++++++++++++++++++++++
\section*{Toy example: synthetic waveform inversion}
%++++++++++++++++++++++++++++++++++++++++++++++++++++++++++
%
In order to get familiar with applying \ASKI{} for full waveform inversion, you might consider to 
download and reproduce the synthetic inversion presented in Florian's dissertation 
\cite{_743d334d-dfa4-4a16-8cc5-91cdadc95271} (chapter 5.1) as well as in our GJI paper \cite{Schumacher16} 
(section 4.1), see section~\myref{guide,sec:example_C_borehole}.
%
%++++++++++++++++++++++++++++++++++++++++++++++++++++++++++
\section*{\ASKI{} versions}
%++++++++++++++++++++++++++++++++++++++++++++++++++++++++++
%
\ASKI{}'s release version numbering does not follow any standard of version numbering. 
Since releases were not very frequent so far and the source code was not publically available under version control, 
it was considered sufficient to have only a simple numbering for the purpose of distinguishing release versions.
In case that developments and releases become more frequent in the future, the \ASKI{} developers might consider
to follow some standard for future release version numbering.

\subsection*{\ASKI{} \lcodetitle{0.3}}
\ASKI{}'s first release version was numbered \lcode{0.3}, with \lcode{0} indicating a pre-release
that was not very well tested at that point and \lcode{3} being the third version of internal
development when porting from another internal versioning repository.

\subsection*{\ASKI{} \lcodetitle{1.0}}
After some while of intensive testing and application of \ASKI{} to synthetic and real-world cases, 
as well as development of a lot of \ASKI{} tools, version \lcode{1.0} was released as the first
ready-to-use version of \ASKI{}.

\subsection*{\ASKI{} \lcodetitle{1.1}}
In general, version \lcode{1.1} should be compatible with \lcode{1.0} in terms of file formats and general use
of the software. The main reason for this release was the fix of a pointer problem with \lcode{gfortran}.
Compared with the previous release, some more tools are available (\lcode{addSpikeCheckerToKim}, 
\lcode{createSpectralFilters}, \lcode{create_ASKI_evstat_filters.py}) and forward-code-specific 
definition of complex frequencies is available (e.g. for use with Gemini). Some bugs were fixed.

\subsection*{\ASKI{} \lcodetitle{1.2}}
Version \lcode{1.2} is just an intermediate version number and denotes the status of the code when moving the 
development repository permanently to git and providing the current working ready-to-use version of \ASKI{} by the 
master branch of repository \url{https://github.com/seismology-RUB/ASKI}.

Significant changes to version \lcode{1.1}:
\begin{itemize}
\item 
There is an additional subdirectory \lcode{devel/} containing developer tools and developer documentation.

\item
The source files of the \ASKI{} user manual as well as a compiled pdf of the manual are provided now in subdirectory
\lcode{doc/}

\item 
Inversion grids of type \lcode{chunksInversionGrid} now provide base cell refinement capabilities (at the moment
a random ``toy'' method is implmemented for illustration, but serious refinement method can now be implemented
easily into module \lcode{chunksInversionGrid}.

\item New/renamed/removed tools:

  \lcode{chunksInvgrid2vtk} for special vtk files related to chunks and refined cells 

  \lcode{createShoreLines} (Fortran executable) and \lcode{create_shore_lines.py} (Python program
  utilizing the \lcode{f2py} interface generator) for generating shore line vtk files from native binary GSHHS 
  shore line data files

  Executable \lcode{createMeasuredData} was renamed in \lcode{transformMeasuredData}

  Python script \lcode{create_ASKI_evstat_filters.py} is removed (functionality not required anymore)
\end{itemize}

\subsection*{The \lcodetitle{gitHub} master branch}
After porting \ASKI{} to \url{https://github.com/seismology-RUB/ASKI} in August 2016, the current version of its
master branch should serve as a stable version to use, along with the current versions of the forward code 
packages supported by \ASKI{}.

%
%-------------------------------
% CHAPTER Guide
\setcounter{chapter}{-1}
\chapter{\ASKI{} workflows} \label{guide}
%\phantomsection  % so hyperref creates bookmarks
%\addcontentsline{toc}{chapter}{Guide}
% -*-LaTex-*-

%-----------------------------------------------------------------------------
%   Copyright 2016 Florian Schumacher (Ruhr-Universitaet Bochum, Germany)
%
%   This file is part of the ASKI manual as a LaTeX document with main file
%   manual.tex
%
%   Permission is granted to copy, distribute and/or modify this document
%   under the terms of the GNU Free Documentation License, Version 1.3
%   or any later version published by the Free Software Foundation;
%   with no Invariant Sections, no Front-Cover Texts, and no Back-Cover Texts.
%   A copy of the license is included in the section entitled ``GNU
%   Free Documentation License''. 
%-----------------------------------------------------------------------------
%
This chapter is intended to guide you, dependent on what workflow you want to follow, through all necessary steps
to achieve your goals. Make sure that you have read section entitled ``How to get started''. 

If you don't know about \ASKI{} yet, we recommend you to quickly read through the previous section 
entitled ``What is \ASKI{}?'', which explains some basic terminology in \ASKI{} and the concepts it is based on.

The sections below address possible operations you can conduct with \ASKI{}. For every operation, we only refer 
to the necessary basic steps (by $\rightarrow$), which are described in chapter~\myref{basic_steps}.\\
Make sure to have a complete read-through before hastily doing anything!

Good Luck!
%
%
%++++++++++++++++++++++++++++++++++++++++++++++++++++++++++
\newpage
\section{Full Waveform Inversion using pre-integrated spectral Waveform Sensitivity Kernels} \label{guide,sec:classic_inversion}
%\phantomsection  % so hyperref creates bookmarks
%\addcontentsline{toc}{section}{Full Waveform Inversion - Classical Waveform Sensitivity Kernels}
%++++++++++++++++++++++++++++++++++++++++++++++++++++++++++
%
%
%here short introduction as to what ``Full Waveform Sensitivity Kernel Inversion'' actually is, followed by
%the steps to do:

Iterative inversion scheme which uses waveform sensitivity kernels to gain model updates form data residua.

You should have already installed \ASKI{}, see section entitled ``How to get started''. 
In addition to \ASKI{}, you will need to install a supported software which 
solves the forward problem \myaref{basic_steps,sec:forward_problem}.
%
%----------------------------------------------------------
\subsection*{Before The First Iteration Step}
%----------------------------------------------------------
%
\begin{itemize}
\item Create a main parameter file (e.g.\ in the parent directory of your specific inversion directory, 
or where you collect main parameter files for all your inversion projects)
\myaref{basic_steps,sec:main_parfile}. You will need this file as an input argument to almost all programs/scripts.\\
Set \lcode{MAIN_PATH_INVERSION} to a correct value. 
The directory does not need to exist yet, if not, then it will be created. \\
You should keep the default values of \lcode{ITERATION_STEP_PATH} and \lcode{PARFILE_ITERATION_STEP} unless
you know what you are doing. If you wish to use different values for the complete inversion process, 
it makes senese to adjust them now.\\
All other parameters can be adjusted later. 
%
\item Create a directory structure for the expected number of iteration steps of your inversion 
\myaref{basic_steps,sec:create_dir}
%
\item Make yourself familiar with the form of data used in \ASKI{} \myaref{basic_steps,sec:data_general}. \\
  Set \lcode{APPLY_EVENT_FILTER}, \lcode{APPLY_STATION_FILTER}, \lcode{PATH_MEASURED_DATA}, 
  \lcode{PATH_EVENT_FILTER}, \lcode{PATH_STATION_FILTER}, 
  \lcode{FILE_EVENT_LIST}, and \lcode{FILE_STATION_LIST}, as well as \lcode{MEASURED_DATA_FREQUENCY_STEP}, 
  \lcode{MEASURED_DATA_NUMBER_OF_FREQ} and \lcode{MEASURED_DATA_INDEX_OF_FREQ}  in your main 
  parameter file, before preparing your data in the required form \myaref{basic_steps,sec:measured_data}, 
  as well as any filters \myaref{basic_steps,sec:filters}.
%
\item Set \lcode{FORWARD_METHOD} in your main 
parameter file to the value of your choice. If you want to use different methods in the course of 
inverting one dataset (e.g.\ starting with a 1D method, continuing with a 3D method), then it may make 
sense to create a different directory structure for each method and using the final model of one method
as the starting model for the next method.
%
\item Choose a model parametrization by setting \lcode{MODEL_PARAMETRIZATION} in the main parameter file 
to a value of your choice (which is supported by your forward method).
%
\end{itemize}
%
%----------------------------------------------------------
\subsection*{Before Each Iteration Step (including the first one)}
%----------------------------------------------------------
%
\begin{itemize}
\item Set \lcode{CURRENT_ITERATION_STEP}
in your main parameter file to the correct value. When continuing your inversion with a different method, you
may also keep the current iteration step index (in order for you not to get confused) and leave 
subdirectories of your \lcode{MAIN_PATH_INVERSION} empty (or delete them after creation if they 
are not needed):\\
e.g.\ an inversion with one method could start with iteration step 4 (and respective subdirectory), 
if you have already conducted 3 iteration steps with other methods.
%
\item Define the inversion grid of the current iteration \myaref{basic_steps,sec:invgrid}.
%
\item Set all parameters in the specific iteration step parameter file to correct values 
\myaref{basic_steps,sec:iter_parfile}, including the correct reference to your inversion grid.
Refer to the documentation of your forward method on how to set filenames \lcode{FILE_WAVEFIELD_POINTS}
and \lcode{FILE_KERNEL_REFERENCE_MODEL} (as the handling of these file are method dependent).
%
\item Dependent on your method and model parametrization, take care about communicating the current
model (inverted in the previous iteration) to your forward method \myaref{basic_steps,sec:export_kim}. 
Before the first iteration, however, you need to define some starting model \myaref{basic_steps,sec:start_model}.
%
\end{itemize}
%
%----------------------------------------------------------
\subsection*{Conducting An Iteration Step}
%----------------------------------------------------------
%
\begin{itemize}
\item Compute forward wavefields and Green tensor components w.r.t.\ the current model 
by your method. Refer to the respective documentation of your method.\\
After that, you may prepare the synthetic data in the way \ASKI{} expects it (see sections~\myref{basic_steps,sec:data_general} 
and~\myref{files,sec:synth_data}). Refer to the documentation of your method on how to do it.
%
\item Set \lcode{TYPE_INTEGRATION_WEIGHTS} in the iteration specific parameter file 
\myaref{basic_steps,sec:intw}.
You should keep the default values of \lcode{FILE_INTEGRATION_WEIGHTS},
unless you know what you are doing (can be any name, will be created, but is referred to, afterwards. 
You could compute different types of weights and store them in different files, this way). 
%
\item Initiate basic requirements for all programs and scripts \myaref{basic_steps,sec:initBasics}.
%
\item Define data and model space, where the paths are mainly important for now \myaref{basic_steps,sec:dmspace}.
%
\item Compute sensitivity kernels for your specific set of paths and your set of model parameters 
  \myaref{basic_steps,sec:compute_kernels}. \\
  If desired, you may have a look at your kernels \myaref{basic_steps,sec:plot_kernels}.
%
\item Choose a specific data and model space. You may well play around with different subsets of data or
smoothing (next step) in the course of inverting for different models \myaref{basic_steps,sec:dmspace}.
%
\item Finally compute the inverted model by solving the kernel system \myaref{basic_steps,sec:solve_kernel_system},
  possibly varying the regularization constraints.
%
\item After deciding whether or not to do another iteration of full waveform inversion and choosing a model 
  for the next iteration \myaref{basic_steps,sec:investigate_convergence}, repeat all operations beginning 
  with ``Before Each Iteration Step (including the first one)''.
\end{itemize}
%
%
%++++++++++++++++++++++++++++++++++++++++++++++++++++++++++
\newpage
\section{Toy Example Using \lcodetitle{SPECFEM3D\_Cartesian} as a forward solver} \label{guide,sec:example_SPEC_C}
%\phantomsection  % so hyperref creates bookmarks
%\addcontentsline{toc}{section}{Toy Example Using \lcodetitle{SPECFEM3D\_Cartesian} as a forward solver}
%++++++++++++++++++++++++++++++++++++++++++++++++++++++++++
%
%
Please refer to section~\myref{guide,sec:example_C_borehole}.
%
%
%++++++++++++++++++++++++++++++++++++++++++++++++++++++++++
\newpage
\section{Toy Example Using \lcodetitle{SPECFEM3D\_GLOBE} as a forward solver} \label{guide,sec:example_SPEC_GLOB}
%\phantomsection  % so hyperref creates bookmarks
%\addcontentsline{toc}{section}{Toy Example Using \lcodetitle{SPECFEM3D\_GLOBE} as a forward solver}
%++++++++++++++++++++++++++++++++++++++++++++++++++++++++++
%
%
This section describes a very short example of computing a spherical waveform kernel for just one event-station 
pair using the setting of example "regional\_Greece\_small" from the \lcode{SPECFEM3D_GLOBE 7.0.0} release package 
(example uses 4 CPU cores only).

The kernel files of this example package were computed with \ASKI{} version 1.0, which however should
be reproducable also with higher versions of \ASKI{}
since the changes to version 1.2 do not effect the 
computations or compatibility of the output files. It may, however, 
occur that numbers might not be reproduced in a numerically exact way, which is also likely to happen
when a different compiler (version) is used. 

You should have already installed the \ASKI{} main package, see section entitled ``How to get started''. 

Then download the example
package \lcode{ASKI_example_spherical_small.tar.gz} which is attached to release \lcode{v1.0} 
of gitHub repository \url{https://github.com/seismology-RUB/ASKI} (198 MB, probably go to 
\url{https://github.com/seismology-RUB/ASKI/releases/tag/v1.0}).
The wavefield files in this example package were produced by version \lcode{1.0} of the extension package
\lcode{SPECFEM3D_GLOBE for ASKI} 
(\url{https://github.com/seismology-RUB/SPECFEM3D_GLOBE_for_ASKI/releases/tag/v1.0}), 
but you should be able to re-produce them as well with version \lcode{1.2} of the extension package
(\url{https://github.com/seismology-RUB/SPECFEM3D_GLOBE_for_ASKI/releases/tag/v1.2}).

\emph{Please note} that this example is not documented as detailed as the 
Cartesian cross-borehole example~\myaref{guide,sec:example_C_borehole}. Even if you do not intent to use 
the forward code \lcode{SPECFEM3D_Cartesian}, you are still advised to have a look at the Cartesian
cross-borehole example (or get to know \ASKI{} otherwise), in order to practice the operations of the main \ASKI{}
package and, thus, learn about how to conduct any steps \emph{after} the forward problem is solved, in particular
how to compute kernels. Otherwise, refer to the general FWI workflow until the point of kernel 
computation~\myaref{guide,sec:classic_inversion} in order to re-produce the kernels provided in this 
example package.

\emph{Most importantly}, you must change the value of \lcode{MAIN_PATH_INVERSION}
in file\\
\lcode{ASKI_example_spherical_small/main_parfile} to
\lcode{"your_base_path/"} in order for all the example directory to work. 
If you do not want to overwrite the existing output, alternatively you 
should create a separate inversion directory from scratch~\myaref{basic_steps,sec:create_dir}
and leave \lcode{ASKI_example_spherical_small/}
as a reference only. 
%
%
%++++++++++++++++++++++++++++++++++++++++++++++++++++++++++
\newpage
\section{Cross Borehole Example} \label{guide,sec:example_C_borehole}
%\phantomsection  % so hyperref creates bookmarks
%\addcontentsline{toc}{section}{Cross Borehole Example}
%++++++++++++++++++++++++++++++++++++++++++++++++++++++++++
%
%
This section describes the synthetic cross borehole example presented in Florian's dissertation 
\cite{_743d334d-dfa4-4a16-8cc5-91cdadc95271} (chapter 5.1) as well as in our GJI paper \cite{Schumacher16} 
(section 4.1). It is an example of lull waveform inversion using pre-integrated spectral waveform sensitivity 
kernels (as described in the first recipe~\myref{guide,sec:classic_inversion} of this chapter. 

The example files referred to in the following, were computed with \ASKI{} version 1.0, which however should
not pose any problem since the changes from version 1.0 to 1.2 do not effect the computations or compatibility
of any output files. It may, however, 
occur that numbers might not be reproduced in a numerically exact way, which is also likely to happen
when a different compiler (version) is used. 

You should have already installed \ASKI{}, see section entitled ``How to get started''. 

Then download the example
package \lcode{ASKI_inversion_cross_borehole.tar.gz} which is attached to release \lcode{v1.0} 
of gitHub repository \url{https://github.com/seismology-RUB/ASKI} (566 MB, probably go to 
\url{https://github.com/seismology-RUB/ASKI/releases/tag/v1.0}).
In order to get started with this example package, you may use the commands in script
\lcode{ASKI_inversion_cross_borehole_run.sh}, provided for download at the very same location
(\emph{please adjust variable} \lcode{ASKI} \emph{at the beginning of the script!}).

In this package, all files of the first 2 iterations (plus illustrating extras) are provided. 
In order to keep the download small, wavefield output is not provided  for
iterations 3 to 12. You should be able to reproduce the wavefields immendiately for any iteration step
from the provided inverted models without reproducing all preceeding iterations. 
The pre-integrated sensitivity kernel files are only provided for
iterations 1 to 3. Kernel files for the other iterations can be downloaded by
interested users seperately as package 
\lcode{ASKI_inversion_cross_borehole_sensitivity_kernels_iter04-iter12.tar.gz}
(907 MB, same location \url{https://github.com/seismology-RUB/ASKI/releases/tag/v1.0}).

\emph{Most importantly}, you must change the value of \lcode{MAIN_PATH_INVERSION}
in file\\
\lcode{ASKI_inversion_cross_borehole/main_parfile_cross_borehole} to
\lcode{"your_base_path/"} in order for all the example directory to work. 
If you do not want to overwrite the existing output, alternatively you 
should create a separate inversion directory from scratch~\myaref{basic_steps,sec:create_dir}
 and leave \lcode{ASKI_inversion_cross_borehole/}
as a reference only (this strategy is followed by script \lcode{ASKI_inversion_cross_borehole_run.sh}. 

This example was generated using \lcode{SPECFEM3D_Cartesian 3.0} for \ASKI{} version 1.0.
(\url{https://github.com/seismology-RUB/SPECFEM3D_Cartesian_for_ASKI/releases/tag/v1.0}).
Even if you intend
to use a different forward method, you can still practice the operations of the main \ASKI{}
package and, thus, learn about how to conduct any steps in
the full waveform inversion \emph{after} the forward problem is solved. Just read the description
of the contained files in \\
\lcode{ASKI_inversion_cross_borehole/README}.

If you want to reproduce also the forward wavefields from \lcode{SPECFEM3D_Cartesian 3.0}, 
you need to install it, of course~\myaref{basic_steps,sec:forward_problem}.
%
%
%++++++++++++++++++++++++++++++++++++++++++++++++++++++++++
\newpage
\section{Time-Domain Sensitivity Kernels} \label{guide,sec:time_kernels}
%\phantomsection  % so hyperref creates bookmarks
%\addcontentsline{toc}{section}{Time-Domain Sensitivity Kernels}
%++++++++++++++++++++++++++++++++++++++++++++++++++++++++++
%
%
This section describes, how to compute time-domain waveform sensitivity kernels for a specific set 
of source-receiver paths with respect to a certain background earth model as an operation seperate of any
other \ASKI{} operations. The kernels in time-domain are much more intuitive to look at for human beings, 
than the standard frequency-domain sensitivity kernels. You may, as well, compute time-domain sensitivity 
kernels from the kernels produced in any iteration step of a full waveform inversion 
(page \pageref{guide,sec:classic_inversion}). For this purpose, apply the steps 
``Transforming to Time-Domain Sensitivity Kernels'' (below) after you computed the spectral kernels in 
your iteration step, as the time-domain waveform kernels are produced by an inverse Fourier transform 
from the standard frequency-domain waveform sensitivity kernels on which \ASKI{} is based.

Please do not get confused by the general terminology of \emph{inversion} and \emph{iteration}, etc. 
Technically you will be conducting an incomplete first iteration step of a full waveform inversion, 
using all the program infrastructure which is also used for a full waveform inversion. 

You should have already installed \ASKI{}, see section entitled ``How to get started''. 
In addition to \ASKI{}, you will need to install a supported software which 
solves the forward problem \myaref{basic_steps,sec:forward_problem}.
%
%----------------------------------------------------------
\subsection*{Preceding Considerations}
%----------------------------------------------------------
%
\begin{itemize}
\item Create a main parameter file (e.g.\ in the parent directory of your specific inversion directory, 
or where you collect main parameter files for all your inversion projects or analyses)
\myaref{basic_steps,sec:main_parfile}. You will need this file as an input argument to almost all programs/scripts.\\
Set \lcode{MAIN_PATH_INVERSION} to a correct value.
The directory does not need to exist yet, if not, then it will be created. \\
Set \lcode{ITERATION_STEP_PATH} and 
\lcode{PARFILE_ITERATION_STEP} to desired values, 
or leave the default values, if present.\\
All other parameters can be adjusted later. 
%
\item Create a directory structure for only one iteration step \myaref{basic_steps,sec:create_dir}
%
\item Even if you do not have any measured data, it might still be beneficial for you to make yourself 
(roughly) familiar with the form of data used in \ASKI{} \myaref{basic_steps,sec:data_general}. \\
In your main parameter file, set the following values: 
  \begin{itemize}
  \item Set \lcode{FILE_EVENT_LIST} and \lcode{FILE_STATION_LIST} to define the sources and receivers which are
    involved in the paths that you would like to compute the kernels for. 
  \item Dependent on the (length of the) time series you want to deal with, define the frequency discretization of the
    spectral kernels that will be produced first. This must be done by \lcode{MEASURED_DATA_FREQUENCY_STEP}, 
    \lcode{MEASURED_DATA_NUMBER_OF_FREQ} and \lcode{MEASURED_DATA_INDEX_OF_FREQ}.
  \end{itemize}
In general, for the pure kernel computation you do not need any measured data. So, here you do not need to prepare
data in the \ASKI{} required form.
%
\item Set \lcode{APPLY_EVENT_FILTER}, \lcode{APPLY_STATION_FILTER} (and if required \lcode{PATH_EVENT_FILTER} and 
  \lcode{PATH_STATION_FILTER}) in your main parameter file. In case the forward wavefields were calculated w.r.t.\ an
  impulse source time function (or the source-time function was deconvolved), you should apply filters 
  \myaref{basic_steps,sec:filters} that taper down the amplitude spectrum before the maximum frequency used. 
  Otherwise, the inverse Fourier transform below can create artefacts.
%
\item Set \lcode{FORWARD_METHOD} in your main parameter file to the value of your choice. 
%
\item Choose a model parametrization by setting \lcode{MODEL_PARAMETRIZATION} in the main parameter file 
to a value of your choice (which is supported by your forward method).
%
\item Set \lcode{CURRENT_ITERATION_STEP} in your main parameter file to value \lcode{1}, as you are technically
starting to conduct the first (and only) iteration step of a full waveform inversion.
%
\item It is highly recommended to set \lcode{DEFAULT_VTK_FILE_FORMAT} to \lcode{BINARY} in the main parameter file, 
  since a lot of \lcode{vtk} files might be generated (one for every time step). Using the binary format significantly
  reduces storage.

\item Define the inversion grid \myaref{basic_steps,sec:invgrid}, which controls the spacial volumetric discretization 
(resolution) of the computed sensitivity kernels. In case of just computing (time) kernels to look at, it is not
crucial to regard this resolution as the resolution of some inverted model, as no inversion will be conducted on the 
inversion grid. Since there might be a lot of \lcode{vtk} files generated (one for every time step), it is highly recommended to set\\
\lcode{VTK_GEOMETRY_TYPE = CELL_CENTERS} in the inversion grid parameter file (if your type of inversion grid
supports that functionality). This significantly reduces the amount of geometry information written to the 
\lcode{vtk} files, hence their size.
%
\item Set all parameters in the specific iteration step parameter file to correct values 
\myaref{basic_steps,sec:iter_parfile}, including the correct reference to the inversion grid. Set 
\lcode{ITERATION_STEP_NUMBER_OF_FREQ} and \lcode{ITERATION_STEP_INDEX_OF_FREQ} to the \emph{same} values as the
\lcode{MEASURED_DATA_NUMBER_OF_FREQ} and \lcode{MEASURED_DATA_INDEX_OF_FREQ} in the main parameter file!
Refer to the documentation of your forward method on how to set filenames \lcode{FILE_WAVEFIELD_POINTS}
and \lcode{FILE_KERNEL_REFERENCE_MODEL} (as the handling of these file are method dependent).
%
\item Dependent on your method and model parametrization, define your background model with respect to which the
kernels will be computed. If you have some inverted model file, use \myaref{basic_steps,sec:export_kim}. For defining
a starting model, see \myaref{basic_steps,sec:start_model}.
%
\end{itemize}
%
%----------------------------------------------------------
\subsection*{Computing Spectral Waveform Sensitivity Kernels}
%----------------------------------------------------------
%
\begin{itemize}
\item Compute forward wavefields and Green tensor components w.r.t.\ the current model 
by your method. Refer to the respective documentation of your method.\\
After that, you may prepare the synthetic data in the way \ASKI{} expects it (see sections~\myref{basic_steps,sec:data_general} 
and~\myref{files,sec:synth_data}). Refer to the documentation of your method on how to do it.
%
\item Set \lcode{TYPE_INTEGRATION_WEIGHTS} in the iteration specific parameter file 
\myaref{basic_steps,sec:intw}. If you intend to compute the time kernels on wavefield points only, you will technically
never use the integration weights, but nevertheless they need to be correctly created (in that case use a simple
type of weights like 5 or 0).
You should keep the default values of \lcode{FILE_INTEGRATION_WEIGHTS},
unless you know what you are doing (can be any name, will be created, but is referred to, afterwards. 
You could compute different types of weights and store them in different files, this way). 
%
\item Initiate basic requirements for all programs and scripts \myaref{basic_steps,sec:initBasics}.
%
\item If you have many paths, you may define a data and model space concentrating on defining paths 
  \myaref{basic_steps,sec:dmspace}.
  If you have only one path or just a few, it is possible (and propably also convenient) to just continue to the
  computation of the kernels.
%
\item Compute the standard frequency-domain sensitivity kernels for your specific set of paths (or the one or 
  few paths, one after another). In case you want to transform to time-domain kernels \emph{on wavefield points} 
  (and not pre-integrated kernels on inversion grid),
  use option \lcode{-wp} \myaref{basic_steps,sec:compute_kernels} \\
  If desired, you may have a look at the standard frequency-domain kernels \myaref{basic_steps,sec:plot_kernels}.
\end{itemize}
%
%----------------------------------------------------------
\subsection*{Transforming to Time-Domain Sensitivity Kernels}
%----------------------------------------------------------
%
\begin{itemize}
\item Transform the standard frequency-domain waveform kernels to time domain. In case you want to get  
  time-domain kernels \emph{on wavefield points}, (and not pre-integrated kernels on inversion grid),
  you must use option \lcode{-wp} (in this case, you already should have used \lcode{-wp} to compute 
  spectral kernels) \myaref{basic_steps,sec:compute_time_kernels}.\\
  Make sure that the wavefield spectra have a sufficiently small amplitude spectrum at high frequencies, in order for the
  inverse Fourier Transform to work satisfactorily. Otherwise use filters.
%
\item Plot the time kernels \myaref{basic_steps,sec:plot_time_kernels}.
\end{itemize}
%
%++++++++++++++++++++++++++++++++++++++++++++++++++++++++++
\newpage
\section{Analysis of Acquisition Geometry using Kernel Focussing} \label{guide,sec:acq_ana_focus}
%\phantomsection
%\addcontentsline{toc}{section}{Analysis of Acquisition Geometry using Kernel Focussing}
%++++++++++++++++++++++++++++++++++++++++++++++++++++++++++
%
{\bf THIS SECTION IS WORK IN PROGRESS!}

%Short introduction as to what ``Analysis of Acquisition Geometry using Kernel Focussing'' actually is:
The general concept of Backus-Gilbert-based focussing of sensitivity kernels on a defined focussing region 
in the model space is published in \cite{_743d334d-dfa4-4a16-8cc5-91cdadc95271}, Chapter 7.1. On the basis
of this way of finding linear combinations of data which are sensitive only to changes in a model subspace, 
the dataset can be analysed w.r.t.\ finding event-station paths (and components / frequencies) which optimally
illuminate the model subspace. 

Executable \lcode{focusSpectralKernels} \myaref{programs_scripts,sec:bin_prog,sec:focus_spec_kernel} produces the coefficients of the linear combination of data required
for the analysis. However, this section does not yet give a detailed list of steps how to conduct such 
an analysis. 

\begin{itemize}
\item step 1
\item step 2
\item ...
\end{itemize}
%
%++++++++++++++++++++++++++++++++++++++++++++++++++++++++++
\newpage
\section{Full Waveform Inversion - Focused Waveform Sensitivity Kernels} \label{guide,sec:focused_inversion}
%\phantomsection
%\addcontentsline{toc}{section}{Full Waveform Inversion - Focused Waveform Sensitivity Kernels}
%++++++++++++++++++++++++++++++++++++++++++++++++++++++++++
%
{\bf THIS SECTION IS WORK IN PROGRESS!}

%Short introduction as to what ``Full Waveform Inversion using Focused Data for Maximum Resolution'' actually is:
The general concept of Backus-Gilbert-based focussing of sensitivity kernels on a defined focussing region 
in the model space is published in \cite{_743d334d-dfa4-4a16-8cc5-91cdadc95271}, Chapter 7.1. On the basis
of this way of finding linear combinations of data which are sensitive only to changes in a model subspace, 
different disjoint subdivisions of the model space can be chosen, finding the finest subdivision that
is resolved by the data (i.e.\ where focussing of sensitivity on each subdivision cell is still successfull).
This gives the finest model space that the data can resolve. Also this gives a set of ``new'' data consisting of
linear combinations of the original data, for which each of this new data is sensitive only to a certain
spot in the model space. Doing inversion steps of full waveform inversion with the focussed kernels and the 
focussed data, the hope is to overall conduct a waveform inversion at the maximum resolving power of the data. 
This can also be seen as a sophisticated form of regularizing the overall inverse problem. 

Such a ``focussed waveform inversion'' still remains an idea and was not yet approached to be implemented
in \ASKI{}. Please don't hesitate to implement and test this concept.

todo list:
\begin{itemize}
\item step 1
\item step 2
\item ...
\end{itemize}


%
%-------------------------------
% CHAPTER Basic Steps
%\setcounter{chapter}{-1}
\chapter{Basic Steps} \label{basic_steps}
% -*-LaTex-*-

%-----------------------------------------------------------------------------
%   Copyright 2016 Florian Schumacher (Ruhr-Universitaet Bochum, Germany)
%
%   This file is part of the ASKI manual as a LaTeX document with main file
%   manual.tex
%
%   Permission is granted to copy, distribute and/or modify this document
%   under the terms of the GNU Free Documentation License, Version 1.3
%   or any later version published by the Free Software Foundation;
%   with no Invariant Sections, no Front-Cover Texts, and no Back-Cover Texts.
%   A copy of the license is included in the section entitled ``GNU
%   Free Documentation License''. 
%-----------------------------------------------------------------------------
%
%% ATTENTION NOTE FOR USERS ACCIDENTLY GETTING STARTED WITH ASKI (in a hurry) WITHOUT
%% FOLLOWING THE INTENDED PROCEDURE HOW TO READ THIS MANUAL
\inotice{This chapter is \emph{not} intended to be read through item by item (as a manual)! If you are new to \ASKI{} and
  accidently looked up this chapter in the hope to find what you're looking for, you are strongly advised to
  quickly read section ``How to get started'' at the beginning of this document, explaining how to use this manual.}

In general, in this chapter we provide only basic information. For more detail on 
specific steps or objects, we always refer to the respective sections below in this document.
%
%++++++++++++++++++++++++++++++++++++++++++++++++++++++++++
\section{Create Main Parameter File} \label{basic_steps,sec:main_parfile}
%++++++++++++++++++++++++++++++++++++++++++++++++++++++++++
%
The simplest way to create a specific main parameter file for your operation is to modify / adjust 
a copy of the template file \lcode{template/main_parfile_template}. 

Refer to the commented documentation in \lcode{main_parfile_template} or 
to sections~\myref{files,sec:parfiles} and~\myref{files,sec:main_parfile}.
%
%++++++++++++++++++++++++++++++++++++++++++++++++++++++++++
\section{Iteration Step Parameter Files} \label{basic_steps,sec:iter_parfile}
%++++++++++++++++++++++++++++++++++++++++++++++++++++++++++
%
Having created a directory environment for your operation, as described in section~\myref{basic_steps,sec:create_dir},
there should automatically have been created template
parameter files in each directory of an iteration step, having filenames as defined by
parameter \lcode{PARFILE_ITERATION_STEP} in the main parameter file.

Refer to the commented documentation in those template files or 
to sections~\myref{files,sec:parfiles} and~\myref{files,sec:iter_parfile}.
%
%++++++++++++++++++++++++++++++++++++++++++++++++++++++++++
\section{Create Directory Environment} \label{basic_steps,sec:create_dir}
%++++++++++++++++++++++++++++++++++++++++++++++++++++++++++
%
Call python script \lcode{create_ASKI_dir.py}
\begin{lstlisting}
USAGE: please give 2 arguments:
[1] main parmeter file of inversion
[2] number of iteration steps

EXAMPLE:
create_ASKI_dir.py ./main_parfile_Aegean1 10
\end{lstlisting}
Put your main parameter file (see~\myref{basic_steps,sec:main_parfile}) as the first, and the expected 
number of iteration steps as the second argument. \\
You can always \emph{recall this script at any later time} with a larger number of iteration steps. 
All existing directories \emph{will not be affected}, only additional non-existing objects will be 
created. Recalling this script with a smaller number of steps will not delete anything.
%
%++++++++++++++++++++++++++++++++++++++++++++++++++++++++++
\section{Data in \ASKI} \label{basic_steps,sec:data_general}
%++++++++++++++++++++++++++++++++++++++++++++++++++++++++++
%
One certain data sample in \ASKI{} is characterized by a seismic \emph{source}, a \emph{component} 
of a seismic \emph{station} (receiver), and a \emph{frequency}, as well as if it is \emph{real} or \emph{imaginary} part 
of the complex spectral values. It has the value of displacement of the ground in the unit of meters.
%
%----------------------------------------------------------
\subsection*{Events and Stations}
%----------------------------------------------------------
%
The events file (\myref{files,sec:event_list}) and stations file (\myref{files,sec:station_list}) constitute a
collection of \emph{all} events (stations) which will be involved \emph{in any way} in your \ASKI{} operation.\\
All programs/scripts will refer to a specific event (station) by its event-ID (station-name).
%
%----------------------------------------------------------
\subsection*{Station Components}
%----------------------------------------------------------
%
All programs/scripts will refer to a specific station component by the following abbreviatory names.\\
Dependent on the coordinate system in which the stations are defined (Cartesian or spherical, which is 
defined by the first line of the station list file), the supported names of station components may have a 
different meaning:
\subsubsection{Cartesian stations}
\begin{itemize}
\item[] \lcode{CX}: Cartesian \lcode{X}-coordinate (first Cartesian coordinate)
\item[] \lcode{CY}: Cartesian \lcode{Y}-coordinate (second Cartesian coordinate)
\item[] \lcode{CZ}: Cartesian \lcode{Z}-coordinate (third Cartesian coordinate)
\item[] \lcode{N}: same as \lcode{-CX}
\item[] \lcode{S}: same as \lcode{CX}
\item[] \lcode{E}: same as \lcode{CY}
\item[] \lcode{W}: same as \lcode{-CY}
\item[] \lcode{UP}: same as \lcode{CZ}
\item[] \lcode{DOWN}: same as \lcode{-CZ}
\end{itemize}
\subsubsection{Spherical stations}
\begin{itemize}
\item[] \lcode{CX}: Cartesian \lcode{X}-coordinate with \lcode{X}-axis through equator and $0^\circ$-meridian
\item[] \lcode{CY}: Cartesian \lcode{Y}-coordinate with \lcode{Y}-axis through equator and $90^\circ$E-meridian
\item[] \lcode{CZ}: Cartesian \lcode{Z}-coordinate with \lcode{Z}-axis through north pole
\item[] \lcode{N}: local north
\item[] \lcode{S}: local south
\item[] \lcode{E}: local east
\item[] \lcode{W}: local west
\item[] \lcode{UP}: local up
\item[] \lcode{DOWN}: local down
\end{itemize}
%
%----------------------------------------------------------
\subsection*{Frequency Discretization}
%----------------------------------------------------------
%
In \ASKI{}, frequencies are given by a frequency step $\Delta f$[Hz] and by a set of integer valued 
frequency indices.

For specific frequency index $k\in\mathbb{N}$, the corresponding real-valued frequency $f_k$ [Hz] computes as
$f_k \,=\, k \,\cdot\, \Delta f$.
E.g.\ $\Delta f \,=\, 10$ Hz and frequency indices $k \,=\, 2,\; 3,\; 5,\; 7,\; 10$ define the set of 
discrete frequencies $f_k \,=\, 20.0,\; 30.0,\; 50.0,\; 70.0,\; 100.0$ [Hz].

Some forward methods (at the moment only Gemini) implicitely use \emph{complex} frequencies, which are related
to the above definition of real-valued frequencies $f_k \,=\, k \,\cdot\, \Delta f$.
Gemini, e.g., adds to these real-valued frequencies a constant imaginary part $\sigma = -5\Delta f/2\pi$,
thus implicitely using the complex frequencies $f_k \,=\, k \cdot \Delta f \,+\,i\cdot \sigma$,
where $i$ is the imaginary unit.
%
%++++++++++++++++++++++++++++++++++++++++++++++++++++++++++
\section{Prepare Measured Data} \label{basic_steps,sec:measured_data}
%++++++++++++++++++++++++++++++++++++++++++++++++++++++++++
%
For measured data given in some basic data formats like Seismic Unix and time-series given as textfiles per trace,
the executable \lcode{transformMeasuredData} \myaref{programs_scripts,sec:bin_prog,sec:transform_measured_data} 
converts the time-domain data to the special frequency-domain
form required by \ASKI{}. Executing \lcode{transformMeasuredData} (without arguments), will print a short help message how to use it.

Otherwise you might well prepare measured data files on your own as required by \ASKI{}, see sections~\myref{basic_steps,sec:data_general} 
and~\myref{files,sec:measured_data}. 
%
%++++++++++++++++++++++++++++++++++++++++++++++++++++++++++
\section{Prepare Frequency-Domain Filters} \label{basic_steps,sec:filters}
%++++++++++++++++++++++++++++++++++++++++++++++++++++++++++
%
In \ASKI{}, synthetic wavefields (for synthetic data and sensitivity kernels) are preferred to 
be modelled as impulse responses, i.e.\ w.r.t.\ an impulsive Dirac source-time-function, and filtered afterwards
in the frequency domain (by multiplication of wavefield spectra and spectral filter) before being compared to 
measured data. Therefore, \ASKI{} uses spectral filters,
independently associated with each seismic source and each station component. This way, independent 
source-time-functions for each event, as well as independent instrument responses for each station component
can be modelled. Whether or not to use event-associated filters or station-associated filters at all, is 
controlled in the main parameter file via flags \lcode{APPLY_EVENT_FILTER, APPLY_STATION_FILTER} 
\myaref{files,sec:main_parfile,itm:path_mdata_filters}.

Executable \lcode{createSpectralFilters} \myaref{programs_scripts,sec:bin_prog,sec:create_spec_filters} 
provides means to generate event filters from time-domain wavelets
(i.e.\ specific source-time-functions) or as Butterworth filters. Executing \lcode{createSpectralFilters}
(without arguments) will print a short help message how to use it.
For now, there is no possibility to 
generate station filters by this executable, since it is assumed that instrument responses were 
already deconvolved from the data. 

If you need station filters, or require filters that cannot be generated by \lcode{createSpectralFilters}, 
you would have to generate filter files by yourself in the required format, see section~\myref{files,sec:filters}.
Even if you want to use the same filter for each event (station component), you would need to create a filter file
for each event (station component), e.g.\ by copy-paste-rename operations.

At the moment, \ASKI{} does \emph{not} support to have an independent filter for each event-station path (i.e.\ for
each seismogram seperately). 
%
%++++++++++++++++++++++++++++++++++++++++++++++++++++++++++
\section{Define an Inversion Grid} \label{basic_steps,sec:invgrid}
%++++++++++++++++++++++++++++++++++++++++++++++++++++++++++
%
There are different types of \ASKI{} inversion grids suitable for different geometries, 
forward methods, hence, applications.

All inversion grids are defined by setting parameters \lcode{TYPE_INVERSION_GRID} and 
\lcode{PARFILE_INVERSION_GRID} in the parameter file of the current iteration step.

In the following, we present the supported inversion grid types and explain the particular
parameters in the respective inversion grid parameter file.
%
%----------------------------------------------------------
\subsection{\lcodetitle{chunksInversionGrid}} \label{basic_steps,sec:invgrid,sub:chunks}
%----------------------------------------------------------
%
The chunks inversion grid consists of 1, 2, 3 or 6 chunks of a cubed sphere (for the order and positioning
of the chunks, see fig.~\ref{basic_steps,sec:invgrid,sub:chunks,fig:grid-ichunk}). 
The chunk width is always 90 degrees, except for 1-chunk grids
where smaller chunks are allowed. The inversion grid base cells are equi-angularly distributed,
resulting in cells which are more or less evenly distributed in size (by contrast to the 
schunk inversion grid, where the lateral projections of the cells onto the tangential plane
have an equi-distant distribution on the plane).\\ 
2-chunk grids consist of two neighbouring 90-degree chunks, 3-chunk grids of three chunks 
which are ALL neighbours of each other (i.e. essentially half of the Earth).\\
Inside the chunks, the inversion grid cells are constructed on the basis of regularly distributed
base cells in the fashion of the schunk inversion grid (or scart inversion grid), having a certain number of refinement
blocks in depth inside which a fixed regular (equi-angular) lateral resolution and fixed depth resolution can be chosen.
Thereafter, by certain mechanisms the base cells
can be locally refined, i.e.\ subdivided into a specific distribution of subcells.
\emph{At the moment, only an experimental random refinement
is implemented for illustration} (see fig.~\ref{basic_steps,sec:invgrid,sub:chunks,fig:grid-ref}{}),
in the future, cell refinements based e.g.\ on the ray coverage of the data could be implemented.

\begin{figure}[ht]
  \centering
  \includegraphics[width=0.65\textwidth]{images/chunksInversionGrid_manual.png}
  \caption{Example of a chunks inversion grid (with 3 chunks in global projection)}
  \label{basic_steps,sec:invgrid,sub:chunks,fig:grid}
\end{figure}

\begin{figure}[ht]
  \centering
  \includegraphics[width=0.65\textwidth]{images/chunksInversionGrid_refined_manual.png}
  \caption[]{Example of a chunks inversion grid (with 3 chunks in global projection) with
  randomly refined base cells:\\\hspace{\textwidth}
  %% \verb+CHUNKS_INVGRID_CREF_METHOD = EXPERIMENTAL_RANDOM_SUBDIVISION+\\\hspace{\textwidth}
  %% \verb+CHUNKS_INVGRID_CREF_PARAMETERS = 0 10+\\\hspace{\textwidth}
  %% \verb+CHUNKS_INVGRID_BASE_NREF_BLOCKS = 2+\\\hspace{\textwidth}
  %% \verb+CHUNKS_INVGRID_BASE_NLAY = 4 3+\\\hspace{\textwidth}
  %% \verb+CHUNKS_INVGRID_BASE_THICKNESS = 180. 250.+\\\hspace{\textwidth}
  %% \verb+CHUNKS_INVGRID_BASE_NLAT = 18 12+\\\hspace{\textwidth}
  %% \verb+CHUNKS_INVGRID_BASE_NLON = 18 12+}
  \protect\lcode{CHUNKS_INVGRID_CREF_METHOD = EXPERIMENTAL_RANDOM_SUBDIVISION}\\\hspace{\textwidth}
  \protect\lcode{CHUNKS_INVGRID_CREF_PARAMETERS = 0 10}\\\hspace{\textwidth}
  \protect\lcode{CHUNKS_INVGRID_BASE_NREF_BLOCKS = 2}\\\hspace{\textwidth}
  \protect\lcode{CHUNKS_INVGRID_BASE_NLAY = 4 3}\\\hspace{\textwidth}
  \protect\lcode{CHUNKS_INVGRID_BASE_THICKNESS = 180. 250.}\\\hspace{\textwidth}
  \protect\lcode{CHUNKS_INVGRID_BASE_NLAT = 18 12}\\\hspace{\textwidth}
  \protect\lcode{CHUNKS_INVGRID_BASE_NLON = 18 12}}
  \label{basic_steps,sec:invgrid,sub:chunks,fig:grid-ref}
\end{figure}

\begin{figure}[ht]
  \centering
  \includegraphics[width=0.65\textwidth]{images/chunksInversionGrid_6-chunks_ichunk_LOCAL_FLAT_numbers.png}
  \caption{distribution of chunks of the chunksInversionGrid in \lcode{LOCAL_FLAT} projection}
  \label{basic_steps,sec:invgrid,sub:chunks,fig:grid-ichunk}
\end{figure}

The inversion grid is defined via a paramter file, a template of which is file 
\lcode{template/chunksInversionGrid_parfile_template} (the template file defines the inversion grid
shown in fig.~\ref{basic_steps,sec:invgrid,sub:chunks,fig:grid}{}).
The meaning of the keywords in the parameter file defining an inversion grid of type \lcode{chunksInversionGrid}
are as follows:

\subsubsection{\lcode{CHUNKS_INVGRID_GEOM_NCHUNK}} %\label{files,sec:iter_parfile,itm:invgrid_rad}
Number of chunks (1, 2, 3, or 6).
%- - - - - - - - - - - - - - - - - - - - - - - - - - - - -
\subsubsection{\lcode{CHUNKS_INVGRID_GEOM_RMAX}} 
Maximum radius (upper boundary of the chunk(s)) [must be the same unit in which wavefield points are given, usually be in km]
%- - - - - - - - - - - - - - - - - - - - - - - - - - - - -
\subsubsection{\lcode{CHUNKS_INVGRID_GEOM_CLAT}, \lcode{CHUNKS_INVGRID_GEOM_CLON}} 
Center of cubed-sphere chunk in degrees.
%- - - - - - - - - - - - - - - - - - - - - - - - - - - - -
\subsubsection{\lcode{CHUNKS_INVGRID_GEOM_WLAT}, \lcode{CHUNKS_INVGRID_GEOM_WLON}} 
Assuming \lcode{ROT = 0}.\\
Width of cubed-sphere chunk in degrees (must be 90.0 for NCHUNK $> 1$).
%- - - - - - - - - - - - - - - - - - - - - - - - - - - - -
\subsubsection{\lcode{CHUNKS_INVGRID_GEOM_ROT}} 
Rotation of chunk around local vertical counterclockwise in degrees.
%- - - - - - - - - - - - - - - - - - - - - - - - - - - - -
\subsubsection{\lcode{CHUNKS_INVGRID_BASE_NREF_BLOCKS}}
Number of blocks of layers within which lateral cell width remains comstant.\\
(compare \lcode{scartInversionGrid}~\myref{basic_steps,sec:invgrid,sub:scart}: definition and usage of \lcode{NREF_BLOCKS}, 
\lcode{NLAY}, \lcode{THICKNESS}, \lcode{NX}, \lcode{NY})
%- - - - - - - - - - - - - - - - - - - - - - - - - - - - -
\subsubsection{\lcode{CHUNKS_INVGRID_BASE_NLAY}}
Number of layers within each block (integer array).\\
(compare \lcode{scartInversionGrid}~\myref{basic_steps,sec:invgrid,sub:scart}: definition and usage of \lcode{NREF_BLOCKS}, 
\lcode{NLAY}, \lcode{THICKNESS}, \lcode{NX}, \lcode{NY})
%- - - - - - - - - - - - - - - - - - - - - - - - - - - - -
\subsubsection{\lcode{CHUNKS_INVGRID_BASE_THICKNESS}}
Thickness of layers in each block in km (real array).\\
(compare \lcode{scartInversionGrid}~\myref{basic_steps,sec:invgrid,sub:scart}: definition and usage of \lcode{NREF_BLOCKS}, 
\lcode{NLAY}, \lcode{THICKNESS}, \lcode{NX}, \lcode{NY})
%- - - - - - - - - - - - - - - - - - - - - - - - - - - - -
\subsubsection{\lcode{CHUNKS_INVGRID_BASE_NLAT}, \lcode{CHUNKS_INVGRID_BASE_NLON}}
Number of cells in latitude/longitude direction for each layer block.\\
(compare \lcode{scartInversionGrid}~\myref{basic_steps,sec:invgrid,sub:scart}: definition and usage of \lcode{NREF_BLOCKS}, 
\lcode{NLAY}, \lcode{THICKNESS}, \lcode{NX}, \lcode{NY})
%- - - - - - - - - - - - - - - - - - - - - - - - - - - - -
\subsubsection{\lcode{CHUNKS_INVGRID_CREF_METHOD}, \lcode{CHUNKS_INVGRID_CREF_PARAMETERS}}
Definition of individual base cell refinement. If only base cells should be used, and no base cell refinement
should be done at all, set \lcode{CHUNKS_INVGRID_CREF_METHOD = NONE} (in this case, \lcode{CHUNKS_INVGRID_CREF_PARAMETERS} has
no meaning).

At the moment, only the following experimental method is supported (for illustration and testing):\\
\lcode{CHUNKS_INVGRID_CREF_METHOD = EXPERIMENTAL_RANDOM_SUBDIVISION} , requiring 
two integer values given by \lcode{CHUNKS_INVGRID_CREF_PARAMETERS}  (if real values are given, 
they are rounded down by \lcode{int()}):\\
\emph{first value}: 
input value seed for function \lcode{srand(seed)} (you can reproduce random results, using the same seed value)\\
\emph{second value}: upper bound of subcells by which any base cell can be subdivided (number of subcells cannot exceed
this number). \emph{Note}: if subcells are subdivided themselves, these sub-subcells add to this number,
but their parent cell (original subcell) will \emph{not} be seen as an external inversion grid cell. This
means, that the actual number of subcells of a base cell that are used as cells of the inversion grid can be 
smaller than \emph{second value}.\\
By the simple (and old-school) pseudo-random generator \lcode{rand}, a cell will be divided by 2, 3 or 4 in 
every dimension (also subcells will themselves be subdivided until upper limit is reached).

\emph{Ideas for future development}:\\ 
Give here the absolute paths of the station and event list files, as well 
as a data model space info file containing the paths. According to the 
thereby defined ray coverage, the base cells could be locally refined in 
a recursive manner (each cell could e.g.\ be halfened in each dimension of 
space recursively if a certain criterion is met). 
The refinement criteria could be dependent on a maximum global depth and
a threashold value could handle the number of rays which a cell
must be intersected with before it is refined, etc.

%- - - - - - - - - - - - - - - - - - - - - - - - - - - - -
\subsubsection{\lcode{CHUNKS_INVGRID_FILE}}
Filename of binary inversion grid file (relative to \lcode{MAIN_PATH_INVERSION}/\lcode{ITERATION_STEP_PATH/})
to which a newly created inversion grid will be written and from which the inversion grid
will be read if it exists and if not recreating the inversion grid.
%- - - - - - - - - - - - - - - - - - - - - - - - - - - - -
\subsubsection{\lcode{VTK_PROJECTION}}
\lcode{VTK_PROJECTION} is one of:
\begin{itemize}
\item[]'\lcode{GLOBAL}': center of chunk(s) in \lcode{CLAT}, \lcode{CLON}, with applied \lcode{CHUNKS_INVGRID_GEOM_ROT}
\item[]'\lcode{LOCAL_CURV}': center of chunk in x=y=0, NO \lcode{CHUNKS_INVGRID_GEOM_ROT} applied, but with curvature
\item[]'\lcode{LOCAL_FLAT}': center of chunk in x=y=0, NO \lcode{CHUNKS_INVGRID_GEOM_ROT}, NO curvature, i.e. projection of each depth onto the tangential x-y-plane
\item[]'\lcode{LOCAL_NORTH_CURV}': same as '\lcode{LOCAL_CURV}', but WITH \lcode{CHUNKS_INVGRID_GEOM_ROT} applied, i.e. x really points to south
\item[]'\lcode{LOCAL_NORTH_FLAT}': same as '\lcode{LOCAL_FLAT}', but WITH \lcode{CHUNKS_INVGRID_GEOM_ROT} applied, i.e. x really points to south
\end{itemize}
%- - - - - - - - - - - - - - - - - - - - - - - - - - - - -
\subsubsection{\lcode{VTK_GEOMETRY_TYPE}}
Select the geometry type. \lcode{VTK_GEOMETRY_TYPE} is one of:
\begin{itemize}
\item[]'\lcode{CELLS}': data on inversion grids will be written on the volumetric inversion grid CELLS (hexahedral) to vtk files (as \lcode{UNSTRUCTURED_GRID} datasets) $\rightarrow$ intuitive volumetric view
\item[]'\lcode{CELL_CENTERS}': data on inversion grids will be written on the cell center POINTS to vtk files (as \lcode{POLYDATA} datasets) $\rightarrow$ smaller files, better to apply filters in ParaView
\end{itemize}
%- - - - - - - - - - - - - - - - - - - - - - - - - - - - -
\subsubsection{\lcode{SCALE_VTK_COORDS}, \lcode{VTK_COORDS_SCALING_FACTOR}}
Scale vtk geometry coordinates by factor \lcode{VTK_COORDS_SCALING_FACTOR}, if \lcode{SCALE_VTK_COORDS = .True.}
this may be helpful if coordinate values (e.g. in m) get so large that they cause problems in paraview.
%
%----------------------------------------------------------
\subsection{\lcodetitle{schunkInversionGrid}} \label{basic_steps,sec:invgrid,sub:schunk}
%----------------------------------------------------------
The spherical chunk inversion grid is derived from a simple planar cartesian
grid by projection of each grid plane onto a spherical shell, assuming that
the cartesian grid plane touches the spherical shell at its center.
Since cells have uniform size in the planar cartesian grid plane, cells
on the spherical shell become successively smaller with distance from the 
center of the spherical chunk. Radially, the inversion grid may consist of
several layer blocks each of which may have its own cell size.\\
To describe points in the spherical chunk, we either use LOCAL cartesian coordinates
with the center of the chunk at x=y=0 and z=r, or GLOBAL cartesian coordinates
with z pointing towards the north pole, x pointing towards the equator at lon=0
and y pointing towards the equator at lon=90.\\
The center of the chunk may be shifted to any place on the sphere and, in addition,
may be rotated counterclockwise around its local vertical.

\begin{figure}[ht]
  \centering
  \includegraphics[width=0.65\textwidth]{images/schunkInversionGrid_manual.png}
  \caption{Example of a simple chunk inversion grid (global projection, note the slight curvature of the small chunk)}
  \label{basic_steps,sec:invgrid,sub:schunk,fig:grid}
\end{figure}

The inversion grid is defined via a paramter file, a template of which is file 
\lcode{template/schunkInversionGrid_parfile_template} (the template file defines the inversion grid
shown in the above figure).
The meaning of the keywords in the parameter file defining an inversion grid of type \lcode{schunkInversionGrid}
are as follows:

\subsubsection{\lcode{SCHUNK_INVGRID_CLAT},\lcode{SCHUNK_INVGRID_CLON}} %\label{files,sec:iter_parfile,itm:invgrid_rad} 
Geographic latitude/longitude of center of spherical chunk in degrees.
%- - - - - - - - - - - - - - - - - - - - - - - - - - - - -
\subsubsection{\lcode{SCHUNK_INVGRID_RMAX}}
Maximum radius (upper boundary of spherical chunk) in km.
%- - - - - - - - - - - - - - - - - - - - - - - - - - - - -
\subsubsection{\lcode{SCHUNK_INVGRID_WLAT}, \lcode{SCHUNK_INVGRID_WLON}}
Assuming \lcode{ROT = 0}.\\
Width of spherical chunk parallel to longitude/latitude in degrees.
%- - - - - - - - - - - - - - - - - - - - - - - - - - - - -
\subsubsection{\lcode{SCHUNK_INVGRID_ROT}}
Rotation of chunk around local vertical counterclockwise in degrees.
%- - - - - - - - - - - - - - - - - - - - - - - - - - - - -
\subsubsection{\lcode{SCHUNK_INVGRID_NREF_BLOCKS}}
Number of blocks of layers within which lateral cell width remains constant.\\
(compare \lcode{scartInversionGrid}~\myref{basic_steps,sec:invgrid,sub:scart}: definition and usage of \lcode{NREF_BLOCKS}, 
\lcode{NLAY}, \lcode{THICKNESS}, \lcode{NX}, \lcode{NY})
%- - - - - - - - - - - - - - - - - - - - - - - - - - - - -
\subsubsection{\lcode{SCHUNK_INVGRID_NLAY}}
Number of layers within each block (integer array).\\
(compare \lcode{scartInversionGrid}~\myref{basic_steps,sec:invgrid,sub:scart}: definition and usage of \lcode{NREF_BLOCKS}, 
\lcode{NLAY}, \lcode{THICKNESS}, \lcode{NX}, \lcode{NY})
%- - - - - - - - - - - - - - - - - - - - - - - - - - - - -
\subsubsection{\lcode{SCHUNK_INVGRID_THICKNESS}}
Thickness of layers in each block in km (real array).\\
(compare \lcode{scartInversionGrid}~\myref{basic_steps,sec:invgrid,sub:scart}: definition and usage of \lcode{NREF_BLOCKS}, 
\lcode{NLAY}, \lcode{THICKNESS}, \lcode{NX}, \lcode{NY})
%- - - - - - - - - - - - - - - - - - - - - - - - - - - - -
\subsubsection{\lcode{SCHUNK_INVGRID_NLAT}, \lcode{SCHUNK_INVGRID_NLON}}
Number of cells in latitude/longitude direction for each layer block.\\
(compare \lcode{scartInversionGrid}~\myref{basic_steps,sec:invgrid,sub:scart}: definition and usage of \lcode{NREF_BLOCKS}, 
\lcode{NLAY}, \lcode{THICKNESS}, \lcode{NX}, \lcode{NY})
%- - - - - - - - - - - - - - - - - - - - - - - - - - - - -
\subsubsection{\lcode{VTK_PROJECTION}}
\lcode{VTK_PROJECTION} is one of:
\begin{itemize}
\item[] '\lcode{GLOBAL}': center of chunk in \lcode{CLAT}, \lcode{CLON}, with applied \lcode{SCHUNK_INVGRID_ROT}
\item[] '\lcode{LOCAL_CURV}': center of chunk in x=y=0, NO \lcode{SCHUNK_INVGRID_ROT} applied, but with curvature
\item[] '\lcode{LOCAL_FLAT}': center of chunk in x=y=0, NO \lcode{SCHUNK_INVGRID_ROT} applied, NO curvature, i.e. projection of each depth onto the tangential x-y-plane
\item[] '\lcode{LOCAL_NORTH_CURV}': same as '\lcode{LOCAL_CURV}', but WITH \lcode{SCHUNK_INVGRID_ROT} applied, i.e. x really points to south
\item[] '\lcode{LOCAL_NORTH_FLAT}': same as '\lcode{LOCAL_FLAT}', but WITH \lcode{SCHUNK_INVGRID_ROT} applied, i.e. x really points to south
\end{itemize}
%- - - - - - - - - - - - - - - - - - - - - - - - - - - - -
\subsubsection{\lcode{VTK_GEOMETRY_TYPE}}
Select the geometry type. \lcode{VTK_GEOMETRY_TYPE} is one of:
\begin{itemize}
\item[]'\lcode{CELLS}': data on inversion grids will be written on the volumetric inversion grid CELLS (hexahedral) to vtk files (as \lcode{UNSTRUCTURED_GRID} datasets) $\rightarrow$ intuitive volumetric view
\item[]'\lcode{CELL_CENTERS}': data on inversion grids will be written on the cell center POINTS to vtk files (as \lcode{POLYDATA} datasets) $\rightarrow$ smaller files, better to apply filters in ParaView
\end{itemize}
%- - - - - - - - - - - - - - - - - - - - - - - - - - - - -
\subsubsection{\lcode{SCALE_VTK_COORDS}, \lcode{VTK_COORDS_SCALING_FACTOR}}
Scale all VTK point coordinates by an additional factor,
if \lcode{SCALE_VTK_COORDS = .false.}, then \lcode{VTK_COORDS_SCALING_FACTOR} is ignored.
%- - - - - - - - - - - - - - - - - - - - - - - - - - - - -
%
%----------------------------------------------------------
\subsection{\lcodetitle{scartInversionGrid}} \label{basic_steps,sec:invgrid,sub:scart}
%----------------------------------------------------------
%
A \lcode{S}imple \lcode{CART}esian inversion grid covers a Cartesian cuboid which can
be shifted to a certain location in Cartesian space and may be rotated about the local vertical
axis. Its cells are distributed in layers. Each layer has a certain thickness and a regularly
distributed number of inversion grid cells along each lateral direction of the cuboid.

Please consult the documentation of your forward method (\myref{basic_steps,sec:forward_problem}) if it supports
inversion grids of type \lcode{scartInversionGrid}. \\
All coordinates, e.g.\ of events and stations or wavefield points, are interpreted by this type of inversion grid as
\lcode{X} (first coordinate), \lcode{Y} (second coordinate), \lcode{Z} (third coordinate). Their
units (e.g.\ meters or kilometers) are not assumed by the inversion grid and are essentially defined by the wavefield
points, hence, they might be method dependent and must be overall consistend.\\
Every type of integration weights is supported by this type of inversion grid, except weights of type 
\lcode{6} (external integration weights).

\begin{figure}[ht]
  \centering
  \includegraphics[width=0.5\textwidth]{images/scartInversionGrid_manual.png}
  \caption{Example of a simple Cartesian inversion grid}
  \label{basic_steps,sec:invgrid,sub:scart,fig:grid}
\end{figure}

The shape of the cuboid, as well as the distribution of inversion grid cells, are defined 
via a paramter file, a template of which is file \lcode{template/scartInversionGrid_parfile_template}.
In the following, the particular parameters are explained, with the example
values always refering to the inversion grid as displayed in figure~\myref{basic_steps,sec:invgrid,sub:scart,fig:grid}.

%- - - - - - - - - - - - - - - - - - - - - - - - - - - - - 
\subsubsection{\lcode{SCART_INVGRID_CX}}
\lcode{X}-coordinate of center of cuboid (real number)\\
Example:\\
\lcode{SCART_INVGRID_CX = 50.0}
%- - - - - - - - - - - - - - - - - - - - - - - - - - - - - 
\subsubsection{\lcode{SCART_INVGRID_CY}}
\lcode{Y}-coordinate of center of cuboid (real number)\\
Example:\\
\lcode{SCART_INVGRID_CY = -30.0}
%- - - - - - - - - - - - - - - - - - - - - - - - - - - - - 
\subsubsection{\lcode{SCART_INVGRID_ZMAX}}
Maximum \lcode{Z}-coordinate of cuboid (real number), i.e.\ \lcode{Z}-coordinate of the 
``surface'' of the inversion grid\\
Example:\\
\lcode{SCART_INVGRID_ZMAX = 0.0}
%- - - - - - - - - - - - - - - - - - - - - - - - - - - - - 
\subsubsection{\lcode{SCART_INVGRID_WX}}
Width of cuboid in \lcode{X}-direction (real number)\\
Example:\\
\lcode{SCART_INVGRID_WX = 100.0}
%- - - - - - - - - - - - - - - - - - - - - - - - - - - - - 
\subsubsection{\lcode{SCART_INVGRID_WY}}
Width of cuboid in \lcode{Y}-direction (real number)\\
Example:\\
\lcode{SCART_INVGRID_WY = 150.0}
%- - - - - - - - - - - - - - - - - - - - - - - - - - - - - 
\subsubsection{\lcode{SCART_INVGRID_ROT}}
Angle in degrees of anti-clockwise rotation about the local \lcode{Z}-axis through the 
lateral center of the cuboid (real number)\\
Example:\\
\lcode{SCART_INVGRID_ROT = 60.0}
%- - - - - - - - - - - - - - - - - - - - - - - - - - - - - 
\subsubsection{\lcode{SCART_INVGRID_NREF_BLOCKS,SCART_INVGRID_NLAY,SCART_INVGRID_THICKNESS}}
For an arbitrary number of \lcode{SCART_INVGRID_NREF_BLOCKS} blocks of layers,
the vectors \lcode{SCART_INVGRID_NLAY} (integer values) and \lcode{SCART_INVGRID_THICKNESS} (real values), 
both of length \lcode{SCART_INVGRID_NREF_BLOCKS}, define the \lcode{Z}-direction refinement of each block,
whereby \lcode{SCART_INVGRID_NLAY(i)} defines the number of layers in block \lcode{i}, and 
\lcode{SCART_INVGRID_THICKNESS(i)} defines the thickness of all layers contained in block \lcode{i}.\\
Hence, the overall \lcode{Z}-direction coverage of the inversion grid is defined by 
\lcode{SCART_INVGRID_ZMAX} (which is the coordinate of the top of the first layer in the first refinement block)
and \lcode{SCART_INVGRID_ZMAX - SUM_i(THICKNESS(i) * NLAY(i))} (coordinate of the bottom of the last layer in
%and \lstinline[breaklines=true]$SCART_INVGRID_ZMAX - SUM_i(THICKNESS(i) * NLAY(i))$ (coordinate of the bottom of the last layer in
%and \nolinkurl{SCART_INVGRID_ZMAX - SUM_i(THICKNESS(i) * NLAY(i))} (coordinate of the bottom of the last layer in
last refinement block).\\
Example:\\
\lcode{SCART_INVGRID_NREF_BLOCKS =  3}\\
\lcode{SCART_INVGRID_NLAY =  4   5   2}\\
\lcode{SCART_INVGRID_THICKNESS =  5.0  10.0  20.0}
%- - - - - - - - - - - - - - - - - - - - - - - - - - - - - 
\subsubsection{\lcode{SCART_INVGRID_NX}}
Vector (of length \lcode{SCART_INVGRID_NREF_BLOCKS}) of integer values, defining number of 
inversion grid cells in \lcode{X}-direction, one value for each refinement block\\
Example:\\
\lcode{SCART_INVGRID_NX = 20 10 6}
%- - - - - - - - - - - - - - - - - - - - - - - - - - - - - 
\subsubsection{\lcode{SCART_INVGRID_NY}}
Vector (of length \lcode{SCART_INVGRID_NREF_BLOCKS}) of integer values, defining number of 
inversion grid cells in \lcode{Y}-direction, one value for each refinement block\\
Example:\\
\lcode{SCART_INVGRID_NX = 30 15 9}
%- - - - - - - - - - - - - - - - - - - - - - - - - - - - - 
\subsubsection{\lcode{USE_LOCAL_INVGRID_COORDS_FOR_VTK}}
Logical value to indicate whether to use local inversion grid coordinates for vtk geometry, i.e.\ no rotation 
by \lcode{SCART_INVGRID_ROT} and no shift by \lcode{SCART_INVGRID_CX}, \lcode{SCART_INVGRID_CY}, 
\lcode{SCART_INVGRID_ZMAX} (cuboid centered in \lcode{X=Y=0} and \lcode{ZMAX=0})\\
Example:\\
\lcode{USE_LOCAL_INVGRID_COORDS_FOR_VTK = .false.}
%- - - - - - - - - - - - - - - - - - - - - - - - - - - - -
\subsubsection{\lcode{VTK_GEOMETRY_TYPE}}
Select the geometry type. \lcode{VTK_GEOMETRY_TYPE} is one of:
\begin{itemize}
\item[]'\lcode{CELLS}': data on inversion grids will be written on the volumetric inversion grid CELLS (hexahedral) to vtk files (as \lcode{UNSTRUCTURED_GRID} datasets) $\rightarrow$ intuitive volumetric view
\item[]'\lcode{CELL_CENTERS}': data on inversion grids will be written on the cell center POINTS to vtk files (as \lcode{POLYDATA} datasets) $\rightarrow$ smaller files, better to apply filters in ParaView
\end{itemize}
%- - - - - - - - - - - - - - - - - - - - - - - - - - - - - 
\subsubsection{\lcode{SCALE_VTK_COORDS,VTK_COORDS_SCALING_FACTOR}}
Scale vtk geometry coordinates by factor \lcode{VTK_COORDS_SCALING_FACTOR} (real number), if 
\lcode{SCALE_VTK_COORDS = .true.}. This may be helpful if coordinate values (e.g.\ in meters) 
get so large that they cause problems when plotting in paraview.\\
Example:\\
\lcode{SCALE_VTK_COORDS = .false.}\\
\lcode{VTK_COORDS_SCALING_FACTOR = 1.0}
%- - - - - - - - - - - - - - - - - - - - - - - - - - - - - 
%
%----------------------------------------------------------
\subsection{\lcodetitle{ecartInversionGrid}} \label{basic_steps,sec:invgrid,sub:ecart}
%----------------------------------------------------------
%
{\bf EXPERIMENTAL FEATURE:} \emph{so far, this type of inversion grid works for tetrahedral cells only,
since support for hexahedreal cells is not completed throughout (neighbour detection not yet implemented,
no localization of wavefield points / computation of integration weights implemented, no computation of cell volume). 
However, even for tetrahedra the automatic detection 
of neighbours did not work properly in some test cases (compare \ASKI{} developers manual, ``To do'' section)! 
The neighbours are only required for model smoothing in case of regularizing the Kernel system of equations by smoothing
conditions (and possibly for importing inverted models in the forward code, depends on the forward method in use).
If you need neighbours, please check them by executable} \lcode{invgrid2vtk} \emph{along with option} \lcode{-all_nb} \emph{. Execute} 
\lcode{invgrid2vtk} \emph{(without arguments)} \myaref{programs_scripts,sec:bin_prog,sec:invgrid_vtk} \emph{for further details on how to use it.
Otherwise, kernel computation and pre-integration of kernels onto the inversion grid, as well as plotting 
etc., should work properly! Alternatively you may implement an alternative smoothing method that does not require neighbours
(compare the suggestions in the \ASKI{} developers manual, ``To do'' section).}

An \lcode{E}xternal \lcode{CART}esian inversion grid is defined by several text files containing the definintion 
of nodes (i.e.\  essentially the corner points, or rather the control nodes of the inversion grid cells) and the 
definition of cells by refering to the nodes. At the moment, 4-node tetrahedral cells are fully supported, and 8-node 
hexahedral cells are partly supported. 
Those files may be produced by any meshing tool. In case you are interested to export meshes your own way, 
section~\myref{files,sec:ecart_invgrid} defines the required file formats.\\
\ASKI{} provides the \lcode{python} module \lcode{cubit2ASKIecartInversionGrid.py} which can be used with the 
meshing software \lcode{Cubit} in a \lcode{python} script by first importing the module:\\
\lcode{import cubit2ASKIecartInversionGrid}\\
and at the very end of your meshing process calling:\\
\lcode{cubit.cmd('compress all')}\\
\lcode{cubit2ASKIecartInversionGrid.export2ASKI('EXPORT_PATH')}\\
whereby you may replace \lcode{EXPORT_PATH} by some location where the output files will be written.

Please consult the documentation of your forward method (\myref{basic_steps,sec:forward_problem}) if it supports
inversion grids of type \lcode{ecartInversionGrid}. \\
All coordinates, e.g.\ of events and stations or wavefield points, are interpreted by this type of inversion grid as
\lcode{X} (first coordinate), \lcode{Y} (second coordinate), \lcode{Z} (third coordinate). Their
units (e.g.\ meters or kilometers) are not assumed by the inversion grid and are essentially defined by the wavefield
points, hence, they might be method dependent and must be overall consistend.\\
Every type of integration weights is supported by this type of inversion grid, except weights of type 
\lcode{6} (external integration weights).

\begin{figure}[ht]
  \centering
  \includegraphics[width=0.5\textwidth]{images/ecartInversionGrid_manual.png}
  \caption{Example of an external Cartesian inversion grid created by Cubit}
  \label{basic_steps,sec:invgrid,sub:ecart,fig:grid}
\end{figure}

The nodes and cell files, e.g.\ produced by \lcode{Cubit}, are referred to in a parameter file, a template 
of which is file \lcode{template/ecartInversionGrid_parfile_template}. In the following, the particular 
parameters are explained. An example inversion grid of this type is displayed in 
figure~\myref{basic_steps,sec:invgrid,sub:ecart,fig:grid}.

%- - - - - - - - - - - - - - - - - - - - - - - - - - - - - 
\subsubsection{\lcode{ECART_INVGRID_USE_NODES_COMMON}}
Logical value to indicate whether to use one common nodes coordinates file for 
all cell types (only use parameter \lcode{ECART_INVGRID_FILE_NODES} below), or to use an individual 
nodes coordinates file for each cell type (use parameters files \lcode{ECART_INVGRID_FILE_NODES_TET4}, 
\lcode{ECART_INVGRID_FILE_NODES_HEX8}, \dots below).\\
When using module \lcode{cubit2ASKIecartInversionGrid} you should set:\\
\lcode{ECART_INVGRID_USE_NODES_COMMON = .True.}
%- - - - - - - - - - - - - - - - - - - - - - - - - - - - - 
\subsubsection{\lcode{ECART_INVGRID_FILE_NODES_COMMON}}
File name relative to \lcode{MAIN_PATH_INVERSION/ITERATION_STEP_PATH/} of nodes coordinates file to be commonly used
for definition of cells of all types in case of \lcode{ECART_INVGRID_USE_NODES_COMMON = .True.}\\
When using module \lcode{cubit2ASKIecartInversionGrid} you should set:\\
\lcode{ECART_INVGRID_FILE_NODES_COMMON = node_coordinates}
%- - - - - - - - - - - - - - - - - - - - - - - - - - - - - 
\subsubsection{\lcode{ECART_INVGRID_FILE_NODES_TET4}}
File name relative to \lcode{MAIN_PATH_INVERSION/ITERATION_STEP_PATH/} of nodes coordinates file to be used
for definition of tet4-type cells in case of \lcode{ECART_INVGRID_USE_NODES_COMMON = .False.}\\
%- - - - - - - - - - - - - - - - - - - - - - - - - - - - - 
\subsubsection{\lcode{ECART_INVGRID_FILE_NODES_HEX8}}
File name relative to \lcode{MAIN_PATH_INVERSION/ITERATION_STEP_PATH/} of nodes coordinates file to be used
for definition of hex8-type cells in case of \lcode{ECART_INVGRID_USE_NODES_COMMON = .False.}\\
%- - - - - - - - - - - - - - - - - - - - - - - - - - - - - 
\subsubsection{\lcode{ECART_INVGRID_FILE_CELLS_TET4}}
File name relative to \lcode{MAIN_PATH_INVERSION/ITERATION_STEP_PATH/} of cell connectivity file for 
definition of tet4-type cells.\\
When using module \lcode{cubit2ASKIecartInversionGrid} you should set:\\
\lcode{ECART_INVGRID_FILE_CELLS_TET4 = cell_connectivity_tet4}
%- - - - - - - - - - - - - - - - - - - - - - - - - - - - - 
\subsubsection{\lcode{ECART_INVGRID_FILE_CELLS_HEX8}}
File name relative to \lcode{MAIN_PATH_INVERSION/ITERATION_STEP_PATH/} of cell connectivity file for 
definition of hex8-type cells.\\
When using module \lcode{cubit2ASKIecartInversionGrid} you should set:\\
\lcode{ECART_INVGRID_FILE_CELLS_HEX8 = cell_connectivity_hex8}
%- - - - - - - - - - - - - - - - - - - - - - - - - - - - - 
\subsubsection{\lcode{ECART_INVGRID_FILE_NEIGHBOURS}}
File name relative to \lcode{MAIN_PATH_INVERSION/ITERATION_STEP_PATH/} of cell neighbours file. If not
present, this file will be created when first using the inversion grid. If present, its content defines 
the neighbour structure of the inversion grid cells. If, however, the inversion grid
is to be recreated (e.g.\ when calling \lcode{initBasics -recr}, see section~\myref{basic_steps,sec:initBasics}),
this file is recreated.
%- - - - - - - - - - - - - - - - - - - - - - - - - - - - - 
\subsubsection{\lcode{ECART_INVGRID_FILE_NEIGHBOURS_IS_BINARY}}
Logical value to indicate whether \lcode{ECART_INVGRID_FILE_NEIGHBOURS} should be binary or not.
%- - - - - - - - - - - - - - - - - - - - - - - - - - - - -
\subsubsection{\lcode{VTK_GEOMETRY_TYPE}}
Select the geometry type. \lcode{VTK_GEOMETRY_TYPE} is one of:
\begin{itemize}
\item[]'\lcode{CELLS}': data on inversion grids will be written on the volumetric inversion grid CELLS (hexahedral) to vtk files (as \lcode{UNSTRUCTURED_GRID} datasets) $\rightarrow$ intuitive volumetric view
\item[]'\lcode{CELL_CENTERS}': data on inversion grids will be written on the cell center POINTS to vtk files (as \lcode{POLYDATA} datasets) $\rightarrow$ smaller files, better to apply filters in ParaView
\end{itemize}
%- - - - - - - - - - - - - - - - - - - - - - - - - - - - - 
\subsubsection{\lcode{SCALE_VTK_COORDS,VTK_COORDS_SCALING_FACTOR}}
Scale vtk geometry coordinates by factor \lcode{VTK_COORDS_SCALING_FACTOR} (real number), if 
\lcode{SCALE_VTK_COORDS = .true.}. This may be helpful if coordinate values (e.g.\ in meters) 
get so large that they cause problems when plotting in paraview.\\
Example:\\
\lcode{SCALE_VTK_COORDS = .false.}\\
\lcode{VTK_COORDS_SCALING_FACTOR = 1.0}
%
%----------------------------------------------------------
\subsection{\lcodetitle{specfem3dInversionGrid}} \label{basic_steps,sec:invgrid,sub:specfem3d}
%----------------------------------------------------------
%
An inversion grid of type \lcode{specfem3dInversionGrid} is method dependent and is to be used with 
\lcode{METHOD = SPECFEM3D} only. Whole spectral elements are used as inversion grid cells and all 
GLL points inside such an element as the wavefield points. All information regarding the element 
geometry, including information on neighbour cells and the values of the jacobian for every wavefield 
point contained in an element are read from files which are produced by \lcode{SPECFEM3D} methods.

Every type of integration weights is supported by this type of inversion grid, including weights of type 
\lcode{6} (external integration weights, i.e.\ the very integration weights that already \lcode{SPECFEM3D}
uses to integrate over spectral elements).

Please refer to the documentation of your \lcode{SPECFEM3D} forward method (\myref{basic_steps,sec:forward_problem})
on how to generate any files required for using an inversion grid of type \lcode{specfem3dInversionGrid}.

As with all other types of inversion grids, a parameter file defines any details of the grid, a template 
of which is file \lcode{template/specfem3dInversionGrid_parfile_template}.
In the following, the particular parameters are explained.

%- - - - - - - - - - - - - - - - - - - - - - - - - - - - - 
\subsubsection{\lcode{SPECFEM3D_ASKI_MAIN_FILE}}
File name relative to \lcode{MAIN_PATH_INVERSION/ITERATION_STEP_PATH/} of file from \lcode{SPECFEM3D}
containing all information about the inversion grid (any \lcode{.main} output file)\\
Example:\\
\lcode{SPECFEM3D_ASKI_MAIN_FILE = kernel_displacements/kernel_displ_S001.main}
%- - - - - - - - - - - - - - - - - - - - - - - - - - - - -
\subsubsection{\lcode{VTK_GEOMETRY_TYPE}}
Select the geometry type. \lcode{VTK_GEOMETRY_TYPE} is one of:
\begin{itemize}
\item[]'\lcode{CELLS}': data on inversion grids will be written on the volumetric inversion grid CELLS (hexahedral) to vtk files (as \lcode{UNSTRUCTURED_GRID} datasets) $\rightarrow$ intuitive volumetric view
\item[]'\lcode{CELL_CENTERS}': data on inversion grids will be written on the cell center POINTS to vtk files (as \lcode{POLYDATA} datasets) $\rightarrow$ smaller files, better to apply filters in ParaView
\end{itemize}
%- - - - - - - - - - - - - - - - - - - - - - - - - - - - - 
\subsubsection{\lcode{SCALE_VTK_COORDS,VTK_COORDS_SCALING_FACTOR}}
Scale vtk geometry coordinates by factor \lcode{VTK_COORDS_SCALING_FACTOR} (real number), if 
\lcode{SCALE_VTK_COORDS = .true.}. This may be helpful if coordinate values (e.g.\ in meters) 
get so large that they cause problems when plotting in paraview.\\
Example:\\
\lcode{SCALE_VTK_COORDS = .false.}\\
\lcode{VTK_COORDS_SCALING_FACTOR = 1.0}
%- - - - - - - - - - - - - - - - - - - - - - - - - - - - - 
\subsubsection{\lcode{SPECFEM3D_R_EARTH_KM}}
In case of using \lcode{SPECFEM3D_GLOBE}, define here the maximum radius [km] of the sphere, by which 
Earth is approximated (only used to interpret ``depth'' and ``altitude'' of events and stations when 
plotting to vtk files; ignored for \lcode{SPECFEM3D_Cartesian} applications).\\
Example:\\
\lcode{SPECFEM3D_R_EARTH_KM = 6371.0}
%
%++++++++++++++++++++++++++++++++++++++++++++++++++++++++++
\section{Define a Starting Model} \label{basic_steps,sec:start_model}
%++++++++++++++++++++++++++++++++++++++++++++++++++++++++++
%
There are two possibilities to define an earth model for the forward wave propagation in your first iteration:

On the one hand you may use any (standard) earth model provided by the forward method you are using, if appropriate.

If this is not possible, or the models provided do not meet your needs, you may use executable 
\lcode{createStartmodelKim} along with the inversion grid of your first iteration (which you should 
have already defined) to produce an inverted model file containing some simple model on this inversion grid. 
Executing \lcode{createStartmodelKim} (without arguments) will print a short help message how to use the program 
\myaref{programs_scripts,sec:bin_prog,sec:create_startmod_kim}. Afterwards you may 
export the produced model file to your forward method as explained in section~\myref{basic_steps,sec:export_kim}.
Template files of starting model descriptions may be found in \lcode{template/}.
%
%++++++++++++++++++++++++++++++++++++++++++++++++++++++++++
\section{Export Inverted Model} \label{basic_steps,sec:export_kim}
%++++++++++++++++++++++++++++++++++++++++++++++++++++++++++
%
The executable \lcode{exportKim} exports an inverted model file (``kim'' stands for ``K''ernel 
``I''nverted ``M''odel) along with the respective inversion grid specifications to a text file, 
which may be used to communicate such a model to a forward method or postprocess the model values in any way. 
Executing \lcode{exportKim} (without argument) will print a short help message how to use it 
\myaref{programs_scripts,sec:bin_prog,sec:exp_Kim}.
%
%++++++++++++++++++++++++++++++++++++++++++++++++++++++++++
\section{Solving the Forward Problem} \label{basic_steps,sec:forward_problem}
%++++++++++++++++++++++++++++++++++++++++++++++++++++++++++
%
In the following, all wave propagation codes which are supported by \ASKI{} are listed.
Refer to the given documentation on any details regarding the interaction of the forward codes with \ASKI{}.
\subsection*{\lcode{Gemini II}}
\lcode{Gemini II} (by \cite{friederich_wd1995}) is supported in this release version, for Cartesian as well as spherical setting. 
The \lcode{Gemini} routines producing spectral output for \ASKI{}, however, might not yet be available. 
We hope to be able to provide them in due time. Until now, please contact us via \url{http://www.rub.de/aski}
if you want to use \lcode{Gemini II} as a forward solver for \ASKI{}.
\subsection*{\lcode{SPECFEM3D_Cartesian}}
The Cartesian spectral element code \lcode{SPECFEM3D_Cartesian} (by \cite{TrKoLi08}), 
version \lcode{3.0} is fully supported by this \ASKI{} release version, cf.~\cite{Specfem3D_Cartesian_for_ASKI}.
A copy of the code (extended for use with \ASKI{}), as well as the code package \lcode{SPECFEM3D_Cartesian_for_ASKI}
and documentation is available via \url{https://github.com/seismology-RUB/SPECFEM3D_Cartesian_for_ASKI}
(follow the directions in file \lcode{README.md} and the provided manual).
\subsection*{\lcode{SPECFEM3D_GLOBE}}
The global spectral element code \lcode{SPECFEM3D_GLOBE} (by \cite{TrKoLi08}), 
version \lcode{7.0.0}, is fully supported by this \ASKI{} release version, cf.~\cite{Specfem3D_Globe_for_ASKI}.
A copy of the code (extended for use with \ASKI{}), as well as the code package \lcode{SPECFEM3D_GLOBE_for_ASKI}
and documentation is available via \url{https://github.com/seismology-RUB/SPECFEM3D_GLOBE_for_ASKI}
(follow the directions in file \lcode{README.md} and the provided manual).
\subsection*{\lcode{NEXD}}
The nodal discontinous Galerkin code \lcode{NEXD} (by \cite{Lambrecht.2015}) is fully supported by this 
\ASKI{} release version. \lcode{NEXD}, including \ASKI{} support, will be available 
under terms of the GPL in the near future via \url{http://www.rub.de/nexd}.
%
%++++++++++++++++++++++++++++++++++++++++++++++++++++++++++
\section{Choose Integration Weights} \label{basic_steps,sec:intw}
%++++++++++++++++++++++++++++++++++++++++++++++++++++++++++
%
In order to numerically integrate the sensitivity kernels, which are computed on the wavefield points, 
over the inversion grid cells by a weightet summation of values, there are different 
types of integration weights provided, following different rules of integration.

The integer values of the type have the following meaning:
\begin{itemize}
  \item[] \lcode{0} $\rightarrow$ all weights are the same, \lcode{weight = 1/number_of_points_in_box}, 
    i.e.\ no integration(!), just building the average sensitivity value (e.g.\ convenient for comparison of 
    sensitivities computed with different methods on different forward grids)
  \item[] \lcode{1} $\rightarrow$ Scattered Data Integration, as in \cite{Levin99}, polynomial degree 1
  \item[] \lcode{2} $\rightarrow$ Scattered Data Integration, as in \cite{Levin99}, polynomial degree 2, 
    i.e.\ approximation order 3 (?)
  \item[] \lcode{3} $\rightarrow$ Scattered Data Integration, as in \cite{Levin99}, polynomial degree 3, 
    i.e.\ approximation order 4 (?)
  \item[] \lcode{4} $\rightarrow$ for each cell, compute the highest possible order of Scattered Data Iintegration
    integration after \cite{Levin99} (trying types 3,2,1 (in that order) until computation was successful)
  \item[] \lcode{5} $\rightarrow$ average of function values, multiplied with volume of box (i.e.\ $\sim$ linear integration)
  \item[] \lcode{6} $\rightarrow$ external integration weights, to be used along with a suitable inversion grid 
    (e.g.\ of type \lcode{specfem3dInversionGrid}, see section~\myref{basic_steps,sec:invgrid,sub:specfem3d})
\end{itemize}

A detailed description of some of the integration weights, especially the weights after \cite{Levin99} can be found in 
section~\myref{programs_scripts,sec:fmod_intw}.
%
%++++++++++++++++++++++++++++++++++++++++++++++++++++++++++
\section{Create a Data and Model Space} \label{basic_steps,sec:dmspace}
%++++++++++++++++++++++++++++++++++++++++++++++++++++++++++
%
In order to choose a set of data samples which to invert and a set of model values which to invert for, 
you need to define a data space and a model space. Essentially, if you have $m$ data samples, the space in which 
the data live is just \R m (analogously, for $n$ model values, the model lives in \R n ). You only need to define which
data sample (model value) refers to which dimension (i.e.\ entry in vector) of the data space (model space), respectively.\\
The $m\times n$ sensitivity kernel matrix will then connect a vector of model updates from model space in \R n 
to your specific data vector from \R m.

In the following, it is described how data samples and model values are characterized in this software package and 
how you can choose specific subsets to be used. 
%----------------------------------------------------------
\subsection{How Data Samples are Characterized}
%----------------------------------------------------------
Since \ASKI{} derives a model update in an iteration of FWI by frequency-domain sensitivity equations, 
the data needs to be provided frequency domain, too.\\
A data sample is uniquely characterized by a seismic \emph{source}, a \emph{component} 
of a seismic \emph{station}, and a \emph{frequency}, as well as whether it is the \emph{real} or \emph{imaginary} part 
of the complex spectral values. Refer to \myref{basic_steps,sec:data_general} for details on data in \ASKI{}.
%----------------------------------------------------------
\subsection{How Model Values are Characterized} \label{basic_steps,sec:dmspace,sub:mparam}
%----------------------------------------------------------
A model value is uniquely  characterized by a parameter name (must be a valid parameter name of the model
parametrization as defined by \lcode{MODEL_PARAMETRIZATION} \myref{files,sec:main_parfile,itm:mod_pmtrz}) 
and an inversion grid cell index in valid range.
%----------------------------------------------------------
\subsection{How to Define Data and Model Space: Choosing a Set of Data Samples and Model Values}
%----------------------------------------------------------
Create a text file as described in section \myref{files,sec:dmspace}, e.g.\ by adjusting the provided
template files in \lcode{template/}.
%
%++++++++++++++++++++++++++++++++++++++++++++++++++++++++++
\section{Initiate Basic Requirements} \label{basic_steps,sec:initBasics}
%++++++++++++++++++++++++++++++++++++++++++++++++++++++++++
%
Use the executable \lcode{initBasics}. Executing \lcode{initBasics} (without arguments) 
will print a short help message how to use it \myaref{programs_scripts,sec:bin_prog,sec:in_basics}.

It first checks if all parameters needed are present in the parameter files and then creates all
basic requirements for \ASKI{} operations:\\
It reads in required files like event list and station list files, the wavefield points and the 
kernel reference model. \\
Furthermore, it creates the inversion grid (possibly storing some inversion grid files, dependent 
on the type of grid), localizes the wavefield points inside it and computes 
the integration weights, which are written to file. Once those files exist, \lcode{initBasics} 
and all other programs will always read the integration weights (and possibly (part of) the inversion grid) 
from file, \emph{regardless} of what the parameter files say! So if at some you point want to use 
different integration weights or a different inversion grid, you will have to either delete the respective 
file(s) and rerun \lcode{initBasics}, or run \lcode{initBasics -recr} in order
to recreate them. 

Also a lot of \lcode{.vtk} files with statistics are produced having base filename \lcode{FILEBASE_BASIC_STATS} as 
defined in the parameter file of the current iteration step. Those files mainly regard the inversion grid, 
the wavefield points and the integration weights, where the respective filenames are extended (by something 
with ``.vtk''). It is \emph{highly recommended} to call \lcode{initBasics -recr} in order to assure that all
those \lcode{.vtk} files are produced and to actually have a look at them before continuing any \ASKI{} operation!

If you use a spherical setting, you might find it useful for your plots to generate \lcode{.vtk} files containing 
shore lines in the specific \lcode{VTK_PROJECTION} of the inversion grid in use \myaref{basic_steps,sec:shoreLines}.
%
%++++++++++++++++++++++++++++++++++++++++++++++++++++++++++
\section{Compute Spectral Waveform Sensitivity Kernels} \label{basic_steps,sec:compute_kernels}
%++++++++++++++++++++++++++++++++++++++++++++++++++++++++++
%
The kernels are computed by combining green tensor component(s) and forward wavefield for a given event-station
pair (called ``path'' below. By default, the kernels are integrated over the inverison grid cells. I.e.\ there is one sensitivity 
kernel file for a specific event-station path. This file contains sensitivity values of the requested 
station components for the requested model parameters of your model parametrization, with 
the values living on the inverison grid cells. Alternatively, kernels can be computed on the wavefield points,
without pre-integration (for inspection only, cannot be used for waveform inversion in that case).

Use executable \lcode{computeKernels}. Executing \lcode{computeKernels} (without arguments) 
will print a short help message how to use it \myaref{programs_scripts,sec:bin_prog,sec:com_kernel}.
It makes sense, to only compute kernel files for those event-station paths that you are going to use (defined by
your data-model-space file).

You can define the set of event-station paths, station components and model parameters for which 
sensitivities should be computed in two ways:
\begin{itemize}
\item[way 1] compute a kernel for only one event-station path, defined by eventID and station name using options 
\lcode{-evid}, \lcode{-stname}, \lcode{-comp}, \lcode{-param}
\item[way 2] input a data-model-space file (as defined by \myref{basic_steps,sec:dmspace}) by option 
\lcode{-dmspace}, defining all event-station paths and respective station components and model parameters 
for which kernels should be computed; optionally define range of path index
by \lcode{-ipath1}, \lcode{-ipath2}
\end{itemize}
Setting option \lcode{-wp} will compute plain kernel values on wavefield points (no pre-integration onto inversion
grid). \emph{This option will produce separate files, so you can compute both, kernels on wavefield points and
on inversion grid}.
%
%++++++++++++++++++++++++++++++++++++++++++++++++++++++++++
\section{Transform to Time-Domain Sensitivity Kernels} \label{basic_steps,sec:compute_time_kernels}
%++++++++++++++++++++++++++++++++++++++++++++++++++++++++++
%
The time kernels are computed from the standard frequency-domain kernels (which were computed path-wise)
by applying an inverse Fourier transform. 

Use executable \lcode{spec2timeKernels}. Executing \lcode{spec2timeKernels} (without arguments) will print a short
help message how to use it \myaref{programs_scripts,sec:bin_prog,sec:spec_time_kernels}.

The set of event-station paths, station components and model parameters for which time kernels should be produced
are defined in the same ``two ways'' (by the same options) as computing spectral waveform kernels 
\myaref{basic_steps,sec:compute_kernels}.

Additionally, you need to define the time discretization of the produced time kernel by options 
\lcode{-dt}, \lcode{-nt1}, \lcode{-nt2}, \lcode{-t0}.
%
%++++++++++++++++++++++++++++++++++++++++++++++++++++++++++
\section{Plot Spectral Waveform Sensitivity Kernels} \label{basic_steps,sec:plot_kernels}
%++++++++++++++++++++++++++++++++++++++++++++++++++++++++++
%
One way to plot a specific sensitivity Kernel in frequency domain, i.e.\ the sensitivity spectra for a specific
event-station path, is to procuce vtk files using executable \lcode{kernel2vtk}. 
Executing \lcode{kernel2vtk} will print a short help message how to use it \myaref{programs_scripts,sec:bin_prog,sec:kernel_2_vtk}.

Please note, that the output \lcode{.vtk} files (one for every frequency) might get large, dependent on the resolution 
of the inversion grid, since the geometry information of the inversion grid cells is contained in each \lcode{.vtk} file.
If the files become too large for you, you might consider setting \lcode{DEFAULT_VTK_FILE_FORMAT = BINARY} in your 
main parameter file.
%
%++++++++++++++++++++++++++++++++++++++++++++++++++++++++++
\section{Plot Time Sensitivity Kernels} \label{basic_steps,sec:plot_time_kernels}
%++++++++++++++++++++++++++++++++++++++++++++++++++++++++++
%
One way to plot a specific sensitivity Kernel in time domain, is to procuce vtk files using executable \lcode{timeKernel2vtk}. 
Executing \lcode{timeKernel2vtk} (without arguments) will print a short help message how to use it \myaref{programs_scripts,sec:bin_prog,sec:timeKernel_2_vtk}.\\

Please note, that the output \lcode{.vtk} files (one for every time step) might get large, dependent on the resolution 
of the inversion grid, since the geometry information of the inversion grid cells is contained in each \lcode{.vtk} file.
If the files become too large for you, you might consider setting \lcode{DEFAULT_VTK_FILE_FORMAT = BINARY} in your 
main parameter file.
For a lot of time steps, the \lcode{.vtk} file format is actually not optimally chosen by \ASKI{}, since the complete geometry
information is contained in every time-step file (a lot of redundant information is written to hard disc).
%
%++++++++++++++++++++++++++++++++++++++++++++++++++++++++++
\section{Solve Kernel System} \label{basic_steps,sec:solve_kernel_system}
%++++++++++++++++++++++++++++++++++++++++++++++++++++++++++
%
At the moment, there are 3 executable programs implemented which solve the kernel linear system.
\begin{itemize}
\item
\lcode{solveKernelSystem} is a serial program which sets up the kernel matrix, 
reads in synthetic and measured data, adds regularization condutions to the system if 
requested (smoothing, damping) and solves the system based on \lcode{LAPACK} libraries.\\
Executing \lcode{solveKernelSystem} (without arguments) will print a short help message how to use it 
\myaref{programs_scripts,sec:bin_prog,sec:sol_Ker_Sys}.
\item 
\lcode{solveParKernelSystem} does essentially the same as \lcode{solveKernelSystem}, but uses
parallelized \lcode{ScaLAPACK} libraries to solve the (regularized) kernel linear system. You
must compile executable \lcode{solveParKernelSystem} separately (\lcode{make all} does not
compile it, make sure you have \lcode{ScaLAPACK} libraries installed and linked in the 
\lcode{Makefile}).
Executing \lcode{solveParKernelSystem} (without arguments) will print a short help message how to use it 
\myaref{programs_scripts,sec:bin_prog,sec:solve_par_kernel_sys}.
This executable requires an additional parameter file, see template file\\
\lcode{templated/solveParKernelSystem_parfile_template}.
\item 
\lcode{solveCglsrKernelSystem} sets up the (regularized) kernel linear system, as the above
excecutables do, but the uses a conjugate gradient algorithm to solve the kernel linear system
in a least-squares sense \myaref{programs_scripts,sec:bin_prog,sec:solve_cgls_kernel_sys}.
This executable requires an additional parameter file, see template file\\
\lcode{templated/solveCglsKernelSystem_parfile_template}.\\
Executing \lcode{solveCglsKernelSystem} (without arguments) will print a short help message how to use it 
\myaref{programs_scripts,sec:bin_prog,sec:solve_cgls_kernel_sys}.
\end{itemize}
%
%++++++++++++++++++++++++++++++++++++++++++++++++++++++++++
\section{Investigate State of Convergence of Waveform Inversion} \label{basic_steps,sec:investigate_convergence}
%++++++++++++++++++++++++++++++++++++++++++++++++++++++++++
%
These kinds of operations probalby still need further development in \ASKI{}. 

Whenever solving the linear system of kernel equations in an interation step of full waveform inversion
(see above section~\myref{basic_steps,sec:solve_kernel_system}), the current linear misfit is
printed to output files or on screen, i.e.\ the
the sum of residual squares of the linear system of equations. Also the current misfit between synthetics 
and measured data can be computed by executable \lcode{computeMisfit}
\myaref{programs_scripts,sec:bin_prog,sec:comp_misfit}.
However, comparing these two during variation of the regularization parameters does not give sensible
information on which regularization to choose. The development of the non-linear misfit, i.e.\ comparing
synthetic data based on the new models with the measured data, can only be observed after solving the
forward problem for each of the models to choose from. There is not yet an automated way implemented in
\ASKI{} that does this. Evaluating the non-linear misfit, i.e.\ conducting suitable forward calculations, 
would have to be done by hand for the selected models resulting from the selected regularization parameters.

If you would like to do traditional resolution tests, 
executable \lcode{computeDataFromKernelSystem} \myaref{programs_scripts,sec:bin_prog,sec:comp_data_kernel_sys}
computes the ``measured'' data $d$ in the equation $d-s = K (m1 - mref)$ for any given model m1.
This can be used for traditional linear computation of checkerboard data by forward multiplication 
of the sensitivity matrix with an artificial model vector containing checker anomalies.
A model containing checker or spike-like anomalies can be produced by the preliminary hard-coded executable
\lcode{addSpikeCheckerToKim} \myaref{programs_scripts,sec:bin_prog,sec:addSpikeCheckerToKim}.
%
%++++++++++++++++++++++++++++++++++++++++++++++++++++++++++
\section{Generate Shore Line Vtk Files} \label{basic_steps,sec:shoreLines}
%++++++++++++++++++++++++++++++++++++++++++++++++++++++++++
%
\ASKI{} provides two implementational realizations for creating shore line \lcode{.vtk} files from GSHHS data 
provided in native binary format. The current version of this widely used geographical dataset is available
via \url{https://www.ngdc.noaa.gov/mgg/shorelines/data/gshhg/latest/}. You should download the 
dataset in form of native binary files (probably package named like \lcode{gshhg-bin-?.?.?.zip}).

The files \lcode{gshhs_c.b}, \lcode{gshhs_l.b}, \lcode{gshhs_i.b}, \lcode{gshhs_h.b}, \lcode{gshhs_f.b} 
(different resolutions), can be
used as an input for the following executables.

%----------------------------------------------------------
\subsection{excutable purely written in Fortran}
%----------------------------------------------------------
The executable \lcode{createShoreLines} \myaref{programs_scripts,sec:bin_prog,sec:shore_lines} 
is purely written in Fortran and is by default compiled when you compile the \ASKI{} binaries.

%----------------------------------------------------------
\subsection{python program based on Fortran-to-python interface generator \lcodetitle{f2py}} \label{basic_steps,sec:shoreLines,ssec:py}
%----------------------------------------------------------
The python program \lcode{create_shore_lines.py} \myaref{programs_scripts,sec:py,sec:shore_lines}
is based on the Fortran-to-python interface generator \lcode{f2py} and requires an additional module to be
compiled manually before you are able to use it.

Make sure your system has installed the \lcode{f2py} interface generator.

Go to the \ASKI{} installation directory \lcode{ASKI_1.1/}. In \lcode{Makefile}, adjust the \lcode{f2py_COMPILER}
and \lcode{BLAS_F2PY}, if necessary. Excecute\\
\lcode{make shore_lines_module_for_python}

The additional module \lcode{create_shore_lines_f2py.so} should have been created in directory \lcode{ASKI_1.1/py}
which is neede by python program \lcode{create_shore_lines.py}.

%
%++++++++++++++++++++++++++++++++++++++++++++++++++++++++++
\section{Path-Specific Approach to an Iteration of Waveform Inversion} \label{basic_steps,sec:path_specific}
%++++++++++++++++++++++++++++++++++++++++++++++++++++++++++
%
\ASKI{} supports a ``path-specific'' approach to an iteration of full waveform inversion, which is enabled
by flag \lcode{USE_PATH_SPECIFIC_MODELS} in the iteration step parameter file \myaref{files,sec:iter_parfile}.

This approach was applied by \cite{Lamara.2015b} in his first iteration of full waveform inversion
using the 1D code Gemini as a forward solver. Due to the complex structure of the investigated region
(Aegean), it was infeasible to choose a single 1D background model for solving the total forward problem. 
Therefore, an individual 1D background model was chosen for each event-station path which explained
the particular seismogram of that path very well. 

This path-specific concept requires to account for different kernel reference models and to introduce a
correction term in the linear system of kernel equations in order to invert for a global 3D model. Any 
required functionality is implemented in \ASKI{} and was successfully applied. We refer to \cite{Lamara.2015b}
and \cite{Schumacher16} on a more detailed description of the concept.

The required quantity of the correction term, called synthetic correction, can be computed in \ASKI{} by 
executable \lcode{computeCorrectionSyntheticData} \myaref{programs_scripts,sec:bin_prog,sec:comp_correct_syn_data}.

%
%-------------------------------
% APPENDIX CONTAINING CHAPTERS WITH MORE DETAILS
%\appendix
%
%-------------------------------
% APPENDIX CHAPTER file formats
\chapter{Files} %\label{files}
%-----------------------------------------------------------------------------
%   Copyright 2013 Florian Schumacher
%
%   This file is part of the ASKI manual as a LaTeX document with main file
%   manual.tex
%
%   Permission is granted to copy, distribute and/or modify this document
%   under the terms of the GNU Free Documentation License, Version 1.3
%   or any later version published by the Free Software Foundation;
%   with no Invariant Sections, no Front-Cover Texts, and no Back-Cover Texts.
%   A copy of the license is included in the section entitled ``GNU
%   Free Documentation License''. 
%-----------------------------------------------------------------------------
%
This chapter collects documentation on file formats involved in \ASKI. 
%
%++++++++++++++++++++++++++++++++++++++++++++++++++++++++++
\section{Parameter Files} \label{files,sec:parfiles}
%++++++++++++++++++++++++++++++++++++++++++++++++++++++++++
%
Parameter files are simple text files.

The following type of lines are ignored:
\begin{itemize}
\item comment lines, i.e.\ lines STARTING with an arbitrary number of blanks followed by a ``\#'' character
\item empty lines and lines containing blanks only
\item lines not containing any ``='' character
\end{itemize}

How to specify one parameter:
\begin{itemize}
\item valid lines have the form ``keyword = value'' (blanks leading or following ``keyword'', ``='', or ``value'' are ignored)
\item in a valid line, all characters in front of ``='' (without leading and appending blanks)
are interpreted as the keyword, allowing for blank characters within the keyword (e.g.\ for lines  
\mbox{``\hspace{5mm}key~word~=~value\hspace{3mm}''}, the string ``key~word'' is used as the keyword)
\item all characters behind ``='' (without leading and appending blanks) are interpreted as the value string from 
which the value is read, which in particular means that ``\#'' comments at the end of a line (such as 
\mbox{``\hspace{3mm}keyword = value\hspace{2mm}\# comment\hspace{3mm}''}) are \emph{not} allowed!
\end{itemize}
By convention, specify \emph{paths} (i.e.\ directory names, which will be concatenated with a filename of 
a file in that directory) always ending on ``/'' and specify \emph{filenames} always \emph{without} leading ``/''.
%
%----------------------------------------------------------
\subsection{Main Parameter File} \label{files,sec:main_parfile}
%----------------------------------------------------------
%
Here, shortly all keywords required in the main parameter file for your specific program operation, are described.
%- - - - - - - - - - - - - - - - - - - - - - - - - - - - - 
\subsubsection{\lcode{FORWARD_METHOD}} \label{files,sec:main_parfile,itm:forward_method}
\begin{itemize}
\item[] \lcode{GEMINI}
\item[] \lcode{SPECFEM3D} $\rightarrow$ \lcode{SPECFEM3D_Cartesian} , \lcode{SPECFEM3D_GLOBE}
\end{itemize}
For details on the methods and references to their documentation, refer to section~\ref{basic_steps,sec:forward_problem}
%- - - - - - - - - - - - - - - - - - - - - - - - - - - - -
\subsubsection{\lcode{MODEL_PARAMETRIZATION}} \label{files,sec:main_parfile,itm:mod_pmtrz}
\begin{itemize}
\item[] \lcode{isoLame} $\rightarrow$ isotropic Lam\´e parameters $\rho$, $\lambda$, $\mu$
\item[] \lcode{isoVelocity} $\rightarrow$ isotropic seismic velocities $\rho$, $v_p$, $v_s$
\end{itemize}
Nothing else supported yet
%- - - - - - - - - - - - - - - - - - - - - - - - - - - - -
\subsubsection{\lcode{MAIN_PATH_INVERSION}} \label{files,sec:main_parfile,itm:main_path}
All subpaths for filenames are considered relative to this main path. This
directory is thought to contain all your relevant output and (temporary) data.\\
Example: \lcode{MAIN_PATH_INVERSION = /scratch/inversions/Aegean1/}
%- - - - - - - - - - - - - - - - - - - - - - - - - - - - - 
\subsubsection{\lcode{CURRENT_ITERATION_STEP}} \label{files,sec:main_parfile,itm:cur_iter_step}
Example: \lcode{CURRENT_ITERATION_STEP = 3}
%- - - - - - - - - - - - - - - - - - - - - - - - - - - - - 
\subsubsection{\lcode{ITERATION_STEP_PATH}} \label{files,sec:main_parfile,itm:iter_path}
Relative to main path, defining name of subdirectory of \lcode{MAIN_PATH_INVERSION} which contains 
all relevant (meta)data of an inversion step. A three-digit integer (= \lcode{CURRENT_ITERATION_STEP}) 
and ``/'' will be appended to \lcode{ITERATION_STEP_PATH} (i.e.\ ``001/'', ``002/'', \dots) defining the 
first, second \dots iteration step directory.\\
Example: \lcode{ITERATION_STEP_PATH = iteration_step_}
%- - - - - - - - - - - - - - - - - - - - - - - - - - - - - 
\subsubsection{\lcode{PARFILE_ITERATION_STEP}} \label{files,sec:main_parfile,itm:iter_parfile}
File name of iteration step specific parameter file, relative to \lcode{MAIN_PATH_INVERSION/ITERATION_STEP_PATH}
Example: \lcode{PARFILE_ITERATION_STEP = iter_parfile}
%- - - - - - - - - - - - - - - - - - - - - - - - - - - - - 
\subsubsection{\lcode{PATH_MEASURED_DATA,PATH_EVENT_FILTER,PATH_STATION_FILTER}} 
\label{files,sec:main_parfile,itm:path_mdata_filters}
Paths where \ASKI finds files related to the measured data files. These paths can be everywhere, e.g.\ close to where you 
have stored/processed your (time domain) data, or in directory \lcode{MAIN_PATH_INVERSION}, etc.\ \dots
The naming convention of files in these directories is:\\
\lcode{FILE_MEASURED_DATA}: \lcode{data_EVENTID_STATIONNAME_COMP},
\lcode{FILE_EVENT_FILTER}: \lcode{filter_EVENTID},
\lcode{FILE_STATION_FILTER}: \lcode{filter_STATIONNAME_COMP}, 
where filters are dependet on component and \lcode{STATIONNAME} and \lcode{EVENTID} are 
defined in \lcode{FILE_STATION_LIST} and \lcode{FILE_EVENT_LIST} file, and \lcode{COMP} 
is a valid component supported by module \lcode{componentTransformation}\\
Example: \\
\lcode{PATH_MEASURED_DATA = /mydata/your_name_of_inversion/ASKI_data/}\\
\lcode{PATH_EVENT_FILTER = /mydata/your_name_of_inversion/ASKI_event_filter/}\\
\lcode{PATH_STATION_FILTER = /mydata/your_name_of_inversion/ASKI_station_filter/}
%- - - - - - - - - - - - - - - - - - - - - - - - - - - - - 
\subsubsection{\lcode{FILE_EVENT_LIST}} \label{files,sec:main_parfile,itm:file_event_list}
Absolute filename where ASKI finds a text file defining a set of events in the required format 
(\ref{files,sec:event_list})\\
Example: \lcode{FILE_EVENT_LIST = /mydata/your_name_of_inversion/ASKI_events}
%- - - - - - - - - - - - - - - - - - - - - - - - - - - - - 
\subsubsection{\lcode{FILE_STATION_LIST}} \label{files,sec:main_parfile,itm:file_station_list}
Absolute filename where ASKI finds a text file defining a set of stations in the required format 
(\ref{files,sec:station_list})\\
Example: \lcode{FILE_STATION_LIST = /mydata/your_name_of_inversion/ASKI_stations}
%- - - - - - - - - - - - - - - - - - - - - - - - - - - - - 
\subsubsection{\lcode{MEASURED_DATA_FREQUENCY_STEP,MEASURED_DATA_NUMBER_OF_FREQ,MEASURED_DATA_INDEX_OF_FREQ}} 
\label{files,sec:main_parfile,itm:mdata_freq}
Discretized frequency window of measured data (same expected in event\_filter/station\_filter!) given by a frequency 
step \lcode{FREQUENCY_STEP} [Hz] and a vector of frequency indices \lcode{INDEX_OF_FREQ}
(of length \lcode{NUMBER_OF_FREQ}), where for specific frequency index $i$ the corresponding frequency $f_i$ [Hz] 
computes to $f_i = i \cdot$ \lcode{FREQUENCY_STEP}\\
Example:\\
\lcode{MEASURED_FREQUENCY_STEP = 10.}\\
\lcode{MEASURED_NUMBER_OF_FREQ = 5}\\
\lcode{MEASURED_INDEX_OF_FREQ = 2 3 5 7 10}\\
which corresponds to the 5 frequencies $20,30,50,70,100$ Hz
%- - - - - - - - - - - - - - - - - - - - - - - - - - - - - 
\subsubsection{\lcode{DEFAULT_VTK_FILE_FORMAT}} 
Either \lcode{BINARY} or \lcode{ASCII} defining the default type of \lcode{vtk} files 
which will be produced in the course of running the programs.
%
%----------------------------------------------------------
\subsection{Parameter File for Specific Iteration Step} \label{files,sec:iter_parfile}
%----------------------------------------------------------
%
Here, shortly all keywords required in a parameter file for a specific iteration step, i.e.\ \\
\lcode{MAIN_PATH_INVERSION/ITERATION_STEP_PATH/PARFILE_ITERATION_STEP} , are described.
%- - - - - - - - - - - - - - - - - - - - - - - - - - - - - 
\subsubsection{\lcode{ITERATION_STEP_NUMBER_OF_FREQ,ITERATION_STEP_INDEX_OF_FREQ}} \label{files,sec:iter_parfile,itm:iter_freq}
Frequency discretization of this iteration step, must be a subset of global frequency discretization 
for this inversion defined as defined by \ref{files,sec:main_parfile,itm:mdata_freq}. \\
\lcode{ITERATION_STEP_NUMBER_OF_FREQ <= MEASURED_DATA_NUMBER_OF_FREQ} and vector \\
\lcode{ITERATION_STEP_INDEX_OF_FREQ} (of length \lcode{ITERATION_STEP_NUMBER_OF_FREQ})
must only contain indices contained in \lcode{MEASURED_DATA_INDEX_OF_FREQ}\\
All indices here are assumed in accordance with the global frequency step \lcode{MEASURED_DATA_FREQUENCY_STEP}
%- - - - - - - - - - - - - - - - - - - - - - - - - - - - - 
\subsubsection{\lcode{TYPE_INVERSION_GRID,PARFILE_INVERSION_GRID}} \label{files,sec:iter_parfile,itm:invgrid}
Type of inversion grid (as supported, cf.\ \ref{basic_steps,sec:invgrid}) and corresponding
filename of parameter file defining this inversion grid, relative to \\
\lcode{MAIN_PATH_INVERSION/ITERATION_STEP_PATH/}
%- - - - - - - - - - - - - - - - - - - - - - - - - - - - - 
\subsubsection{\lcode{TYPE_INTEGRATION_WEIGHTS}}
Type of integration weights (integer number), cf.\ \ref{basic_steps,sec:intw} for supported values.
%- - - - - - - - - - - - - - - - - - - - - - - - - - - - - 
\subsubsection{\lcode{FILE_INTEGRATION_WEIGHTS}} 
Filename of the integration weights file, which will be created and used, relative to 
\lcode{MAIN_PATH_INVERSION}. 
%- - - - - - - - - - - - - - - - - - - - - - - - - - - - -
\subsubsection{\lcode{FILE_WAVEFIELD_POINTS}} 
Filename of the wavefield points file, relative to \lcode{MAIN_PATH_INVERSION}, which
is in general created by the method you are using. Just refer here to this file. 
%- - - - - - - - - - - - - - - - - - - - - - - - - - - - -
\subsubsection{\lcode{FILE_KERNEL_REFERENCE_MODEL}} \label{files,sec:iter_parfile,itm:model}
Dependent on the method you are using, these filenames may be handled individually. Please refer to the respective 
documentation of the methods for recommendations how to use these parameters, or which naming to choose.
%- - - - - - - - - - - - - - - - - - - - - - - - - - - - - 
\subsubsection{\lcode{FILEBASE_BASIC_STATS}}
Base filename of vtk stats output files (related to inversion grid, wavefield points, integration weights,
events, stations), relative to \lcode{MAIN_PATH_INVERSION/ITERATION_STEP_PATH}. 
%- - - - - - - - - - - - - - - - - - - - - - - - - - - - - 
\subsubsection{\lcode{PATH_OUTPUT_FILES}}
Folder relative to which some sensitivity analysis and inversion programs write their output (relatively small output
like models, coefficients etc., NO wavefields/kernels etc.!), relative to 
\lcode{MAIN_PATH_INVERSION/ITERATION_STEP_PATH}.  Be sure, the path ends on ``/".
%- - - - - - - - - - - - - - - - - - - - - - - - - - - - -
\subsubsection{\lcode{PATH_KERNEL_DISPLACEMENTS}} 
Subdirectory of current iteration step path
\lcode{MAIN_PATH_INVERSION/ITERATION_STEP_PATH} which contains the 
kernel displacement files. Be sure, the path ends on ``/".
%- - - - - - - - - - - - - - - - - - - - - - - - - - - - -
\subsubsection{\lcode{PATH_KERNEL_GREEN_TENSORS}} 
Subdirectory of current iteration step path
\lcode{MAIN_PATH_INVERSION/ITERATION_STEP_PATH} which contains the 
kernel green tensor files. Be sure, the path ends on ``/".
%- - - - - - - - - - - - - - - - - - - - - - - - - - - - -
\subsubsection{\lcode{PATH_SENSITIVITY_KERNELS}} 
Subdirectory of current iteration step path
\lcode{MAIN_PATH_INVERSION/ITERATION_STEP_PATH} which contains the 
velocity kernel files. Be sure, the path ends on ``/".
%- - - - - - - - - - - - - - - - - - - - - - - - - - - - -
\subsubsection{\lcode{PATH_SYNTHETIC_DATA}} 
Subdirectory of current iteration step path
\lcode{MAIN_PATH_INVERSION/ITERATION_STEP_PATH} which contains the 
files with synthetic data. Be sure, the path ends on ``/".

%
%++++++++++++++++++++++++++++++++++++++++++++++++++++++++++
\section{Event List File} \label{files,sec:event_list}
%++++++++++++++++++++++++++++++++++++++++++++++++++++++++++
%
Please find the template event list file \lcode{template/file_event_list_template}.

\begin{itemize}
\item first line contains single character ``C'' or ``S'', defining the coordinate system (``C''artesian or ``S''pherical)
  with respect to which the given event coordinates lat,lon are interpreted
\item each following non-empty line of the file is interpreted as a definition of one event and must 
  contain the following space-separated values:
  \begin{itemize}
  \item[eventid]  13 character name, (e.g. \lcode{2006.10.2977} or \lcode{061113_141238}) 
    should \emph{not} contain whitespace!
  \item[origintime]  characters of form \lcode{yyyymmdd_hhmmss_nnnnnnnnn} or \lcode{yyyymmdd_hhmmss}
    (i.e.\ with or without nano-seconds), e.g. \lcode{20130320_170012} or 
    \lcode{20130320_170002_718000000}
  \item[lat] latitude in degrees, \lcode{-90 <= lat <= 90} (``S'') or first coordinate in 
    wavefield points / inversion grid - frame (``C'') $\rightarrow$  read the section on inversion grid definitions 
    (\ref{basic_steps,sec:invgrid})
  \item[lon] longitude in degrees, \lcode{0 <= lon <= 360} (``S'') or second  coordinate in 
    wavefield points / inversion grid -frame (``C'') (read \ref{basic_steps,sec:invgrid})
  \item[depth] source depth in km (``S''), or third coordinate in wavefield points / inversion grid -frame (``C'') 
    (read \ref{basic_steps,sec:invgrid})
  \item[typ] source type:  0 = force, 1 = moment tensor, -1: not specified
  \item[mag] factor on source mechanism
  \item[mom/frce] either 3 values (force vector) or 6 values (moment tensor)
  \end{itemize}
\end{itemize}
%
%++++++++++++++++++++++++++++++++++++++++++++++++++++++++++
\section{Station List File} \label{files,sec:station_list}
%++++++++++++++++++++++++++++++++++++++++++++++++++++++++++
%
Please find the template station list file \lcode{template/file_station_list_template}.

\begin{itemize}
\item first line contains single character ``C'' or ``S'', defining the coordinate system (``C''artesian or ``S''pherical)
  with respect to which the given event coordinates lat,lon are interpreted
\item each following non-empty line of the file is interpreted as a definition of one station and must 
  contain the following space-separated values:
  \begin{itemize}
  \item[station\_name] 5 character name, which should \emph{neither} contain whitespace \emph{nor} underscors ``\_''!
  \item[network\_code] 6 character network code
  \item[lat] latitude in degrees, \lcode{-90 <= lat <= 90} (``S'') or first coordinate in 
    wavefield points / inversion grid - frame (``C'') $\rightarrow$  read the manual on inversion grid definitions 
    (\ref{basic_steps,sec:invgrid})
  \item[lon] longitude in degrees, \lcode{0 <= lon <= 360} (``S'') or second  coordinate in 
    wavefield points / inversion grid -frame (``C'') (read \ref{basic_steps,sec:invgrid})
  \item[elevation] altitude of station (``S''), or third coordinate in wavefield points / inversion grid -frame (``C'') (read \ref{basic_steps,sec:invgrid})
  \end{itemize}
\end{itemize}
%
%++++++++++++++++++++++++++++++++++++++++++++++++++++++++++
\section{Measured Data Files} \label{files,sec:measured_data}
%++++++++++++++++++++++++++++++++++++++++++++++++++++++++++
%
All measured data files are expected to be in the directory \lcode{PATH_MEASURED_DATA} as defined
in the main parameter file.

One measured data file contains all data values for one specific receiver component and a specific event.
Its filename is by convention \lcode{data_EVENTID_STATIONNAME_COMP}

The files are text files containing $1$ column of \lcode{MEASURED_DATA_NUMBER_OF_FREQ} complex numbers, 
which can be understood by \lcode{FORTRAN} \lcode{read} command.

Line \lcode{i} contains measured data values for the \lcode{i}\textsuperscript{th} frequency, as defined by vector
of indices \lcode{MEASURED_DATA_INDEX_OF_FREQ} and frequency step \lcode{MEASURED_DATA_FREQUENCY_STEP}. \\
In particular, this means that \emph{all} measured data files must contain the \emph{same} frequency discretization, given
by parameters \lcode{MEASURED_DATA_INDEX_OF_FREQ}, \lcode{MEASURED_DATA_FREQUENCY_STEP} of the main 
parameter file.
%
%++++++++++++++++++++++++++++++++++++++++++++++++++++++++++
\section{Synthetic Data Files} \label{files,sec:synth_data}
%++++++++++++++++++++++++++++++++++++++++++++++++++++++++++
%
All synthetic data files are expected to be in the directory \lcode{PATH_SYNTHETIC_DATA} as defined 
in the parameter file of the current iteration step.

One synthetic data file contains the complete synthetic data values for one specific path (i.e.\ a specific
source-receiver combination). Its filename is by convention \lcode{synthetics_EVENTID_STATIONNAME}

The files are text files containing \lcode{ITERATION_STEP_NUMBER_OF_FREQ} lines and $3$ columns of 
complex numbers, which can be understood by \lcode{FORTRAN} \lcode{read} command:\\
Line \lcode{i} contains synthetic data values for the \lcode{i}\textsuperscript{th} frequency, as defined by vector
of indices \lcode{ITERATION_STEP_INDEX_OF_FREQ} and frequency step \lcode{MEASURED_DATA_FREQUENCY_STEP}.
The $3$ complex numbers on a line refer to the $3$ Cartesian components \lcode{CX}, \lcode{CY}, \lcode{CX}.
%
%++++++++++++++++++++++++++++++++++++++++++++++++++++++++++
\section{Vtk Files} \label{files,sec:vtk_files}
%++++++++++++++++++++++++++++++++++++++++++++++++++++++++++
%
For visualization of basic objects of the inversion, such as the inversion grid, 
the wavefield points, the integration weights etc., as well as some inversion results
and models, we use the \lcode{vtk} file format.\\
General information on this file format may be found under \url{www.vtk.org/VTK/img/file-formats.pdf}

PUT HERE:\\
General info about the two types of vtk files (invgridVtk, wavefield points Vtk files)\\
Some basic content description about some special vtk files.
%
%++++++++++++++++++++++++++++++++++++++++++++++++++++++++++
\section{Data and Model Space File} \label{files,sec:dmspace}
%++++++++++++++++++++++++++++++++++++++++++++++++++++++++++
%
Files in which a data and model space is defined have the following form. Also have a look at 
example template files \lcode{template/data_model_space_info_template_*}

The blocks described in the subsections below should be put into a file, one after another. The header block must come
first, then the data space block and model parameter block. The order of the latter two is arbitrary, both orders are 
allowed, however if the model parameters block is defined first, an additional check for empty kernel file names will
be done in the processing of the data samples block.

%---------------------------------------------------------
\subsection{Header Block}
%---------------------------------------------------------
{\bf line 1}: currently ignored (file format version specification possible, header comment)

{\bf line 2}: must either contain \lcode{ASCII} or \lcode{BINARY}\\
currently ignored (possible form definition like ASCII / mixed ASCII/BINARY (similar to in vtk(?)))\\
At the moment, this file must be a formatted text file. 
%---------------------------------------------------------
\subsection{Model Parameters Block}
%---------------------------------------------------------

{\bf line}: \lcode{MODEL PARAMETERS}\\
this defines that the definition of the model parameters starts here.

{\bf line}: \lcode{INVERSION_GRID_CELLS value}\\
where \lcode{value} is either \lcode{ALL} (all inversion grid cells are taken) or \lcode{SPECIFIC} 
(specific definition of set of invgrid cells following below)

{\bf line}: \lcode{PARAMETERS value}\\
where \lcode{value} is either \lcode{ALL} (all inversion grid cells are taken) or \lcode{SPECIFIC} 
(specific definition of model parameters for each invgrid cell, following below. Only allowed if 
\lcode{INVERSION_GRID_CELLS SPECIFIC}). 

{\bf If \lcode{PARAMETERS ALL}, line}: \lcode{nparam pmtrization_1 param_1 ... pmtrization_n param_n}\\
defines the parametrization used for all inversion grid cells.

If \lcode{INVERSION_GRID_CELLS SPECIFIC}, the following line must contain the number of cells \lcode{ncell} 
which should be taken, followed by \lcode{ncell} blocks of lines, each defining an inversion grid cell. In case
of \lcode{PARAMETERS ALL}, these blocks consist of a single line line containing an inversion grid cell index.
In case of \lcode{PARAMETERS SPECIFIC}, these blocks consist of two lines: one line containing an inversion grid cell
index and an additional second line of the form\\
\lcode{nparam pmtrization_1 param_1 ... pmtrization_n param_n} \\
defining the parameters to be used for this specific inversion grid cell.

%---------------------------------------------------------
\subsection{Data Samples Block}
%---------------------------------------------------------
line ``DATA SAMPLES''
%
line of form: ``PATHS value'', where value is either ``ALL'' (all paths for a given set of event and station indices
are used), or ``SPECIFIC'' (a specific definition of paths as a series of event and station index pairs follows below)

If ``PATHS ALL'', the next two lines are of form ``nev iev\_1 ... iev\_n'' and ``nstat istat\_1 ... istat\_n'', defining the set
of event and station indices, which form (by all combinations) the used paths. 

line of form: ``COMPONENTS value'', where value is either ``ALL'' (for all paths, the same components are used) or ``SPECIFIC''
(only allowed if ``PATHS SPECIFIC'', for each path a specific set of components may be defined)

If ``COMPONENTS ALL'', the next line is of form ``ncomp comp\_1 ... comp\_n'' defining the component indices for all paths.

line of form: ``FREQUENCIES value'', where value is either ``ALL'' (for all paths, the same frequency indices are used) or ``SPECIFIC''
(only allowed if ``PATHS SPECIFIC'', for each path a specific set of frequency indices may be defined)

If ``FREQUENCIES ALL'', the next line is of form ``nfreq ifreq\_1 ... ifreq\_n'' defining the frequency indices for all paths.

line of form: ``IMRE value'', where value is either ``ALL'' (for all paths, the same set of imaginary/real parts are used) or ``SPECIFIC''
(only allowed if ``PATHS SPECIFIC'', for each path a specific set of imaginary/real parts may be defined)

If ``IMRE ALL'', the next line is of form ``nimre imre\_1 ... imre\_n'' defining imaginary (i.e. imre\_i = ``im'') or real parts 
(imre\_i = ``re'') for all paths.

If ``PATHS SPECIFIC'', the following line must contain the number ``npaths'' of paths which should be used, followed by 
npahts blocks of lines, each defining the path and the data samples for that path. \\
These blocks constist of at least one line containing the event /station index pair ``iev istat''. \\
For each keyword ``COMPONENTS'', ``FREQUENCIES'' and ``IMRE'' -- if ``SPECIFIC'' -- one line is added to such a block 
of lines, in the same form as the line following ``keyword ALL'' (see above), 
defining the specific components, frequencies or set of imaginary/real parts for each of the specific paths.
%
%++++++++++++++++++++++++++++++++++++++++++++++++++++++++++
\section{\lcodetitle{ecartInversionGrid} Files} \label{files,sec:ecart_invgrid}
%++++++++++++++++++++++++++++++++++++++++++++++++++++++++++
%
%---------------------------------------------------------
\subsection{Nodes Coordinates Files}
%---------------------------------------------------------
These files contain a collection of points in space, given in Cartesian \lcode{X}-, \lcode{Y}-, 
\lcode{Z}-coordinates. They must be text files and have the following format.

The first line contains a single integer value, indicating the number of lines to come (i.e.\ the 
number of points).\\
Each following line contains 3 floating point numbers (separated by white space) defining Cartesian 
\lcode{X}-, \lcode{Y}-, \lcode{Z}-coordinates of a point.

%---------------------------------------------------------
\subsection{Cell Connectivity Files}
%---------------------------------------------------------
These files contain the definition of cells, based on points as defined in the nodes coordinates files.
They must be text files and have the following format.

The first line contains a single integer value, indicating the number of lines to come (i.e.\ the 
number of cells).\\
Each following line contains \lcode{n} integer numbers (separated by white space, \lcode{n = 4} 
in case of tet4-type cells, \lcode{n = 8} in case of hex8-type cells), which define the control nodes 
of the cell and correspond to the point indices in the respective nodes coordinates file, whereby the 
lowest point index is 1, corresponding to the second line (first point) in the nodes coordinates file.

The order of the point indices in a line is assumed to correspond to the vtk cell conventions!
In case one of the cell connectivity files not existing, or their first line containing value 0, no cells
of the respective type will be created.

%---------------------------------------------------------
\subsection{Cell Neighbours File}
%---------------------------------------------------------
The terminology ``lines'' below refers to the case of this file not being binary, but a text file. 
In case of this file being binary, the file content is expected value by value as on the rows of the
text file. It will be opened by \lcode{FORTRAN} code with attribute \lcode{access='stream'}
(i.e.\ expecting the values as a simple byte stream) and expects integer values of \lcode.

The first line contains the total number of inversion grid cells \lcode{ncell}.\\
The next \lcode{ncell} lines (one for each cell in order of the cell index) are of the form:\\
\lcode{nnb icell_1 ... icell_nnb}\\
whereby \lcode{nnb} is the number of neighbours of the respective cell (must be \lcode{0} if 
no neighbours) followed by \lcode{nnb} cell indices \lcode{icell_1 ... icell_nnb}, defining 
the neighbour cells, if there are any neighbours.


%%% Local Variables:
%%% mode: latex
%%% TeX-master: "manual"
%%% End:

%
%-------------------------------
% APPENDIX CHAPTER programs and scripts
\chapter{Programs, Scripts and Modules} %\label{programs_scripts_modules}
% -*-LaTex-*-

%-----------------------------------------------------------------------------
%   Copyright 2015 Florian Schumacher (Ruhr-Universitaet Bochum, Germany)
%   and Phillip Gutt (Ruhr-Universitaet Bochum, Germany)
%
%   This file is part of the ASKI manual as a LaTeX document with main file
%   manual.tex
%
%   Permission is granted to copy, distribute and/or modify this document
%   under the terms of the GNU Free Documentation License, Version 1.3
%   or any later version published by the Free Software Foundation;
%   with no Invariant Sections, no Front-Cover Texts, and no Back-Cover Texts.
%   A copy of the license is included in the section entitled ``GNU
%   Free Documentation License''. 
%-----------------------------------------------------------------------------
%
This chapter collects scripts, binary programs or modular program components contained 
in the \ASKI package. It is not refered to any code, here, but give details on application
by the user or background knowledge.
%
%++++++++++++++++++++++++++++++++++++++++++++++++++++++++++
\section{binary programs} \label{programs_scripts,sec:bin_prog}
%++++++++++++++++++++++++++++++++++++++++++++++++++++++++++
%
The commands are written in the form:
\begin{itemize}
\item[]{\bf command} [-optional arguments]...[positional arguments]
\end{itemize}
Commands with several positional argumentes have to be written with all positional arguments in the right order (the order given in the subsection).
%
%- - - - - - - - - - - - - - - - - - - - - - - - - - - - -
%
\subsection{\lcode{combineInvertedModels}} \label{programs_scripts,sec:bin_prog,sec:combine_inverted_models}
%- - - - - - - - - - - - - - - - - - - - - - - - - - - - -
Compute linear combination   \lcode{coef1 * kim1 + coef2 * kim2}   of two models on inversion grid kim1, kim2.

This manual section is still to be written, sorry. I hope to provide it soon in a new manual version. %% FS FS CONTINUE
Calling the executable without arguments will print a very short help message on how to use it.
%
%- - - - - - - - - - - - - - - - - - - - - - - - - - - - -
%
\subsection{\lcode{computeCorrectionSyntheticData}} \label{programs_scripts,sec:bin_prog,sec:comp_correct_syn_data}
%- - - - - - - - - - - - - - - - - - - - - - - - - - - - -
Computes additional files in iteration-step-specific subdirectory \lcode{SYNTHETIC_DATA} which are named as
\lcode{corr_EVENTID_STATIONNAME_COMPONENT}. These files contain the quantities $c_i^0$ (as defined in our GJI paper
eq.\ (26)), which are corrections to synthetic data due to change from path to global reference model.
 %- - - - - - - - - - - - - - - - - - - - - - - - - - - - -
\subsubsection{positional arguments}
%- - - - - - - - - - - - - - - - - - - - - - - - - - - - -
\paragraph{\lcode{dmsi_file}}
Data-model-space-info file
%- - - - - - - - - - - - - - - - - - - - - - - - - - - - -
\paragraph{\lcode{main_parfile}}
Main parameter file of inversion.
%
%- - - - - - - - - - - - - - - - - - - - - - - - - - - - -
%
\subsection{\lcode{computeDataFromKernelSystem}} \label{programs_scripts,sec:bin_prog,sec:comp_data_kernel_sys}
%- - - - - - - - - - - - - - - - - - - - - - - - - - - - -
Computes the 'measured' data d in the equation  d-s(-sc) = K (m1 - mref)  for given model m1.

Can be used for (old-fashioned) linear computation of checkerboard data by forward multiplication of the sensitivity 
matrix with an artificial model vector containing checker anomalies.
%- - - - - - - - - - - - - - - - - - - - - - - - - - - - -
\subsubsection{positional arguments}
%- - - - - - - - - - - - - - - - - - - - - - - - - - - - -
\paragraph{\lcode{dmsi_file}}
Data-model-space-info file
%- - - - - - - - - - - - - - - - - - - - - - - - - - - - -          
\paragraph{\lcode{kim_file}}
kernel-inverted-model file which contains model m1.            
%- - - - - - - - - - - - - - - - - - - - - - - - - - - - -           
\paragraph{\lcode{outdir_data}}
Output directory where the new 'measured' data files are written to.
%- - - - - - - - - - - - - - - - - - - - - - - - - - - - - 
\paragraph{\lcode{main_parfile}}
Main parameter file of inversion. 
%
%- - - - - - - - - - - - - - - - - - - - - - - - - - - - -
%
\subsection{\lcode{computeFocussedMisfit}} \label{programs_scripts,sec:bin_prog,sec:comp_focus_misfit}
%- - - - - - - - - - - - - - - - - - - - - - - - - - - - -
compute focussed misfit of a dataset applying the focussing coefficients produced by program focusSpectralKernels

\subsubsection{positional arguments}
%- - - - - - - - - - - - - - - - - - - - - - - - - - - - -
\paragraph{\lcode{dmspace_file}}
Data model space input file which defines data and model space.
%- - - - - - - - - - - - - - - - - - - - - - - - - - - - -
\paragraph{\lcode{foc_coef_file}}
Output text file containing the focussing coefficients, as written by program \lcode{focusSpectralKernels}
%- - - - - - - - - - - - - - - - - - - - - - - - - - - - -
\paragraph{\lcode{main_parfile}}
Main parameter file of inversion.
%
%- - - - - - - - - - - - - - - - - - - - - - - - - - - - -
%
\subsection{\lcode{computeKernelCoverage}} \label{programs_scripts,sec:bin_prog,sec:compute_kernel_coverage}
%- - - - - - - - - - - - - - - - - - - - - - - - - - - - -
summate the absolute values of the column vectors of a given kernel matrix (for a given number of 
frequency windows)

This manual section is still to be written, sorry. I hope to provide it soon in a new manual version. %% FS FS CONTINUE
Calling the executable without arguments will print a very short help message on how to use it.
%
%- - - - - - - - - - - - - - - - - - - - - - - - - - - - -
%
\subsection{\lcode{computeKernels}} \label{programs_scripts,sec:bin_prog,sec:com_kernel}
%- - - - - - - - - - - - - - - - - - - - - - - - - - - - -
There are two possibele ways to define a set of spectral kernels that are computed:
\begin{itemize}
\item[(way 1):] compute kernel for only one path, defined by eventID and station name using options \lcode{-evid}, \lcode{-staname}, \lcode{-comp} and \lcode{-param}
\item[(way 2):] use flag \lcode{-dmspsce} in connection with optional range definition of the path index (flags \lcode{-ipath1} \lcode{-ipath2})\\
in order to define a subset of the paths contained in the given data-model-space description. If the upper (lower) limit of the path range is not defined (i.e. \lcode{-ipath1} (\lcode{-ipath2}) not set), the maximum (minimum) possible value is used. Then, all kernels for the defined range of paths in the given data-model-space description are computed.
\end{itemize}
%- - - - - - - - - - - - - - - - - - - - - - - - - - - - -
\subsubsection{positional arguments}
%- - - - - - - - - - - - - - - - - - - - - - - - - - - - -
\paragraph{\lcode{main_parfile}}
Main parameter file of inversion.
%- - - - - - - - - - - - - - - - - - - - - - - - - - - - -
\subsubsection{optional arguments}
%- - - - - - - - - - - - - - - - - - - - - - - - - - - - -
\paragraph{\lcode{-evid} \lcode{event_id}} 
Defines the event id of the one path (must belong to an event in main event list). (way 1)
%- - - - - - - - - - - - - - - - - - - - - - - - - - - - -
\paragraph{\lcode{-staname} \lcode{station_name}}
Defines the station name of the one path (must belong to a station in main station list). (way 1)
%- - - - - - - - - - - - - - - - - - - - - - - - - - - - -
\paragraph{\lcode{-comp} \lcode{"comp_1 ... comp_n"}}
Vector of receiver components for which the kernel should be computed. For valid components see \myref{basic_steps,sec:data_general}. (way 1)
%- - - - - - - - - - - - - - - - - - - - - - - - - - - - -
\paragraph{\lcode{-param} \lcode{"param_1 ... param_n"}}
Vector of parameter names for which the kernel should be computed. Only valid parameter names of the chosen model parametrization are accepted. (way 1)
%- - - - - - - - - - - - - - - - - - - - - - - - - - - - -
\paragraph{\lcode{-dmspace} \lcode{dmspace_file}}
Data model space input file to define a set of paths. (way 2)
%- - - - - - - - - - - - - - - - - - - - - - - - - - - - -
\paragraph{\lcode{-ipath1} \lcode{path1}}
\lcode{path1} is the first index of the path loop. By default, \lcode{path1 = 1} (way 2)
%- - - - - - - - - - - - - - - - - - - - - - - - - - - - -
\paragraph{\lcode{-ipath2} \lcode{path2}}
\lcode{path2} is the last index of the path loop. By default, \lcode{path2 = max_number_of_paths} as to data-model-space description. (way 2)
%- - - - - - - - - - - - - - - - - - - - - - - - - - - - -
\paragraph{\lcode{-wp}}
If set, then plain kernel values on WAVEFIELD POINTS are produced. Otherwise (if not set), pre-integrated kernels on inversion grid cells are computed
%
%- - - - - - - - - - - - - - - - - - - - - - - - - - - - -
%
\subsection{\lcode{computeMisfit}} \label{programs_scripts,sec:bin_prog,sec:comp_misfit}
%- - - - - - - - - - - - - - - - - - - - - - - - - - - - -
reads in measured and synthetic data characterized by data model space info file and computes the data misfit

\subsubsection{positional arguments}
%- - - - - - - - - - - - - - - - - - - - - - - - - - - - -
\paragraph{\lcode{dmspace_file}}
Data model space input file which defines data and model space.
%- - - - - - - - - - - - - - - - - - - - - - - - - - - - -
\paragraph{\lcode{main_parfile}}
Main parameter file of inversion.
%- - - - - - - - - - - - - - - - - - - - - - - - - - - - -
\subsubsection{optional arguments}
%- - - - - - - - - - - - - - - - - - - - - - - - - - - - -
\paragraph{\lcode{-jf "jf1..jfn"}}
Vector of \lcode{nf} frequency indices. Additionally to the misfit of the whole dataset defined by \lcode{dmspace_file}, the misfit is computed for data subsets restricted to these individual frequencies.
%
%- - - - - - - - - - - - - - - - - - - - - - - - - - - - -
%
\subsection{\lcode{createMeasuredData}} \label{programs_scripts,sec:bin_prog,sec:create_measured_data}
%- - - - - - - - - - - - - - - - - - - - - - - - - - - - -
Fourier transform of time-domain data to ASKI-conform frequency-domain measured data; frequency discretization of measured data as defined in main parfile

This manual section is still to be written, sorry. I hope to provide it soon in a new manual version. %% FS FS CONTINUE
Calling the executable without arguments will print a very short help message on how to use it.

(see \lcode{template/createMeasuredData_parfile_txt_template} for the required parfile for \lcode{-txt} input data)
 %% FS FS CONTINUE: ADD LIST OF REQUIRED KEYWORDS IN -txt PARFILE (AND THEIR MEANING) AND CONFER TO TEMPLATE FILE IN template/
%- - - - - - - - - - - - - - - - - - - - - - - - - - - - -
%
\subsection{\lcode{createStartmodelKim}} \label{programs_scripts,sec:bin_prog,sec:create_startmod_kim}
%- - - - - - - - - - - - - - - - - - - - - - - - - - - - -
 Create a file of type \lcode{kernel_inverted_model} (\lcode{.kim}) containing pre-defined values on the inversion grid. Can be used to create a start model for the full waveform inversion process.

\subsubsection{Mandatory options}
%- - - - - - - - - - - - - - - - - - - - - - - - - - - - -
\paragraph{\lcode{-igtype invgrid_type}}
\lcode{invgrid_type} is \lcode{TYPE_INVERSION_GRID} as in ASKI iteration step parameter file.
%- - - - - - - - - - - - - - - - - - - - - - - - - - - - -
\paragraph{\lcode{-igpar invgrid_parfile}}
\lcode{invgrid_parfile} is \lcode{PARFILE_INVERSION_GRID} as in ASKI iteration step parameter file.
%- - - - - - - - - - - - - - - - - - - - - - - - - - - - -
\paragraph{\lcode{-igpath invgrid_path}}
\lcode{invgrid_path} is treated as current iteration step path, used for inversion grid to write/read own files.
%- - - - - - - - - - - - - - - - - - - - - - - - - - - - -
\paragraph{\lcode{-mpmtrz model_pmtr}}
\lcode{model_pmtrz} is the model parametrization of the model which is to be created (must be consistent with content of \lcode{model_file}, see \lcode{-mfile}).
%- - - - - - - - - - - - - - - - - - - - - - - - - - - - -
\paragraph{\lcode{-mtype model_type}}
\lcode{model_type} is a string defining the type of \lcode{model_file} and the interpolation type:\\
'\lcode{1D_linear}':\\
linear interpolation of values between coordinates given in 1D model file.\\
'\lcode{3D _structured}':\\
trilinear interpolation of values between coordinates given in 3D structured model file.\\
\\
1D model file must have the following format
\begin{itemize}
\item[]line 1:   \lcode{nval}  \lcode{ncol}
\begin{itemize}
\item[]\lcode{nval} = number of model values /interpolation coordinates to come
\item[]\lcode{ncol} = number of columns to read from file, \lcode{ncol=1+N}, where \lcode{N} is number of parameters
\end{itemize}
\item[]line 2:   \lcode{icoord}  \lcode{param1} ... \lcode{paramN}
\begin{itemize}
\item[]\lcode{icoord} = index of inversion grid / wavefield point coordinate for which the interpolation should be applied (either 1,2 or 3)
\item[]\lcode{param1} ... \lcode{paramN} = N parameter names associated with the values in the columns below
\end{itemize}
\item[]lines 3 ...\lcode{nval}+2:   \lcode{coord val1} ... \lcode{valN}
\begin{itemize}
\item[]these following \lcode{nval} lines contain the interpolation coordinate and N model values for the respective parameters, as defined by line 2. The values '\lcode{coord}' are assumed to be be strictly monotonical(!) and can be either increasing or decreasing. You may choose this monotonicity at your will dependent on the coordinate for which the interpolation should be done (which, dependent on your inversion grid type may be depth or positive z-value...)
\end{itemize}
\end{itemize}
%- - - - - - - - - - - - - - - - - - - - - - - - - - - - -
3D model file must have the following format:
\begin{itemize}
\item[]line 1:   \lcode{nx  ny  nz}
\begin{itemize}
\item[]\lcode{nx,ny} and \lcode{nz} are the number of model points in X-, Y- and Z-direction
\end{itemize}
\item[]line 2:   \lcode{minX  minY  minZ}
\begin{itemize}
\item[]\lcode{minX, minY} and \lcode{minZ} are the smallest coordinates in the model
\end{itemize}
\item[]line 3:   \lcode{maxX  maxY  maxZ}
\begin{itemize}
\item[]\lcode{maxX, maxY} and \lcode{maxZ} are the highest coordinates in the mode
\item[]\lcode{nval}, the total number of vaules in the model is given by \lcode{nval = nx*ny*nz}
\end{itemize}
\item[]line 4:   \lcode{param1} ... \lcode{paramN}
\begin{itemize}
\item[]\lcode{param1} ... \lcode{paramN}, N parameter names associated with the values in the columns below
\end{itemize} 
\item[]line 5+:  \lcode{nval} model values
\item[]The model values have to be sorted like:\\
\\
DO i=minX,maxX+1
\begin{itemize}
\item[]DO j=minY,maxY+1
\begin{itemize}
\item[]DO k=minZ,maxZ+1
\begin{itemize}
\item[]READ model value
\end{itemize}
END DO
\end{itemize}
END DO
\end{itemize}
END DO
\end{itemize}
%- - - - - - - - - - - - - - - - - - - - - - - - - - - - -
\paragraph{\lcode{-mfile model_file}}
\lcode{model_file} is the model input file of type defined by \lcode{model_type}.
%- - - - - - - - - - - - - - - - - - - - - - - - - - - - -
\subsubsection{optional arguments}
%- - - - - - - - - - - - - - - - - - - - - - - - - - - - -
\paragraph{\lcode{-o outbase}}
\lcode{outbase} is output base name (default is \lcode{start_model}).
%- - - - - - - - - - - - - - - - - - - - - - - - - - - - -
\paragraph{\lcode{-bin}}
If set, the output vtk files will be binary, otherwise they will be ascii.
%
%- - - - - - - - - - - - - - - - - - - - - - - - - - - - -
%
\subsection{\lcode{exportKim}} \label{programs_scripts,sec:bin_prog,sec:exp_Kim}
%- - - - - - - - - - - - - - - - - - - - - - - - - - - - -
Program exportKim produces a text file (option \lcode{-otxt}) containing information of cell centers and radii (i.e. rough expansion) of all inversion grid cells and all cell neighbours, as well as model values and respective invgrid cell indices for each model parameter, as contained in a given kernelInvertedModel file. This text file may be used by any forward method to define the simulation model for the next iteration step, or by any tool handling final models.\\
The format of the produced text file is as follows:\\
The first 3 lines contain:
\begin{itemize}
\item[]\lcode{model_parametrization}
\item[]\lcode{number_of_parameters}  \lcode{name_param_1} ... \lcode{name_param_n}
\item[]\lcode{number_of_invgrid_cells}
\end{itemize}
The next \lcode{number_of_invgrid_cells} lines contain for each inversion grid cell:
\begin{itemize}
\item[]\lcode{c1 c2 c3 r nnb inb1} ... \lcode{inbn}
\end{itemize} 
where \lcode{c1,c2,c3} are the 3 coordinates of the cell center (in wavefield point coords), \lcode{r} is the cell radius (i.e. rough expansion of cell), \lcode{nnb} is the number of cell neighbours and \lcode{inb1},...,\lcode{inbn} are their nnb cell indices (if \lcode{nnb}\(>\)\lcode{0}, otherwise line ends on \lcode{nnb=0}). Then \lcode{number_of_parameters} blocks are following in the file, each having the following format:
\begin{itemize}
\item[]\lcode{name_param}
\item[]\lcode{nval}
\item[]\lcode{cell_indx}
\item[]\lcode{model_values}
\end{itemize}
where \lcode{name_param} is the name of the parameter to which the following model values belong, \lcode{nval} is the number of model values following, \lcode{cell_indx} is a vector of \lcode{nval} cell indices to which the model values belong (space separated on one line) and \lcode{model_values} is an vector of the actual \lcode{nval} model values (space separated on one line).\\
\\
Additionally (or alternatively) the program converts the .kim file to vtk files (option \lcode{-ovtk}).\\
The two options \lcode{-otxt} , \lcode{-ovtk} can be used independently of each other.
%- - - - - - - - - - - - - - - - - - - - - - - - - - - - -
\subsubsection{positional arguments}
%- - - - - - - - - - - - - - - - - - - - - - - - - - - - -
\paragraph{\lcode{main_parfile}}
Main parameter file of inversion.
%- - - - - - - - - - - - - - - - - - - - - - - - - - - - -
\subsubsection{Mandatory options}
%- - - - - - - - - - - - - - - - - - - - - - - - - - - - -
\paragraph{\lcode{-kim kernel_inverted_mode_file}}
\lcode{kernel_inverted_mode_file} is the binary file containing the inverted model, which is to be exported.
%- - - - - - - - - - - - - - - - - - - - - - - - - - - - -
\paragraph{\lcode{-otxt outfile_txt}}
\lcode{outfile_txt} is the text output file to which inversion grid and model information will be written.
%- - - - - - - - - - - - - - - - - - - - - - - - - - - - -
\paragraph{\lcode{-ovtk outfile_vtk}}
\lcode{outfile_vtk} is the output file base to which standard vtk files of the .kim file will be written.
%
%- - - - - - - - - - - - - - - - - - - - - - - - - - - - -
%
\subsection{\lcode{focusSpectralKernels}} \label{programs_scripts,sec:bin_prog,sec:focus_spec_kernel}
%- - - - - - - - - - - - - - - - - - - - - - - - - - - - -
compute Backus-Gilbert focussing of sensitivity kernels on a defined focussing region in the model space

This manual section is still to be written, sorry. I hope to provide it soon in a new manual version. %% FS FS CONTINUE
Calling the executable without arguments will print a very short help message on how to use it.

%% \subsubsection{positional arguments}
%% %- - - - - - - - - - - - - - - - - - - - - - - - - - - - -
%% \paragraph{\lcode{dmspace_file}}
%% Data model space input file which defines data and model space.
%% %- - - - - - - - - - - - - - - - - - - - - - - - - - - - -
%% \paragraph{\lcode{outfile_base}}
%% Base name of output files relative to output directory (will be used for all files, with suitable extensions).
%% %- - - - - - - - - - - - - - - - - - - - - - - - - - - - -
%% \paragraph{\lcode{main_parfile}}
%% Main parameter file of inversion.
%% %- - - - - - - - - - - - - - - - - - - - - - - - - - - - -
%% \subsubsection{optional arguments}
%% %- - - - - - - - - - - - - - - - - - - - - - - - - - - - -
%% \paragraph{\lcode{-param "param_1 .. param_n"}}
%% Vector of \lcode{nparam} parameter names which will be focussed on
%
%- - - - - - - - - - - - - - - - - - - - - - - - - - - - -
%
\subsection{\lcode{initBasics}} \label{programs_scripts,sec:bin_prog,sec:in_basics}
%- - - - - - - - - - - - - - - - - - - - - - - - - - - - -
Initiating and testing all basic requirements for ASKI programs (parameter files, event and station list, inversion grid, wavefield points, integration weights, reference model)

\subsubsection{positional arguments}
%- - - - - - - - - - - - - - - - - - - - - - - - - - - - -
\paragraph{\lcode{main_parfile}}
Main parameter file of inversion.
%- - - - - - - - - - - - - - - - - - - - - - - - - - - - -
\subsubsection{optional arguments}
%- - - - - - - - - - - - - - - - - - - - - - - - - - - - -
\paragraph{\lcode{-recr}}
If set, existing files will be recreated with current parfile specifications (overwrites existing files will be read in and not newly created). Affected files are:
\begin{itemize}
\item[-]\lcode{inversion_grid} and \lcode{integration_weigts} and all related \lcode{.vtk} files (including \lcode{wavefield_points.vtk} files)
\item[-]\lcode{stations.vtk}, \lcode{events.vtk}
\item[-]kernel reference model files (on wavefield points, and as interpolation on inversion grid)
\end{itemize}
IF NOT SET (default), EXISTING FILES (especially \lcode{inversion_grid}, \lcode{integration_weights}) WILL BE READ IN ONLY, REGARDLESS OF ANY CHANGES IN PARFILES! So, if you change the specification of inversion grid, or integration weights (or stations, events which is only relevant for \lcode{station.vtk} \lcode{events.vtk}), after having already created the respective files, you should set \lcode{-recr}.
%
%- - - - - - - - - - - - - - - - - - - - - - - - - - - - -
%
\subsection{\lcode{investigateDataResiduals}} \label{programs_scripts,sec:bin_prog,sec:invest_data_residuals}
%- - - - - - - - - - - - - - - - - - - - - - - - - - - - -
Gets some statistics about the data residuals of a given data (sub)set.
%- - - - - - - - - - - - - - - - - - - - - - - - - - - - -
\subsubsection{positional arguments}
%- - - - - - - - - - - - - - - - - - - - - - - - - - - - -
\paragraph{\lcode{dmsi_file}}
Data-model-space-info file
%- - - - - - - - - - - - - - - - - - - - - - - - - - - - -               
\paragraph{\lcode{outdir_stats_files}}
Output directory where residual files per path and component will written.
%- - - - - - - - - - - - - - - - - - - - - - - - - - - - -
\paragraph{\lcode{main_parfile}}
Main parameter file of inversion.
%- - - - - - - - - - - - - - - - - - - - - - - - - - - - -
\subsubsection{optional arguments}
%- - - - - - - - - - - - - - - - - - - - - - - - - - - - -
\paragraph{\lcode{-ovtk outfile_vtk}}
\lcode{outfile_vtk} of output vtk files. If not set, no vtk files will be produced. (Default = dataset)
%
%- - - - - - - - - - - - - - - - - - - - - - - - - - - - -
%
\subsection{\lcode{invgrid2vtk}} \label{programs_scripts,sec:bin_prog,sec:invgrid_vtk}
%- - - - - - - - - - - - - - - - - - - - - - - - - - - - -
Create vtk file(s) of the given inversion grid (useful to look at, to see if the specifications are correct)

\subsubsection{Mandatory options}
%- - - - - - - - - - - - - - - - - - - - - - - - - - - - -
\paragraph{\lcode{-igtype invgrid_type}}
\lcode{invgrid_type} is \lcode{TYPE_INVERSION_GRID} as in ASKI iteration step parameter file.
%- - - - - - - - - - - - - - - - - - - - - - - - - - - - -
\paragraph{\lcode{-igpar invgrid_parfile}}
\lcode{invgrid_parfile} is \lcode{PARFILE_INVERSION_GRID} as in ASKI iteration step parameter file.
%- - - - - - - - - - - - - - - - - - - - - - - - - - - - -
\paragraph{\lcode{-igpath invgrid_path}}
\lcode{invgrid_path} is treated as current iteration step path, used for inversion grid to write/read own files.
%- - - - - - - - - - - - - - - - - - - - - - - - - - - - -
\subsubsection{Optional options}
%- - - - - - - - - - - - - - - - - - - - - - - - - - - - -
\paragraph{\lcode{-o outbase}}
\lcode{outbase} is output base name (default is \lcode{inversion_grid})
%- - - - - - - - - - - - - - - - - - - - - - - - - - - - -
\paragraph{\lcode{-overwr}}
If set, existing output files will be overwritten.
%- - - - - - - - - - - - - - - - - - - - - - - - - - - - -
\paragraph{\lcode{-nb "idx_1 ... idx_n}}
Vecotr of \lcode{n} cell indices, indicating a set of cells the neighbours of which will be written as 
vtk files. IN THE FUTURE: may be helpful to also accept ranges like "20:40".
%- - - - - - - - - - - - - - - - - - - - - - - - - - - - -
\paragraph{\lcode{-all_nb}}
Indicating to use all cell indices to write neighbours for. Must not be set simultaneously along with \lcode{-nb}.
%- - - - - - - - - - - - - - - - - - - - - - - - - - - - -
\paragraph{\lcode{-bin}}
If set, the output vtk files will be binary, otherwise they will be ascii.
%- - - - - - - - - - - - - - - - - - - - - - - - - - - - -
\paragraph{\lcode{-recr}}
If set, the existing inversion grid file(s) (if any existing, and if type creates any) will be recreated with current invgrid parfile specifications. If not set (default), existing inversion grid will be read in only, REGARDLESS OF ANY CHANGES IN THE PARFILE!
%
%- - - - - - - - - - - - - - - - - - - - - - - - - - - - -
%
\subsection{\lcode{kdispl2vtk}} \label{programs_scripts,sec:bin_prog,sec:kdispl_2_vtk}
%- - - - - - - - - - - - - - - - - - - - - - - - - - - - -
Extract kernel displacement spectra to vtk files for certain wavefield and strain components and frequencies

This manual section is still to be written, sorry. I hope to provide it soon in a new manual version. %% FS FS CONTINUE
Calling the executable without arguments will print a very short help message on how to use it.
%
%- - - - - - - - - - - - - - - - - - - - - - - - - - - - -
%
\subsection{\lcode{kernel2vtk}} \label{programs_scripts,sec:bin_prog,sec:kernel_2_vtk}
%- - - - - - - - - - - - - - - - - - - - - - - - - - - - -
Program kernel2vtk writes the pre-integrated spectral sensitivity kernels as vtk files for specific paths, parameters, components and frequencies.
%- - - - - - - - - - - - - - - - - - - - - - - - - - - - -
\subsubsection{positional arguments}
%- - - - - - - - - - - - - - - - - - - - - - - - - - - - -
\paragraph{\lcode{main_parfile}}
Main parameter file of inversion.
%- - - - - - - - - - - - - - - - - - - - - - - - - - - - -
\subsubsection{optional arguments}
%- - - - - - - - - - - - - - - - - - - - - - - - - - - - -
\paragraph{\lcode{-evid} \lcode{event_id}}
Defines the event id of the one path. (must belong to an event in main event list)
%- - - - - - - - - - - - - - - - - - - - - - - - - - - - -
\paragraph{\lcode{-staname} \lcode{station_name}}
Defines the station name of the one path. (must belong to a station in main station list)
%- - - - - - - - - - - - - - - - - - - - - - - - - - - - -
\paragraph{\lcode{-comp "comp_1 ... comp_n"}}
Vector of receiver components for which vtk files should be generated. For valid components see \myref{basic_steps,sec:data_general}.
%- - - - - - - - - - - - - - - - - - - - - - - - - - - - -
\paragraph{\lcode{-param "param_1 ... param_n"}}
Vector of parameter names for which vtk files should be generated. Only valid parameter names of the chosen model parametrization are accepted. 
%- - - - - - - - - - - - - - - - - - - - - - - - - - - - -
\paragraph{\lcode{-ifreq "idx_1 ... idx_n"}}
Vector of \lcode{n} frequency indices for which vtk files should be generated. 
%- - - - - - - - - - - - - - - - - - - - - - - - - - - - -
\paragraph{\lcode{-all_ifreq}}
If set, for all frequency indices of current iteration vtk files will be generated. Must not be set simultaneously along with \lcode{-ifreq}.
%- - - - - - - - - - - - - - - - - - - - - - - - - - - - -
\paragraph{\lcode{-wp}}
If set, then the original kernels on the wavefield points are produced (recalculated!) INSTEAD of the pre-integrated ones on the inversion grid.
%
%- - - - - - - - - - - - - - - - - - - - - - - - - - - - -
%
\subsection{\lcode{kgt2vtk}} \label{programs_scripts,sec:bin_prog,sec:kgt_2_vtk}
%- - - - - - - - - - - - - - - - - - - - - - - - - - - - -
Extract kernel green tensor spectra to vtk files for certain wavefield and strain components and frequencies

This manual section is still to be written, sorry. I hope to provide it soon in a new manual version. %% FS FS CONTINUE
Calling the executable without arguments will print a very short help message on how to use it.
%
%- - - - - - - - - - - - - - - - - - - - - - - - - - - - -
%
\subsection{\lcode{krm2kim}} \label{programs_scripts,sec:bin_prog,sec:krm_kim}
interpolate kernel reference model onto inversion grid and produce a .kim file of it

%- - - - - - - - - - - - - - - - - - - - - - - - - - - - -
\subsubsection{positional arguments}
%- - - - - - - - - - - - - - - - - - - - - - - - - - - - -
\paragraph{\lcode{outfile_base}}
Basename of output model files - will additionally be written as vtk.
%- - - - - - - - - - - - - - - - - - - - - - - - - - - - -
\paragraph{\lcode{main_parfile}}
Main parameter file of inversion.
%- - - - - - - - - - - - - - - - - - - - - - - - - - - - -
\subsubsection{optional arguments}
%- - - - - - - - - - - - - - - - - - - - - - - - - - - - -
\paragraph{\lcode{-krm kernel_reference_mode_file}}
If set, then instead of using the kernel reference model file defined by the iteration step parfile, the given file \lcode{kernel_reference_mode_file} is used to read in the kernel reference model.
%
%- - - - - - - - - - - - - - - - - - - - - - - - - - - - -
%
\subsection{\lcode{paths2vtk}} \label{programs_scripts,sec:bin_prog,sec:path_to_vtk}  
%- - - - - - - - - - - - - - - - - - - - - - - - - - - - -
Plots all paths contained in the data space definition as vtk lines. 
%- - - - - - - - - - - - - - - - - - - - - - - - - - - - -
\subsubsection{positional arguments}
%- - - - - - - - - - - - - - - - - - - - - - - - - - - - -
\paragraph{\lcode{dmsi_file}}
Data-model-space-info file   
%- - - - - - - - - - - - - - - - - - - - - - - - - - - - -
\paragraph{\lcode{outfile}}
Output file (basename) of the vtk file(s).
%- - - - - - - - - - - - - - - - - - - - - - - - - - - - -
\paragraph{\lcode{main_parfile}}
Main parameter file of inversion.  
%
%- - - - - - - - - - - - - - - - - - - - - - - - - - - - -
%

\subsection{\lcode{solveCglsKernelSystem}} \label{programs_scripts,sec:bin_prog,sec:solve_cgls_kernel_sys}
%- - - - - - - - - - - - - - - - - - - - - - - - - - - - -
Solves linear system of sensitivity kernel equations in parallel by a conjugate-gradient method.
The algorithm applied here is the "CGLS1" from paper \cite{bjorck1998stability}.

By default, the conjugate-gradient algorithm uses single precision. In case the program crashes due to what
looks like precision problems (dividing by 0, getting values ``NaN'' etc.), you may try to switch to 
double precision (change the respective line in the very beginning of code file \lcode{f90/solveCglsKernelSystem.f90}
and recompile \lcode{solveCglsKernelSystem}).

This executable is parallelized using \lcode{MPI}. Hence, it should be called like\\
\lcode{mpirun -np 8 solveCglsKernelSystem} (with respective arguments)\\
The executable uses the maximum number of parallel slots being available in the \lcode{MPI} environment, in 
the above example this would be 8.
%- - - - - - - - - - - - - - - - - - - - - - - - - - - - -
\subsubsection{positional arguments}
%- - - - - - - - - - - - - - - - - - - - - - - - - - - - -
\paragraph{\lcode{dmsi_file}}
Data-model-space-info file.
%- - - - - - - - - - - - - - - - - - - - - - - - - - - - -
\paragraph{\lcode{outfile_base}}
Base name of output files (will be used for all files, with suitable extensions)
%- - - - - - - - - - - - - - - - - - - - - - - - - - - - -
\paragraph{\lcode{CG_parfile}}
Parameter file defining details related to the conjugate gradient algorithm used to solve the linear system
(a template is provided: \lcode{template/solveCglsKernelSystem_parfile_template}).
%% FS FS CONTINUE: ADD LIST OF REQUIRED KEYWORDS IN PARFILE (AND THEIR MEANING) AND CONFER TO TEMPLATE FILE IN template/
%- - - - - - - - - - - - - - - - - - - - - - - - - - - - -
\paragraph{\lcode{main_parfile}}
Main parameter file of inversion.
 %- - - - - - - - - - - - - - - - - - - - - - - - - - - - -
\subsubsection{optional arguments}
%- - - - - - - - - - - - - - - - - - - - - - - - - - - - -
\paragraph{\lcode{-regscal type_regul_scaling}}
\lcode{type_regul_scaling} is the type of scaling of regularization constraints, at the moment only 
\lcode{absmax_per_param,overall_factor}, \lcode{absmax_per_param,param_factors} and \lcode{none} are supported
%- - - - - - - - - - - - - - - - - - - - - - - - - - - - -
\paragraph{\lcode{-smoothing scaling_values}}
If set, smoothing conditions are applied. \lcode{scaling_values} is a vector of scaling values consistent 
with \lcode{-regscal}:\\
\lcode{absmax_per_param,overall_factor}: one single factor\\
\lcode{absmax_per_param,param_factors}: one factor per parameter name of current parametrization (in conventional order)\\
\lcode{none} : values given here are ignored
%- - - - - - - - - - - - - - - - - - - - - - - - - - - - -
\paragraph{\lcode{-smoothbnd type_smoothing_boundary}}
\lcode{type_smoothing_boundary} defines the way (non-existing) neighbours are treated in smoothing conditions at outer/inner 
boundaries of the inversion grid. Supported types:\\
\lcode{zero_all_outer_bnd}: apply zero smoothing conditions at \emph{all} outer boundaries.\\
\lcode{zero_burried_outer_bnd,cont_free_surface}: apply zero smoothing conditions at all outer boundaries \emph{except} on free surfaces. \\
If not set, standard average is used everywhere (equivalent to continuity boundary conditions).
%- - - - - - - - - - - - - - - - - - - - - - - - - - - - -
\paragraph{\lcode{-damping scaling_values}}
If set, damping conditions are applied. \lcode{scaling_values} is a vector of scaling values consistent 
with \lcode{-regscal}:\\
\lcode{absmax_per_param,overall_factor}: one single factor\\
\lcode{absmax_per_param,param_factors}: one factor per parameter name of current parametrization (in conventional order)\\
\lcode{none} : values given here are ignored
%- - - - - - - - - - - - - - - - - - - - - - - - - - - - -
\paragraph{\lcode{-odir}}
If set, \lcode{outfile_base} will be assumed relatively to iteration step output files directory.
%- - - - - - - - - - - - - - - - - - - - - - - - - - - - -
\paragraph{\lcode{-startsol file_name_starting_solution}}
The \lcode{.kim} file \lcode{file_name_starting_solution} defines the starting solution of linear system.
%- - - - - - - - - - - - - - - - - - - - - - - - - - - - -
\paragraph{\lcode{-normalize type_data_normalization}}
Optionally, \lcode{type_data_normalization} defines the type of data normalization. Supported types:\\
\lcode{maxamp_mdata_by_paths}: (description follows)\\ %% FS FS CONTINUE
\lcode{maxamp_mdata_by_paths_and_frequency}: (description follows)\\ %% FS FS CONTINUE
\lcode{scale_maxamp_mdata_by_paths}: (description follows) %% FS FS CONTINUE
%
%- - - - - - - - - - - - - - - - - - - - - - - - - - - - -
%
\subsection{\lcode{solveKernelSystem}} \label{programs_scripts,sec:bin_prog,sec:sol_Ker_Sys}
%- - - - - - - - - - - - - - - - - - - - - - - - - - - - -
Do inversion step by solving the kernel linear system defined by data model space info and regularization constraints. Serial programm using LAPACK library.

\subsubsection{positional arguments}
%- - - - - - - - - - - - - - - - - - - - - - - - - - - - -
\paragraph{\lcode{dmspace_file}}
Data model space input file which defines data and model space.
%- - - - - - - - - - - - - - - - - - - - - - - - - - - - -
\paragraph{\lcode{outfile}}
Basename of output model files -- output model will additionally be written as vtk.
%- - - - - - - - - - - - - - - - - - - - - - - - - - - - -
\paragraph{\lcode{main_parfile}}
Main parameter file of inversion.
%- - - - - - - - - - - - - - - - - - - - - - - - - - - - -
\subsubsection{optional arguments}
%- - - - - - - - - - - - - - - - - - - - - - - - - - - - -
\paragraph{\lcode{-regscal type_regul_scaling}}
\lcode{type_regul_scaling} is the type of scaling of regularization constraints, at the moment only 
\lcode{absmax_per_param,overall_factor}, \lcode{absmax_per_param,param_factors} and \lcode{none} are supported
%- - - - - - - - - - - - - - - - - - - - - - - - - - - - -
\paragraph{\lcode{-smoothing scaling_values}}
If set, smoothing conditions are applied. \lcode{scaling_values} is a vector of scaling values consistent 
with \lcode{-regscal}:\\
\lcode{absmax_per_param,overall_factor}: one single factor\\
\lcode{absmax_per_param,param_factors}: one factor per parameter name of current parametrization (in conventional order)\\
\lcode{none} : values given here are ignored
%- - - - - - - - - - - - - - - - - - - - - - - - - - - - -
\paragraph{\lcode{-smoothbnd type_smoothing_boundary}}
\lcode{type_smoothing_boundary} defines the way (non-existing) neighbours are treated in smoothing conditions at outer/inner 
boundaries of the inversion grid. Supported types:\\
\lcode{zero_all_outer_bnd}: apply zero smoothing conditions at \emph{all} outer boundaries.\\
\lcode{zero_burried_outer_bnd,cont_free_surface}: apply zero smoothing conditions at all outer boundaries \emph{except} on free surfaces. \\
If not set, standard average is used everywhere (equivalent to continuity boundary conditions).
%- - - - - - - - - - - - - - - - - - - - - - - - - - - - -
\paragraph{\lcode{-damping scaling_values}}
If set, damping conditions are applied. \lcode{scaling_values} is a vector of scaling values consistent 
with \lcode{-regscal}:\\
\lcode{absmax_per_param,overall_factor}: one single factor\\
\lcode{absmax_per_param,param_factors}: one factor per parameter name of current parametrization (in conventional order)\\
\lcode{none} : values given here are ignored
%- - - - - - - - - - - - - - - - - - - - - - - - - - - - -
\paragraph{\lcode{-odir}}
If set, \lcode{outfile_base} will be assumed relatively to iteration step output files directory.
%- - - - - - - - - - - - - - - - - - - - - - - - - - - - -
\paragraph{\lcode{-normalize type_data_normalization}}
Optionally, \lcode{type_data_normalization} defines the type of data normalization. Supported types:\\
\lcode{maxamp_mdata_by_paths}: (description follows)\\ %% FS FS CONTINUE
\lcode{maxamp_mdata_by_paths_and_frequency}: (description follows)\\ %% FS FS CONTINUE
\lcode{scale_maxamp_mdata_by_paths}: (description follows) %% FS FS CONTINUE
%
%- - - - - - - - - - - - - - - - - - - - - - - - - - - - -
%
\subsection{\lcode{solveParKernelSystem}} \label{programs_scripts,sec:bin_prog,sec:solve_par_kernel_sys}
%- - - - - - - - - - - - - - - - - - - - - - - - - - - - -
Solves linear system of sensitivity kernel equations in parallel by ScaLAPACK libraries.

This executable is parallelized using \lcode{BLACS} libraries which usually are based on \lcode{MPI}. 
Hence, it should probably be called like\\
\lcode{mpirun -np 8 solveParKernelSystem} (with respective arguments)\\
The executable uses the maximum number of parallel slots being available in the \lcode{MPI} environment, in 
the above example this would be 8. \lcode{solveParKernelSystem} minimally requires \lcode{NPROC_ROWS*NPROC_COLUMNS} parallel
slots, as defined by the \lcode{mpi_parfile}. \emph{If there are less slots availble than required, the program will 
raise an error}.
%- - - - - - - - - - - - - - - - - - - - - - - - - - - - -
\subsubsection{positional arguments}
%- - - - - - - - - - - - - - - - - - - - - - - - - - - - -
\paragraph{\lcode{dmsi_file}}
Data-model-space-info file
%- - - - - - - - - - - - - - - - - - - - - - - - - - - - -
\paragraph{\lcode{outfile_base}}
Base name of output files (will be used for all files, with suitable extensions)
%- - - - - - - - - - - - - - - - - - - - - - - - - - - - -
\paragraph{\lcode{mpi_parfile}}
Parameter file defining everything related to the ScaLAPACK parallelization of the linear system
(a template is provided: \lcode{template/solveParKernelSystem_parfile_template}).
%% FS FS CONTINUE: ADD LIST OF REQUIRED KEYWORDS IN PARFILE (AND THEIR MEANING) AND CONFER TO TEMPLATE FILE IN template/
%- - - - - - - - - - - - - - - - - - - - - - - - - - - - -
\paragraph{\lcode{main_parfile}}
Main parameter file of inversion.
 %- - - - - - - - - - - - - - - - - - - - - - - - - - - - -
\subsubsection{optional arguments}
%- - - - - - - - - - - - - - - - - - - - - - - - - - - - -
\paragraph{\lcode{-regscal type_regul_scaling}}
\lcode{type_regul_scaling} is the type of scaling of regularization constraints, at the moment only 
\lcode{absmax_per_param,overall_factor}, \lcode{absmax_per_param,param_factors} and \lcode{none} are supported
%- - - - - - - - - - - - - - - - - - - - - - - - - - - - -
\paragraph{\lcode{-smoothing scaling_values}}
If set, smoothing conditions are applied. \lcode{scaling_values} is a vector of scaling values consistent 
with \lcode{-regscal}:\\
\lcode{absmax_per_param,overall_factor}: one single factor\\
\lcode{absmax_per_param,param_factors}: one factor per parameter name of current parametrization (in conventional order)\\
\lcode{none} : values given here are ignored
%- - - - - - - - - - - - - - - - - - - - - - - - - - - - -
\paragraph{\lcode{-smoothbnd type_smoothing_boundary}}
\lcode{type_smoothing_boundary} defines the way (non-existing) neighbours are treated in smoothing conditions at outer/inner 
boundaries of the inversion grid. Supported types:\\
\lcode{zero_all_outer_bnd}: apply zero smoothing conditions at \emph{all} outer boundaries.\\
\lcode{zero_burried_outer_bnd,cont_free_surface}: apply zero smoothing conditions at all outer boundaries \emph{except} on free surfaces. \\
If not set, standard average is used everywhere (equivalent to continuity boundary conditions).
%- - - - - - - - - - - - - - - - - - - - - - - - - - - - -
\paragraph{\lcode{-damping scaling_values}}
If set, damping conditions are applied. \lcode{scaling_values} is a vector of scaling values consistent 
with \lcode{-regscal}:\\
\lcode{absmax_per_param,overall_factor}: one single factor\\
\lcode{absmax_per_param,param_factors}: one factor per parameter name of current parametrization (in conventional order)\\
\lcode{none} : values given here are ignored
%- - - - - - - - - - - - - - - - - - - - - - - - - - - - -
\paragraph{\lcode{-odir}}
If set, \lcode{outfile_base} will be assumed relatively to iteration step output files directory.
%- - - - - - - - - - - - - - - - - - - - - - - - - - - - -
\paragraph{\lcode{-normalize type_data_normalization}}
Optionally, \lcode{type_data_normalization} defines the type of data normalization. Supported types:\\
\lcode{maxamp_mdata_by_paths}: (description follows)\\ %% FS FS CONTINUE
\lcode{maxamp_mdata_by_paths_and_frequency}: (description follows)\\ %% FS FS CONTINUE
\lcode{scale_maxamp_mdata_by_paths}: (description follows) %% FS FS CONTINUE
%
%- - - - - - - - - - - - - - - - - - - - - - - - - - - - -
%
\subsection{\lcode{spec2timeKernels}} \label{programs_scripts,sec:bin_prog,sec:spec_time_kernels}
%- - - - - - - - - - - - - - - - - - - - - - - - - - - - -
Compute \lcode{time_kernel} files from existing \lcode{spectral_kernel} files (by inverse Fourier transform) 
for the time windows as specified by options t0,dt,nt1,nt2: The total set of time samples consists of 
\lcode{n} time windows. nt1,nt2 are strings containing \lcode{n} integers defining the start (\lcode{nt1}) 
and end (\lcode{nt2}) each time window by a time sample index. 
Times compute as \lcode{t = t0 + jt*dt} where \lcode{nt1} \(\le\) \lcode{jt} \(\le\) \lcode{nt2}.
%- - - - - - - - - - - - - - - - - - - - - - - - - - - - -
There are two possibele ways to define a set of kernels that are transformed:
\begin{itemize}
\item[(way 1):] compute kernel for only one path, defined by eventID and station name using options \lcode{-evid}, \lcode{-staname}, \lcode{-comp} and \lcode{-param}
\item[(way 2):] use flag \lcode{-dmsapce} in connection with optional range definition of the path index (flags \lcode{-ipath1} \lcode{-ipath2})\\
in order to define a subset of the paths contained in the given data-model-space description. If the upper (lower) limit of the path range is not defined (i.e. \lcode{-ipath1} (\lcode{-ipath2}) not set), the maximum (minimum) possible value is used. Then, all kernels for the defined range of paths in the given data-model-space description are computed.
\end{itemize}
%- - - - - - - - - - - - - - - - - - - - - - - - - - - - -
\subsubsection{positional arguments}
%- - - - - - - - - - - - - - - - - - - - - - - - - - - - -
\paragraph{\lcode{main_parfile}}
Main parameter file of inversion.
%- - - - - - - - - - - - - - - - - - - - - - - - - - - - -
\subsubsection{optional arguments}
%- - - - - - - - - - - - - - - - - - - - - - - - - - - - -
\subsubsection{set of kernels to be transformed}
%- - - - - - - - - - - - - - - - - - - - - - - - - - - - -
\subsubsection{way 1}
%- - - - - - - - - - - - - - - - - - - - - - - - - - - - -
\paragraph{\lcode{-evid} \lcode{event_id}}
Defines the event id of the one path. (must belong to an event in main event list)
%- - - - - - - - - - - - - - - - - - - - - - - - - - - - -
\paragraph{\lcode{-staname} \lcode{station_name}}
Defines the station name of the one path. (must belong to a station in main station list)
%- - - - - - - - - - - - - - - - - - - - - - - - - - - - -
\paragraph{\lcode{-comp "comp_1 ... comp_n"}}
Vector of receiver components for which time kernel should be transformed. For valid components see \myref{basic_steps,sec:data_general}.
%- - - - - - - - - - - - - - - - - - - - - - - - - - - - -
\paragraph{\lcode{-param "param_1 ... param_n"}}
Vector of parameter names for which time kernel should be transformed. Only valid parameter names of the chosen model parametrization are accepted. 
%- - - - - - - - - - - - - - - - - - - - - - - - - - - - -
\subsubsection{way 2}
%- - - - - - - - - - - - - - - - - - - - - - - - - - - - -
\paragraph{\lcode{-dmspace}}
Data model space input file to define a set of paths.
%- - - - - - - - - - - - - - - - - - - - - - - - - - - - -
\paragraph{\lcode{-ipath1} \lcode{path1}}
path1 is the first index of the path loop. By default, \lcode{path1 = 1}
%- - - - - - - - - - - - - - - - - - - - - - - - - - - - -
\paragraph{\lcode{-ipath2} \lcode{path2}}
path2 is the last index of the path loop. By default, \lcode{path2 = max_number_of_paths} as to data-model-space description.
%- - - - - - - - - - - - - - - - - - - - - - - - - - - - -
\paragraph{\lcode{-wp}}
If set, then 'ON-WP' spectral kernel files are produced, containing plain kernel values on wavefield 
points. If not set, normal kernel files (pre-integrated) are transformed.
%- - - - - - - - - - - - - - - - - - - - - - - - - - - - -
\subsubsection{time waveform kernels}
%- - - - - - - - - - - - - - - - - - - - - - - - - - - - -
\paragraph{\lcode{-dt time_step}}
Global time step of time discretization.
%- - - - - - - - - - - - - - - - - - - - - - - - - - - - -
\paragraph{\lcode{-nt1 "idx_1 .. idx_n"}}
Vector of \lcode{n} time indices defining the start indices of the \lcode{n} time windows.
%- - - - - - - - - - - - - - - - - - - - - - - - - - - - -
\paragraph{\lcode{-nt2 "idx_1 .. idx_n"}}
Vector of \lcode{n} time indices defining the start indices of the \lcode{n} time windows.
%- - - - - - - - - - - - - - - - - - - - - - - - - - - - -
\paragraph{\lcode{-t0 tzero}}
Optional global time shift which is added to all times defined by \lcode{dt}, \lcode{nt1}, \lcode{nt2} (default \lcode{t0=0})
%
%- - - - - - - - - - - - - - - - - - - - - - - - - - - - -
%
\subsection{\lcode{timeKernel2vtk}} \label{programs_scripts,sec:bin_prog,sec:timeKernel_2_vtk}
%- - - - - - - - - - - - - - - - - - - - - - - - - - - - -
Produces vtk files from \emph{one existing} binary time sensitivity kernel file for a selection of time steps defined by vectors of starting and end indices \lcode{-nt1} , \lcode{-nt2}

\subsubsection{positional arguments}
%- - - - - - - - - - - - - - - - - - - - - - - - - - - - -
\paragraph{\lcode{main_parfile}}
Main parameter file of inversion.
%- - - - - - - - - - - - - - - - - - - - - - - - - - - - -
\subsubsection{optional arguments}
%- - - - - - - - - - - - - - - - - - - - - - - - - - - - -
\paragraph{\lcode{-evid} \lcode{event_id}}
Defines the event id of the one path. (must belong to an event in main event list)
%- - - - - - - - - - - - - - - - - - - - - - - - - - - - -
\paragraph{\lcode{-staname} \lcode{station_name}}
Defines the station name of the one path. (must belong to a station in main station list)
%- - - - - - - - - - - - - - - - - - - - - - - - - - - - -
\paragraph{\lcode{-comp "comp_1 ... comp_n"}}
Vector of receiver components for which vtk files should be generated. For valid components see \myref{basic_steps,sec:data_general}.
%- - - - - - - - - - - - - - - - - - - - - - - - - - - - -
\paragraph{\lcode{-param "param_1 ... param_n"}}
Vector of parameter names for which vtk files should be generated. Only valid parameter names of the chosen model parametrization are accepted. 
%- - - - - - - - - - - - - - - - - - - - - - - - - - - - -
\paragraph{\lcode{-nt1 "idx_1 .. idx_n"}}
Vector of \lcode{n} time indices defining the start indices of the \lcode{n} time windows within which vtk files 
should be generated (the resulting set of time indices must be contained in the binary time kernel files generated by 
\lcode{spec2timeKernels}.
%- - - - - - - - - - - - - - - - - - - - - - - - - - - - -
\paragraph{\lcode{-nt2 "idx_1 .. idx_n"}}
Vector of \lcode{n} time indices defining the start indices of the \lcode{n} time windows within which vtk files 
should be generated (the resulting set of time indices must be contained in the binary time kernel files generated by 
\lcode{spec2timeKernels}.
%
%- - - - - - - - - - - - - - - - - - - - - - - - - - - - -
%
%++++++++++++++++++++++++++++++++++++++++++++++++++++++++++
\section{Integration Weights} \label{programs_scripts,sec:fmod_intw}
%++++++++++++++++++++++++++++++++++++++++++++++++++++++++++
%
The \ASKI module \lcode{integrationWeights} computes integration weights for the set of wavefield points 
in order to integrate the kernels over the inversion grid. As we need to calculate the integrals of the 
kernels over each inversion grid cell separately, the integration weights are computed for each cell in 
such a way that weighting the summation of the kernel values yields the desired integral value:

For each inversion grid cell $\Omega_c \subset \RRR$ which contains wavefield points \wpG the weights 
\weights are computed such that
\begin{equation} \label{programs_scripts,sec:fmod_intw,eq:integration_global}
\int_{\Omega_c} K(\mathbf{x})\,d\mathbf{x} \simeq \sum_{i=1}^{n_c} w_iK(\mathbf{x}_i)
\end{equation}

There are several types of integration weights supported (indicated by dummy variable \lcode{intw_type} of 
subroutine \lcode{createIntegrationWeights}):
%
%----------------------------------------------------------
\setcounter{subsection}{-1}
\subsection{Compute Average (no integration)} \label{programs_scripts,sec:fmod_intw,sub:average}
%----------------------------------------------------------
%
In case of \lcode{intw_type = 0}, function \lcode{createIntegrationWeights} sets 
\[w_i = \frac{1}{n_c} \quad ,\, i=1,\dots,n_c\]
in each inversion grid cell $\Omega_c$.\\
This way, the summation $\sum_{i=1}^{n_c} w_iK\brackr{\mathbf{x}_i^G} = \frac{1}{n_c} \sum_{i=1}^{n_c} 
K\brackr{\mathbf{x}_i^G}$ yields the average kernel value in $\Omega_c$.

This type of integration weights (which are actually no integration weights) may be used to perform some 
sort of interpolation of kernel values onto the inversion grid (e.g.~in order to compare kernel values 
from different methods which use different sets of wavefield points).
%
%----------------------------------------------------------
\subsection{Scattered Data Integration} \label{programs_scripts,sec:fmod_intw,sub:SDI}
%----------------------------------------------------------
%
In case of \lcode{intw_type = 1...4}, a method by David Levin \cite{Levin99} is apllied to a
standardized inversion grid cell $\Omega^S$. For different shapes of inversion grid cells, different 
types of standard cells are used, which are referred to below.

For each inversion grid cell $\Omega_c \subset \RRR$ containing wavefield points \wpG, a transformation 
$T : \Omega_c \rightarrow \Omega^S$ is used to transform cell $\Omega_c$ into the standard cell $\Omega^S$ 
and to compute the respective transformed wavefield points $\mathbf{x}_i^S = T\left(\mathbf{x}_i\right)$ 
contained in $\Omega^S$.

Then \cite{Levin99} is applied to points \wpS and volume $\Omega^S$ to compute integration weights \weightsS
such that
\begin{align}
\int_{\Omega_c} K(\mathbf{x})\,d\mathbf{x} &= 
           \int_{\Omega^S} K\left(T^{-1}\left(\mathbf{x}^S\right)\right) 
           \mathcal{J}_{T^{-1}}\left(\mathbf{x}^S\right)\,d\mathbf{x}^S \nonumber \\
  &\simeq \sum_{i=1}^{n_c} w^S_i\,\mathcal{J}_i\,K(\mathbf{x}_i) 
   \label{programs_scripts,sec:fmod_intw,eq:integration_standard} \\
  & = \sum_{i=1}^{n_c} w_iK(\mathbf{x}_i) \nonumber
\end{align}
where $\mathcal{J}_{T^{-1}}$ denotes the Jacobian of the inverse transformation $T^{-1}$, $\mathcal{J}_i = 
\mathcal{J}_{T^{-1}}\left(\mathbf{x}^S_i\right)$ and the desired weights compute as $w_i = w^S_i \, \mathcal{J}_i$. 
The method of computing such integration weights \weightsS, as presented in \cite{Levin99}, is explained in 
the following.
%
%----------------------------------------------------------
\subsubsection{The Method of Scattered Data Integration}
%----------------------------------------------------------
%
\cite{Levin99}{} follows a composite rule strategy for building the integration weights. For subsets of the 
volume of interest it constructs integration formulae which are as local and as stable as possible and are 
exact for polyinomials $p$ of a certain fixed degree $m$. It is assumed that the integrals of these polynomials 
$p\in\Pi_m$ over the subsets are easily computable.

In notation of \cite{Levin99}{}, the integration weights $A_i$ for a function $f$ on domain $\Omega\subset\Rd$ 
which is given on a set $\brackg{x_i}_{i=1}^N\subset\Omega$ are constructed as
\[
A_i = \sum_{k=1}^K A_i^{(k)} \,,\quad 1 \le i \le N \, , 
\]
where $\Omega$ is subdivided into $K$ disjoint subsets $E_k$. For each $E_k$, the $N$ weights $A_i^{(k)}$ are 
calculated as follows.

We choose a basis $\brackg{p_i}_{i=1}^J$ of the space $\Pi_m$ of all polynomials in $\Rd$ with maximum total 
degree $m$, where $J = \binom{d+m}{m}$ is the dimension of space $\Pi_m$. 
$A_i^{(k)}$ are then defined as the components $a_i = A_i^{(k)}$ of vector $\bar{a} = 
D^{-1}E\brackr{E^tD^{-1}E}^{-1}\bar{c}$, where
\begin{align*}
  D &= 2\text{Diag}\brackg{\eta\brackr{\|x^\ast-x_1\|},\dots,\eta\brackr{\|x^\ast-x_N\|}}\\
  E_{i,j} &= p_j\brackr{x_i}\,,\quad 1\le i\le N,\; 1\le j\le J
\end{align*}
and $\bar{c}$ contains the integrals of the $p_i$ over $E_k$, i.e.~$c_i = \int_{E_k} p_i$. $\eta(r) = 
\exp\brackr{r^2/h^2}$ is a fast increasing weight function which gives the localizing properties of the weights. 
$h$ is approximately the diameter of subsets $E_k$ and $x^\ast$ is some center of $E_k$.

This composite local approach of calculating global integration weights involves $K$ solutions of a full 
linear system of order $J$. 
%
%----------------------------------------------------------
\subsubsection{Application to Hexahedral Inversion Grid Cells}
%----------------------------------------------------------
%
For inversion grid cells of general hexahedral shape, the 3-dimensional cube 
\[\Omega^S = [-1,1]^3 = \brackg{\left. \vecthree x y z \right| -1 \le x,y,z \le 1}\]
is used as the standard cell. For every such inversion grid cell $\Omega_c$, module \lcode{inversionGrid} 
is expected to provide its transformed wavefield points \wpS and their corresponding values of Jacobian 
$\mathcal{J}_i$.

In the context of Scattered Data Integration, the inversion domain $\Omega=\Omega^S=[-1,1]^3$ is subdivided into
$K = n_h^3$ subcubes $E_k$ of edge length $h=2/n_h$. $n_h=\max\brackg{\left\lfloor\sqrt[3]{\frac{n_c}{J}}\right\rfloor,1}$ 
is chosen in such a way that there should be at least $J$ (or all, otherwise) integration points within $E_k$, as 
otherwise the damping by matrix $D^{-1}$ might cause numerical instabilities by making matrix $E^tD^{-1}E$ close 
to singular. 

As $x^\ast$, the center of the respective subcube is chosen.

The desired weights \weightsS are then given by $w^S_i = A_i,\,1\le i\le n_c$
%
%----------------------------------------------------------
\subsubsection{Application to Tetrahedral Inversion Grid Cells}
%----------------------------------------------------------
%
For inversion grid cells of general tetrahedral shape, the 3-dimensional simplex with corners
\[
\vecthree 0 0 0 ,\, \vecthree 1 0 0 ,\, \vecthree 0 1 0 ,\, \vecthree 0 0 1
\]
is used as the standard cell $\Omega^S$. For every such inversion grid cell $\Omega_c$, module 
\lcode{inversionGrid} is expected to provide its transformed wavefield points \wpS and their 
corresponding values of Jacobian $\mathcal{J}_i$.

In the context of Scattered Data Integration, here the inversion domain $\Omega = \Omega^S$ is \emph{not} 
subdivided into any true subsets $E_k$. It is always $K=1$ and $E_1 = \Omega$, mainly because a subdivision 
of the standard tetrahedron is not trivial (compared with e.g.\ the cube $[-1,1]^3$), considering that the integrals 
of the base polynomials must be computed over all subsets $E_k$. 

As $x^\ast$, the barycenter \[ \vecthree{0.25}{0.25}{0.25} \] of the standard simplex is chosen and $h = 1$.

The desired weights \weightsS are then given by $w^S_i = A_i,\,1\le i\le n_c$
%
%----------------------------------------------------------
\subsubsection{Scattered Data Integration, Order 1}
%----------------------------------------------------------
%
\lcode{intw_type = 1}\\
In the context of this subsection \ref{programs_scripts,sec:fmod_intw,sub:SDI}, $m=1$ is used as the 
degree of polynomials which are integrated in an exact way and of course $d=3$. The space $\Pi_1$ of all polynomials in 
$\RRR$ of maximum total degree $m=1$ has dimension $J = \binom{3+m}{m} = \binom{4}{1} = 4$. As a basis of $\Pi_1$ we choose 
$\big\{1,\allowbreak x,\allowbreak y,\allowbreak z\big\}$.
%
%----------------------------------------------------------
\subsubsection{Scattered Data Integration, Order 2}
%----------------------------------------------------------
%
\lcode{intw_type = 2}\\
In the context of this subsection \ref{programs_scripts,sec:fmod_intw,sub:SDI}, $m=2$ is used as the 
degree of polynomials which are integrated in an exact way and of course $d=3$. The space $\Pi_2$ of all polynomials in 
$\RRR$ of maximum total degree $m=2$ has dimension $J = \binom{3+m}{m} = \binom{5}{2} = 10$. As a basis of $\Pi_2$ we choose $\big\{1,
\allowbreak x,\allowbreak y,\allowbreak z,\allowbreak x^2,\allowbreak xy,\allowbreak xz,\allowbreak y^2,
\allowbreak yz,\allowbreak  z^2\big\}$.
%
%----------------------------------------------------------
\subsubsection{Scattered Data Integration, Order 3}
%----------------------------------------------------------
%
\lcode{intw_type = 3}\\
In the context of this subsection \ref{programs_scripts,sec:fmod_intw,sub:SDI}, $m=3$ is used as the 
degree of polynomials which are integrated in an exact way and of course $d=3$. The space $\Pi_3$ of all polynomials in 
$\RRR$ of maximum total degree $m=3$ has dimension $J = \binom{3+m}{m} = \binom{6}{3} = 20$. As a basis of $\Pi_3$ we choose $\big\{1,
\allowbreak x,\allowbreak y,\allowbreak z,\allowbreak x^2,\allowbreak xy,\allowbreak xz,\allowbreak y^2,
\allowbreak yz,\allowbreak  z^2,\allowbreak x^3,\allowbreak  x^2y,\allowbreak  x^2z,\allowbreak  xy^2,
\allowbreak  xyz,\allowbreak  xz^2,\allowbreak  y^3,\allowbreak  y^2z,\allowbreak  yz^2,\allowbreak  
z^3\big\}$.
%
%----------------------------------------------------------
\subsubsection{Scattered Data Integration, Optimal Order}
%----------------------------------------------------------
%
In case of \lcode{intw_type = 4}, function \lcode{createIntegrationWeights} tries to seperately 
find for each inversion grid cell the highest possible order of Scattered Data Integration. 
Starting with highest order $m=3$, it continues to recompute Scattered Data Integration weights of 
order $m = 2$ and $m = 1$ until the computation was successful. If the computation for order $m=1$ 
fails, the integration weights of that cell will be marked erroneous, the computation of the weights
is not successfull in that case.

As the success of the Scattered Data Integration method is strongly dependent on the specific set of 
points \wpS, since matrix $E_{i,j} = p_j\brackr{x_i}$ must have full rank, the strategy of choosing 
the highest possible degree of integration for each cell tries to take all locally availabe information 
of inversion grid and wavefield points into account.
%
%----------------------------------------------------------
\subsection{Linear (first order) Integration} \label{programs_scripts,sec:fmod_intw,sec:linear}
%----------------------------------------------------------
%
In case of \lcode{intw_type = 5}, function \lcode{createIntegrationWeights} sets 
\[w_i = \frac{1}{n_c}\mathrm{vol}\left(\Omega_c\right) \quad ,\, i=1,\dots,n_c\]
in each inversion grid cell $\Omega_c$, where $\mathrm{vol}\left(\Omega_c\right)$ denotes the 
volume of inversion grid cell $\Omega_c$, which is expected to be provided by module \lcode{inversionGrid}
for every cell.\\
This way, the summation $\sum_{i=1}^{n_c} w_iK\brackr{\mathbf{x}_i^G} = 
\mathrm{vol}\left(\Omega_c\right)\frac{1}{n_c} \sum_{i=1}^{n_c} K\brackr{\mathbf{x}_i^G}$ yields 
the average kernel value in $\Omega_c$ multiplied with the volume of $\Omega_c$.

This somehow approximates the generalization of the trapezoidal rule to 3 dimensions, in which the 
integral of a function $f$ over some tetrahedron $\mathcal{T}$, which is defined by 4 incoplanar points 
$\mathbf{t}_1,\dots,\mathbf{t}_4$, is computed by $\mathrm{vol}\left(\mathcal{T}\right)\frac{1}{4} 
\sum_{i=1}^4 f(\mathbf{t}_i)$. 
%
%----------------------------------------------------------
\subsection{External Integration Weights} \label{programs_scripts,sec:fmod_intw,sec:external}
%----------------------------------------------------------
%
In case of \lcode{intw_type = 6}, function \lcode{createIntegrationWeights} does not actually compute
any integration weights. Instead, it calls function \lcode{transformToStandardCellInversionGrid} of module 
\lcode{inversionGrid} with dummy variable \lcode{type_standard_cell} set to value \lcode{-1}, which 
requests the routine to return the total integration weights in variable \lcode{jacobian} instead the 
jacobian values. These returned values are then stored as the integration weights.

This functionality must be supported by the type of inversion grid. At the moment only inversion grids 
of type \lcode{specfem3dInversionGrid} support external type integration weights.
%----------------------------------------------------------
%      

%
%-------------------------------
% BIBLIOGRAPHY
\bibliographystyle{alpha}
\bibliography{bibliography}
\phantomsection  % so hyperref creates bookmarks
\addcontentsline{toc}{chapter}{Bibliography}
% in order to create a bibliography from file bibliography.bib run:
% > pdflatex manual
% > bibtex manual
% > pdflatex manual
% > pdflatex manual
%
%
%-------------------------------
% CHAPTER History (of the document versions, as required by the GNU Free Documentation License)
\chapter*{History}
\phantomsection  % so hyperref creates bookmarks
\addcontentsline{toc}{chapter}{History}
% -*-LaTex-*-

%-----------------------------------------------------------------------------
%   Copyright 2016 Florian Schumacher
%
%   This file is part of the ASKI manual as a LaTeX document with main file
%   manual.tex
%
%   Permission is granted to copy, distribute and/or modify this document
%   under the terms of the GNU Free Documentation License, Version 1.3
%   or any later version published by the Free Software Foundation;
%   with no Invariant Sections, no Front-Cover Texts, and no Back-Cover Texts.
%   A copy of the license is included in the section entitled ``GNU
%   Free Documentation License''. 
%-----------------------------------------------------------------------------
%
This is a section on the history of this document, i.e.\ its previously published versions, as referred to by
the GNU Free Documentation License (version 1.3).

%-------------------------------
% Developer's Manual for ASKI version 1.2, August 2016
\subsection*{ASKI Developer's Manual, ASKI version 1.2, August 2016}

Recognizable snippet from the title page (scaled):

\hspace*{0.5cm}\begin{minipage}{0.5\textwidth}
  \begin{flushright} \tt
      {\LARGE Developer's Manual} \\[1em]
      {\large ASKI} {\rm --} version 1.2 \\[1em]
      Aug 2016 \\
      Florian Schumacher \\ 
      {\footnotesize Ruhr-Universit\"at Bochum, Germany}
  \end{flushright}
\end{minipage}
\vspace*{1ex}

{\bf Title: } ASKI Developer's Manual, ASKI version 1.2, August 2016\\
{\bf Year: } 2016\\
{\bf Authors: } Florian Schumacher (Ruhr-Universit\"at Bochum, Germany)

This version of this document is provided for download (as of August 2016) at\\
\url{https://github.com/seismology-RUB/ASKI/releases/tag/v1.2}\\
being contained in the provided \lcode{.zip} code archive as file \\
\lcode{ASKI-1.2/devel/doc/ASKI_developers_manual_1-2_aug-2016.pdf}\\
Direct link (as of August 2016):\\
\url{https://github.com/seismology-RUB/ASKI/archive/v1.2.zip}



%
%-------------------------------
% CHAPTER GNU Free Documentation License
\input{fdl-1.3}
%

\end{document}
