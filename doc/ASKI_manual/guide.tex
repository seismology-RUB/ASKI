%-----------------------------------------------------------------------------
%   Copyright 2013 Florian Schumacher
%
%   This file is part of the ASKI manual as a LaTeX document with main file
%   manual.tex
%
%   Permission is granted to copy, distribute and/or modify this document
%   under the terms of the GNU Free Documentation License, Version 1.3
%   or any later version published by the Free Software Foundation;
%   with no Invariant Sections, no Front-Cover Texts, and no Back-Cover Texts.
%   A copy of the license is included in the section entitled ``GNU
%   Free Documentation License''. 
%-----------------------------------------------------------------------------
%
This chapter is intended to guide you, dependent on what you want to do, through all necessary steps
to achieve your goals.

If you don't know about \ASKI yet, we recommend you to quickly read through the next section, which 
explains some basic terminology in \ASKI and the concepts it is based on.

The sections below address possible operations you can conduct with \ASKI. For every operation, we only refer 
to the necessary basic steps (by $\rightarrow$), which are described in chapter~\myref{basic_steps}.\\
Make sure to read through a complete item before hastily doing anything!

Good Luck!
%
%
%++++++++++++++++++++++++++++++++++++++++++++++++++++++++++
\section*{What is \ASKI?} \label{guide,sec:ASKI}
\phantomsection  % so hyperref creates bookmarks
\addcontentsline{toc}{section}{What is \ASKI?}
%++++++++++++++++++++++++++++++++++++++++++++++++++++++++++
%
%
\ASKI is a modularized software package which offers analysis tools of seismic data and a full waveform
inversion concept based on waveform sensitivity kernels derived from Born scattering theory. 

Instead of using time-dependent values of ground motion (i.e.\ samples of a time-series of seismic data), 
\ASKI uses freqency-dependent complex values of ground motion at a certain receiver excited by a certain 
seismic source. This, mainly, has reasons of computational feasibility and does not mean any draw-back 
(e.g.\ in the sense that no time-windowing is possible etc.), since \ASKI aims at taking into account
\emph{all} available information contained in a waveform.

Using sensitivity kernels $K$, change in a data sample $\Delta d_i$ is connected to model uptdates $\Delta m$ by 
an integral relation $\Delta d_i = \int_{\Omega} \Delta \vec{m} \, \vec{K}_i$. In order to build a linear 
system, the model update $\Delta \vec{m}$ is assumed to be constant throughout small scattering volumes 
$\Omega_j$, where $\Omega = \overset{\centerdot}{\bigcup}_j \Omega_j$. These volumes constitute the cells of the volumetric 
inversion grid and the sensitivity matrix contains entries of preintegrated kernels $\int_{\Omega_j} K_i$.

The sensitivity kernels $K$ are computed from forward wavefields produced on a set of points in space, which
is dependent on the particular forward method. This set of points is refered to as \emph{wavefield points}. The 
wavefields are written to file, by the respective forward method, which may require very large discspace. 
Providing methods for constructing integration rules for arbitrary point sets contained in arbitrary volumes, 
\ASKI computes integration weights for integration of functions on the wavefield points over the volumetric
cells of the inversion grid. Thereby, the inversion grid takes care of the localization of wavefield points 
inside the inversion grid cells and, if requested, the transformation of cells to a hexahedral (or tetrahedral) standard 
cell for the computation of the integration weights. Hence, some combinations of wavefield points (i.e.\ 
forward methods), integration weight types and inversion grid types are not possible. 

The preintegrated kernel values are also written to files, which may be flexibly read in for arbitrary subsets of data
by the binary programs conducting any sensitivity analysis or an interation step in the iterative full waveform inversion.
Those tools work on the sensitivity matrix, which in the FWI is used in a linear system of equations which
relates a data residuum $\Delta d_i$ to a model update $\Delta m$. After updating the model, wavefields may be computed
with respect to the new model, which again may be improved in the same way, possibly using higher frequencies and 
smaller scatteres. 

Any details on \ASKI and the theory behind, may be found in the near future in respective journal publications. 


%
%
%++++++++++++++++++++++++++++++++++++++++++++++++++++++++++
\newpage
\section*{Time-Domain Sensitivity Kernels} \label{guide,sec:time_kernels}
\phantomsection  % so hyperref creates bookmarks
\addcontentsline{toc}{section}{Time-Domain Sensitivity Kernels}
%++++++++++++++++++++++++++++++++++++++++++++++++++++++++++
%
%
This section describes, how to compute time-domain waveform sensitivity kernels for a specific set 
of sources receivers with respect to a certain background earth model as an operation seperate of any
other \ASKI operations. The kernels in time-domain are much more intuitive to look at for human beings, 
than the standard frequency-domain sensitivity kernels. You may, as well, compute time-domain sensitivity 
kernels from the kernels produced in any iteration step of a full waveform inversion 
(page \pageref{guide,sec:classic_inversion}). For this purpose, apply the steps 
``Transforming to Time-Domain Sensitivity Kernels'' (below) after you computed the standard kernels in 
you iteration step, as the time-domain waveform kernels are produced by an inverse Fourier transform 
from the standard frequency-domain waveform sensitivity kernels on which \ASKI is based.

Please do not get confused by the general terminology of \emph{inversion} and \emph{iteration}, etc. 
Technically you will be conducting an incomplete first iteration step of a full waveform inversion, 
using all the program infrastructure which is also used for a full waveform inversion. 

In addition to \ASKI \myaref{basic_steps,sec:install_ASKI}, you will need software to solve the forward 
problem \myaref{basic_steps,sec:forward_problem}.
%
%----------------------------------------------------------
\subsection*{Preliminary Considerations}
%----------------------------------------------------------
%
\begin{itemize}
\item Create a main parameter file (e.g.\ in the parent directory of your specific inversion directory, 
or where you collect main parameter files for all your inversion projects or analyses)
\myaref{basic_steps,sec:main_parfile}. You will need this file as an input argument to almost all programs/scripts.\\
Set \lcode{MAIN_PATH_INVERSION} to a correct value.
The directory does not need to exist yet, if not, then it will be created. \\
Set \lcode{ITERATION_STEP_PATH} and 
\lcode{PARFILE_ITERATION_STEP} to desired values, 
or leave the default values, if present.\\
All other parameters can be adjusted later. 
%
\item Create a directory structure for only one iteration step \myaref{basic_steps,sec:create_dir}
%
\item Even if you do not have any measured data, it might still be beneficial for you to make yourself 
(roughly) familiar with the from of data used in \ASKI \myaref{basic_steps,sec:data_general}. \\
In your main parameter file, set the following values: 
  \begin{itemize}
  \item Set \lcode{FILE_EVENT_LIST} and \lcode{FILE_STATION_LIST} to define the sources and receivers which are
    involved in the paths that you would like to compute the kernels for. 
  \item Dependent on the (length of the) time series you want to deal with, define the frequency discretization of the
    spectral kernels that will be produced first. This must be done by \lcode{MEASURED_DATA_FREQUENCY_STEP}, 
    \lcode{MEASURED_DATA_NUMBER_OF_FREQ} and \lcode{MEASURED_DATA_INDEX_OF_FREQ}
  \end{itemize}
In general, for the pure kernel computation you do not need any measured data. So, here you do not need to prepare
data in the \ASKI required form.
%
\item Set \lcode{PATH_EVENT_FILTER} and \lcode{PATH_STATION_FILTER} in your main parameter file. The transformation 
  of the standard frequency-domain sensitivity kernels to the time domain, \emph{always} requires 
  event filters (i.e.\ source-time functions) and station filters (i.e.\ receiver responses). Even if you do not want to 
  apply those values, you need to artificially create the required files in pahts \lcode{PATH_EVENT_FILTER} and 
  \lcode{PATH_STATION_FILTER} and may set the spectral filter values to the real value $1\in\mathbb{R}$, i.e.\ 
  $(1\;0)\in\mathbb{C}$ for all frequencies.
%
\item Set \lcode{FORWARD_METHOD} in your main parameter file to the value of your choice. 
%
\item Choose a model parametrization by setting \lcode{MODEL_PARAMETRIZATION} in the main parameter file 
to a value of your choice (which is supported by your forward method)
%
\item Set \lcode{CURRENT_ITERATION_STEP} in your main parameter file to value \lcode{1}, as you are technically
starting to conduct the first (and only) iteration step of a full waveform inversion
%
\item Define the inversion grid \myaref{basic_steps,sec:invgrid}, which controls the spacial volumetric discretization 
(resolution) of the computed sensitivity kernels. In case of just computing (time) kernels to look at, it is not
crucial to regard this resolution as the resolution of some inverted model, as no inversion will be conducted on the 
inversion grid. 
%
\item Set all parameters in the specific iteration step parameter file to correct values 
\myaref{basic_steps,sec:iter_parfile}, including the correct reference to the inversion grid. Set 
\lcode{ITERATION_STEP_NUMBER_OF_FREQ} and \lcode{ITERATION_STEP_INDEX_OF_FREQ} to the \emph{same} values as the
\lcode{MEASURED_DATA_NUMBER_OF_FREQ} and \lcode{MEASURED_DATA_INDEX_OF_FREQ} in the main parameter file!
Refer to the documentation of your forward method on how to set filenames \lcode{FILE_WAVEFIELD_POINTS}
and \lcode{FILE_KERNEL_REFERENCE_MODEL} (as the handling of these file are method dependent).
%
\item Dependent on your method and model parametrization, define your background model with respect to which the
kernels will be computed. If you have some inverted model file, use \myaref{basic_steps,sec:export_kim}. For defining
a starting model, see \myaref{basic_steps,sec:start_model}.
%
\end{itemize}
%
%----------------------------------------------------------
\subsection*{Computing Standard Frequency-Domain Sensitivity Kernels}
%----------------------------------------------------------
%
\begin{itemize}
\item Compute forward wavefields and Green tensors w.r.t.\ the current model 
by your method. Refer to the respective documentation of your method.\\
After that, you may prepare the synthetic data in the way \ASKI expects it (see sections~\myref{basic_steps,sec:data_general} 
and~\myref{files,sec:synth_data}). Refer to the documentation of your method on how to do it.\\
%
\item Set filename \lcode{FILE_INTEGRATION_WEIGHTS} in your main parameter file (can be any name, will be created), 
as well as \lcode{TYPE_INTEGRATION_WEIGHTS} \myaref{basic_steps,sec:intw}
%
\item Initiate basic requirements for all programs and scripts \myaref{basic_steps,sec:initBasics} 
%
\item If you have many paths, you may define a data and model space concentrating on defining paths 
  \myaref{basic_steps,sec:dmspace}
  If you have only one path or just a few, it possible (and propably also convenient) to just continue to the
  computation of the kernels.
%
\item Compute the standard frequency-domain sensitivity kernels for your specific set of paths (or the one or 
  few paths, one after another) and your set of model parameters \myaref{basic_steps,sec:compute_kernels} \\
  If desired, you may have a look at the standard frequency-domain kernels \myaref{basic_steps,sec:plot_kernels}
\end{itemize}
%
%----------------------------------------------------------
\subsection*{Transforming to Time-Domain Sensitivity Kernels}
%----------------------------------------------------------
%
\begin{itemize}
\item Transform the standard frequency-domain waveform kernels to time domain \myaref{basic_steps,sec:compute_time_kernels}.
  Note that the transformation \emph{always} requires event filters (i.e.\ source-time functions) and station filters 
  (i.e.\ receiver responses), which in case they are not required may artificially all have real value $1\in\mathbb{R}$, i.e.\ 
  $(1\;0)\in\mathbb{C}$ for all frequencies.
%
\item Plot the time kernels \myaref{basic_steps,sec:plot_time_kernels}
\end{itemize}
%
%
%++++++++++++++++++++++++++++++++++++++++++++++++++++++++++
\newpage
\section*{Full Waveform Inversion - Classical Waveform Sensitivity Kernels} \label{guide,sec:classic_inversion}
\phantomsection  % so hyperref creates bookmarks
\addcontentsline{toc}{section}{Full Waveform Inversion - Classical Waveform Sensitivity Kernels}
%++++++++++++++++++++++++++++++++++++++++++++++++++++++++++
%
%
%here short introduction as to what ``Full Waveform Sensitivity Kernel Inversion'' actually is, followed by
%the steps to do:

Iterative inversion scheme which uses waveform sensitivity kernels to gain model updates form data residua.

In addition to \ASKI \myaref{basic_steps,sec:install_ASKI}, you will need software to solve the forward 
problem \myaref{basic_steps,sec:forward_problem}.
%
%----------------------------------------------------------
\subsection*{Before The First Iteration Step}
%----------------------------------------------------------
%
\begin{itemize}
\item Create a main parameter file (e.g.\ in the parent directory of your specific inversion directory, 
or where you collect main parameter files for all your inversion projects)
\myaref{basic_steps,sec:main_parfile}. You will need this file as an input argument to almost all programs/scripts.\\
Set \lcode{MAIN_PATH_INVERSION} to a correct value. 
The directory does not need to exist yet, if not, then it will be created. \\
Set \lcode{ITERATION_STEP_PATH} and 
\lcode{PARFILE_ITERATION_STEP} to desired values, 
or leave the default values, if present.\\
All other parameters can be adjusted later. 
%
\item Create a directory structure for the expected number of iteration steps of your inversion 
\myaref{basic_steps,sec:create_dir}
%
\item Make yourself familiar with the from of data used in \ASKI \myaref{basic_steps,sec:data_general}. \\
  Set \lcode{PATH_MEASURED_DATA}, \lcode{PATH_EVENT_FILTER}, \lcode{PATH_STATION_FILTER}, 
  \lcode{FILE_EVENT_LIST}, and \lcode{FILE_STATION_LIST}, as well as \lcode{MEASURED_DATA_FREQUENCY_STEP}, 
  \lcode{MEASURED_DATA_NUMBER_OF_FREQ} and \lcode{MEASURED_DATA_INDEX_OF_FREQ}  in your main 
  parameter file, before preparing your data in the required form \myaref{basic_steps,sec:measured_data}
%
\item Set \lcode{FORWARD_METHOD} in your main 
parameter file to the value of your choice. If you want to use different methods in the course of 
inverting one dataset (e.g.\ starting with a 1D method, continuing with a 3D method), then it may make 
sense to create a different directory structure for each method and using the final model of one method
as the starting model for the next method.
%
\item Choose a model parametrization by setting \lcode{MODEL_PARAMETRIZATION} in the main parameter file 
to a value of your choice (which is supported by your forward method)
%
\end{itemize}
%
%----------------------------------------------------------
\subsection*{Before Each Iteration Step (including the first one)}
%----------------------------------------------------------
%
\begin{itemize}
\item Set \lcode{CURRENT_ITERATION_STEP}
in your main parameter file to the correct value. When continuing your inversion with a different method, you
may also keep the current iteration step index (in order for you not to get confused) and leave 
subdirectories of your \lcode{MAIN_PATH_INVERSION} empty (or delete them after creation if they 
are not needed):\\
e.g.\ an inversion with one method could start with iteration step 4 (and respective subdirectory), 
if you have already conducted 3 iteration steps with other methods.
%
\item Define the inversion grid of the current iteration \myaref{basic_steps,sec:invgrid}
%
\item Set all parameters in the specific iteration step parameter file to correct values 
\myaref{basic_steps,sec:iter_parfile}, including the correct reference to your inversion grid.
Refer to the documentation of your forward method on how to set filenames \lcode{FILE_WAVEFIELD_POINTS}
and \lcode{FILE_KERNEL_REFERENCE_MODEL} (as the handling of these file are method dependent).
%
\item Dependent on your method and model parametrization, take care about communicating the current
model (inverted in the previous iteration) to your forward method \myaref{basic_steps,sec:export_kim}. 
Before the first iteration, however, you need to define some starting model \myaref{basic_steps,sec:start_model}.
%
\end{itemize}
%
%----------------------------------------------------------
\subsection*{Conducting An Iteration Step}
%----------------------------------------------------------
%
\begin{itemize}
\item Compute forward wavefields and Green tensors w.r.t.\ the current model 
by your method. Refer to the respective documentation of your method.\\
After that, you may prepare the synthetic data in the way \ASKI expects it (see sections~\myref{basic_steps,sec:data_general} 
and~\myref{files,sec:synth_data}). Refer to the documentation of your method on how to do it.\\
%
\item Set filename \lcode{FILE_INTEGRATION_WEIGHTS} in your main parameter file (can be any name, will be created), 
as well as \lcode{TYPE_INTEGRATION_WEIGHTS} \myaref{basic_steps,sec:intw}
%
\item Initiate basic requirements for all programs and scripts \myaref{basic_steps,sec:initBasics} 
%
\item Define data and model space, whereby the paths are mainly important for now \myaref{basic_steps,sec:dmspace}
%
\item Compute sensitivity kernels for your specific set of paths and your set of model parameters 
  \myaref{basic_steps,sec:compute_kernels} \\
  If desired, you may have a look at your kernels \myaref{basic_steps,sec:plot_kernels}
%
\item Choose a specific data and model space. You may well play around with different subsets of data or
smoothing (next step) in the course of inverting for different models
%
\item Finally compute the inverted model by solving the kernel system \myaref{basic_steps,sec:solve_kernel_system}
\end{itemize}



%%% Local Variables:
%%% mode: latex
%%% TeX-master: "manual"
%%% End:
